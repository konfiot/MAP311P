\documentclass[a4paper,12pt,twoside]{article}
\usepackage[T1]{fontenc}
\usepackage[utf8]{inputenc}
\usepackage[french]{babel}
\usepackage[titlepage,fancysections,pagenumber]{polytechnique}
\usepackage{amsfonts} \usepackage{amsmath} \usepackage{graphicx} 
\usepackage{pgf}
\usepackage{pgf,tikz}
\usepackage{mathrsfs}
\usetikzlibrary{arrows}
\usetikzlibrary[patterns]

\newcommand{\p}{\mathbb{P}}
\title{Rapport de projet python MAP311}
\subtitle{Enveloppes convexes aléatoires}
\author{David Cheikhi et Arthur Toussaint}

\begin{document}

\maketitle

\section{Des polygones engendrés par l'enveloppe convexe de $n$ points aléatoires}
\section{Une borne théorique inférieure}
	\begin{enumerate}
		\item Je sais pas trop si la réponse attendue est une réponse intuitive ou une vraie démonstration de probas
		\item \label{extr} Si P est extrémal, alors P est nécessairement un des sommets du polygone, on a donc $C_n \subset B_n$ ce qui implique que $\p(B_n) \geq \p(C_n)$
		\item En introduisant $S(r)$ l'aire du cercle de rayon $r$, on à
		\begin{eqnarray}
			\p(C_n) &=& \int_0^1{\p(C_n | R = r) \p(r\leq R \leq r + dr)} \\
				&=& \int_0^1{\p(P_1 \not\subset S_p \cap \ldots \cap P_{n-1} \not\subset S_p)\p(r\leq R \leq r + dr)} \\
				&=& \int_0^1{\p(P_1 \not\subset S_p) \ldots \p(P_{n-1} \not\subset S_p)\p(r\leq R \leq r + dr)} \\
				&=& \int_0^1{\p(P_1 \not\subset S_p)^{n-1}\p(r\leq R \leq r + dr)} \\
				&=& \int_0^1{(1 - \p(P_1 \subset S_p))^{n-1}\p(r\leq R \leq r + dr)} \\
				&=& \int_0^1{\left( 1 - \frac{g(r)}{\pi}\right) ^{n-1}\p(r\leq R \leq r + dr)} \\
				&=& \int_0^1{\left( 1 - \frac{g(r)}{\pi}\right) ^{n-1}\frac{S(r + dr) - S(r)}{S(1)}} \\
				&=& \int_0^1{\left( 1 - \frac{g(r)}{\pi}\right) ^{n-1}\frac{\frac{dS}{dr}dr}{\pi}} \\
				&=& \int_0^1{\left( 1 - \frac{g(r)}{\pi}\right) ^{n-1}\frac{2\pi r dr}{\pi}} \\
				&=& \int_0^1{\left( 1 - \frac{g(r)}{\pi}\right) ^{n-1}2 r dr} \\
		\end{eqnarray}
		\item
			\begin{figure}
			\centering
			\caption{Schéma Q4}
			\label{fig:q4}
			\definecolor{ffqqqq}{rgb}{1.,0.,0.}
			\definecolor{qqttff}{rgb}{0.,0.2,1.}
			\definecolor{uququq}{rgb}{0.25098039215686274,0.25098039215686274,0.25098039215686274}
			\begin{tikzpicture}[line cap=round,line join=round,>=triangle 45,x=5.0cm,y=5.0cm]
			\clip(-1.2,-1.2) rectangle (1.2,1.2);
			\fill[line width=0.pt,color=ffqqqq,fill=ffqqqq,pattern=north east lines,pattern color=ffqqqq] (-0.4358898943540673,0.) -- (-0.4358898943540673,0.9) -- (0.43588989435406733,0.9) -- (0.43588989435406733,0.) -- cycle;
			\draw [line width=0.8pt] (0.,0.) circle (5.cm);
			\draw [line width=0.8pt,dash pattern=on 2pt off 2pt] (-0.4358898943540673,0.9)-- (0.43588989435406733,0.9);
			\draw [line width=0.8pt,domain=-1.2:1.2] plot(\x,{(-0.-0.*\x)/1.});
			\draw [line width=0.8pt,dash pattern=on 2pt off 2pt] (-0.4358898943540673,0.)-- (-0.4358898943540673,0.9);
			\draw [line width=0.8pt,dash pattern=on 2pt off 2pt] (0.43588989435406733,0.9)-- (0.43588989435406733,0.);
			\draw [shift={(0.,0.)},line width=0.8pt,color=qqttff,fill=qqttff,pattern=north east lines,pattern color=qqttff]  plot[domain=1.1197695149986342:2.0218231385911594,variable=\t]({1.*1.*cos(\t r)+0.*1.*sin(\t r)},{0.*1.*cos(\t r)+1.*1.*sin(\t r)}) -- cycle ;
			\begin{scriptsize}
			\draw [fill=uququq] (-0.4358898943540673,0.9) circle (1.5pt);
			\draw[color=uququq] (-0.4117690554175289,0.9488112808172411) node {$C$};
			\draw [fill=uququq] (0.43588989435406733,0.9) circle (1.5pt);
			\draw[color=uququq] (0.4585375073878903,0.9488112808172411) node {$D$};
			\draw [fill=uququq] (-0.4358898943540673,0.) circle (1.5pt);
			\draw[color=uququq] (-0.4717690554175292,0.04826240501858345) node {$E$};
			\draw [fill=uququq] (0.43588989435406733,0.) circle (1.5pt);
			\draw[color=uququq] (0.4685375073878903,0.04826240501858345) node {$F$};
			\draw[color=qqttff] (-0.04046065700054505,1.0916222032853118) node {$S_p$};
			\draw[color=ffqqqq] (0.51666371198668324,0.49013620041979045) node {$S_b$};
			\end{scriptsize}
			\end{tikzpicture}
			\end{figure}
			On cherche tout d'abord les bornes de l'intervalle d'integration. On utilise pour cela la Figure \ref{fig:q4}

			Pour trouver l'aire voulue, on doit integrer entre $x_c$ et $x_d$, ces points sont les points d'intersection entre la droite d'équation $y = r = 1 - s$ et le cercle d'équation $x^2 + y^2 = 1$
			$$ \sqrt{1 - x^2} = y = 1 - s $$ donc 
			\begin{eqnarray}
				x^2	&=& (1 - s)^2 + 1\\
					&=& 1 + 2s - s^2 + 1\\
					&=& 2s - s^2
			\end{eqnarray} 
			donc $x = \pm \sqrt{2s - s^2}$

			On cherche ensuite à déterminer $S_p$, on calcule donc $S_p + S_b - S_b$
			Ainsi, \begin{eqnarray}
				h(s)	&=& S_p \\
					&=& S_p + S_b - S_b \\
					&=& \int^{\sqrt{2s - s^2}}_{-\sqrt{2s - s^2}}{\sqrt{1-x^2}dx} - \int^{\sqrt{2s - s^2}}_{-\sqrt{2s - s^2}}{(1 - s) dx} \\
					&=& \int^{\sqrt{2s - s^2}}_{-\sqrt{2s - s^2}}{(s + \sqrt{1-x^2} - 1) dx}
			\end{eqnarray}

		\item on a $$\sqrt{1 - x^2} = 1 - \frac{x^2}{2} + o(x^2)$$
		donc
		\begin{eqnarray}
			h(s)	&=& \int^{\sqrt{2s - s^2}}_{-\sqrt{2s - s^2}}{(s - \frac{x^2}{2} + o(x^2)) dx} \\
				&=& 2s^{3/2}\sqrt{2-s} - \frac{1}{3}(2s - s^2)^{3/2} + o(2s-s^2)^3 \\\text{Vrai car $ \lim_{x \to 0} 2s - s^2 = 0$}\\ % TODO : Rajouter que c'est vrai car x tend vers 0 quand 2s^2 - s^2 tend vers 0
				&\sim& s^{3/2} (2\sqrt{2-s} - \frac{1}{3}(2 - s)^{3/2}) \\
				&\sim& s^{3/2} (2\sqrt{2} - \frac{1}{3}\sqrt{8}) \\
				&\sim& s^{3/2} \sqrt{2}(2 - \frac{1}{3}\sqrt{4}) \\
				&\sim& s^{3/2} 2\sqrt{2}(1 - \frac{1}{3}) \\
				&\sim& s^{3/2} \frac{4\sqrt{2}}{3} \\
		\end{eqnarray}
		\item	Si $h(s) << \frac{1}{n}$, $\left(1 - \frac{h(s)}{\pi}\right)^{n-1}$ tend vers $0$, et si $h(s) >> \frac{1}{n}$, cette quantité diverge, il faut donc avoir $h(s) \sim \frac{1}{n}$ ce qui implique que $s \sim un^{-2/3}$% TODO ; Mieux démontrer ça

		\item Dans ce cas, on a bien $\left(1 - \frac{h(s)}{\pi}\right)^{n-1}$ qui tend vers $e^{-\frac{K}{\pi}u^{3/2}}$

		De plus, on a bient $2(1-s) \sim 2n^{-2/3}$

		\end{enumerate}

\section{Simulations}
	\subsection{Tirer des points au hasard dans le disque unité}
		On observe des différences dans la distribution des points. On voir que dans le premier cas, la densité des points augmente au fur et à mesure que l'on se rapproche du centre. En effet, on comprend intuitivement que l'on a autant de chances qu'un point se retrouve dans une bande comprise entre les rayons $r$ et $r + dr$ quel que soit r, mais que la surface de cette bande croît avec r, ainsi, la densité est plus élevée en moyenne quand $r$ tend vers $0$ les deux méthodes suivantes semblent donner des points répartis uniformément sur le disque.
		Afin de séparer ces deux dernières méthodes, on prendra donc la plus efficace. Une mesure du temps de calcul de ces trois méthodes sur une génération de 600 points avec 600 répétitions donne le résultat suivant (fichier gen.py) : 
		\begin{verbatim}T1 = 0.24763862291971842 ms (std.dev. 0.0098647314503971)
T2 = 0.8399013678232828 ms (std.dev. 0.07227058109796747)
T3 = 0.4730105400085449 ms (std.dev. 0.023126789901638765)
\end{verbatim}
		On remarque que la troisième méthode est significativement plus rapide que la seconde, et constitue ainsi le candidat qui réalise à la fois les contraintes indispensables (répartition uniforme des points sur le cercle) et qui minimise le temps de calcul. Nous retenons donc cette méthode de tirage afin de réaliser les essais ultérieurs.

		De plus, la première méthode est plus rapide que la troisième seulement car elle n'effectue pas la projection des coordonnées polaires générées en coordonnées cartésiennes, une fois cette conversion effectuée, les méthodes un et trois retrouvent un temps de calcul comparable, ce qui met encore plus en valeur le choix précédement exprimé.

	\subsection{Trouver l'enveloppe convexe de $n$ points}

		L'idée de cet algorithme est de partir d'un point extremal, qu'on sait appartenir à l'enveloppe convexe (question \ref{extr}), puis de tourner autour de la figure, le fait de tourner dans le même sens garantira la convexité de l'enveloppe. on parcourera les points en partant du point le plus a gauche et en parcourant les points dans le sens trigonométrique par rapport au point d'origine lors de ce parcours, on ajoutera les points au fur et à mesure dans la pile qui définira l'enveloppe convexe, si un point effectue un virage à droite, on éliminera un a un les points de la pile jusqu'à ce que le virage entre le nouveau point et le haut de la pile se fasse à gauche. les points éliminés entre le haut de la pile et le point en cours de considération étant contenus dans l'enveloppe engendrée par les points contenus dans la pile et les points encore non considérés.  

		On a donc un invariant de boucle qui est : "Tous les points sont contenus dans l'enveloppe engendrée par l'union des points présents dans la pile avec les points encore non considérés, et l'enveloppe engendrée par les points présents sur la pile est convexe" 

		Ainsi, après avoir parcouru tous les points, il ne reste plus aucun point encore non considéré, l'invariant de boucle devient donc à la fin du parcours, "Tous les points sont contenus dans l'enveloppe engendrée par les points sur la pile, et cette enveloppe est convexe" ce qui correspond bien au résultat attendu

		Nous venons de décrire de façon intuitive l'algoritheme qui permettra de résoudre notre problème, mais il reste encore à définir plus rigoureusement ce que signifie "Tourner a gauche/droite" et "Effectuer un parcours dans le sens trigonométrique"

		Commençons par définir "Effectuer un parcours dans le sens trigonométrique". Pour cela, il nous faut trier les points selon un critère précis. D'un point de vue algorithmique, cette opération prendra $O(n\ln n)$ opération, en supposant que l'opération de comparaison prend un temps constant, ce que nous vérifierons par la suite. Le choix des mots suggère ici de calculer un angle pour chaque point, et de comparer les angles de chaque point. Une première idée est d'associer a chaque point des coordonnées $(x, y)$ le complexe $z = x + iy$ et de calculer pour chacun de ces points la différence entre l'argument du complexe associé à ce point et l'argument du complexe associé au point extrémal choisi au début. Cette valeur marche dans la plupart des cas, mais ne fonctionne pas dans le cas ou l'origine ne se trouve pas à l'intérieur de l'ensemble des points tirés (Ce qui est un évènement de probabilité exponentiellement décroissante). On pourrait alors considérer de calculer en premier lieu le barycentre de l'ensemble de ces points et de prendre ce barycentre comme origine.

		Il existe néanmoins une méthode plus élégante qui ne nécessite pas de calculer un tel point (Calcul de complexité $O(n)$ qui ne rallonge pas asymptotiquement le temps d'execution de l'algorithme mais qui constitue un cout évitable). On va trier les points selon la pente que forme la droite passant par le point extremal et ce point. Afin d'éviter d'avoir des valeurs infinies, on utilisera la fonction atan2(x,y) qui calcule l'arctangente de $\frac{x}{y}$ et renvoie $\pm \frac{\pi}{2}$ si $y = 0$, selon le signe de $x$, et nous permet donc de classer même les points à la verticale du point extremal

		Afin de définir "Tourner à droite" et "Tourner à gauche", on peut reformuler le problèmes en termes marins, on veut savoir si le point C est à babord ou a tribord du bateau modélisé par le vecteur AB, pour cela, on utilise le déterminan de ces trois vecteurs, dont le signe discrimine ces deux états.

		Il est alors pertinent de se demander si un tel algorithme peut être généralisé en dimension plus grande, quand les points tirés ne sont pas dans le plan, mais dans l'espace, l'hyperespace, ou dans un espace de dimension n.



	\subsection{Le vif du sujet}
		Afin d'estimer $\varepsilon_n$, on effectue pour chaque $n$ plusieurs tirages (On à choisi ici 100 qui constitue un bon compromis entre qualité des résultats et vitesse de calcul), on calcule l'enveloppe convexe de ce tirage et on retient la moyenne du nombre de sommet de l'enveloppe convexe comme valeur estimée de $\varepsilon_n$

		L'implémentation d'un tel estimateur (fichier main.py) permet de tracer le graphe Figure~\ref{fig:exp_cercle}, ainsi que de calculer la courbe de la forme $kn^{1/3}$ qui approche au mieux les données expérimentales.

		\begin{figure}[htpb]
			\centering
			%% Creator: Matplotlib, PGF backend
%%
%% To include the figure in your LaTeX document, write
%%   \input{<filename>.pgf}
%%
%% Make sure the required packages are loaded in your preamble
%%   \usepackage{pgf}
%%
%% Figures using additional raster images can only be included by \input if
%% they are in the same directory as the main LaTeX file. For loading figures
%% from other directories you can use the `import` package
%%   \usepackage{import}
%% and then include the figures with
%%   \import{<path to file>}{<filename>.pgf}
%%
%% Matplotlib used the following preamble
%%   \usepackage{fontspec}
%%   \setmainfont{DejaVu Serif}
%%   \setsansfont{DejaVu Sans}
%%   \setmonofont{DejaVu Sans Mono}
%%
\begingroup%
\makeatletter%
\begin{pgfpicture}%
\pgfpathrectangle{\pgfpointorigin}{\pgfqpoint{6.400000in}{4.800000in}}%
\pgfusepath{use as bounding box, clip}%
\begin{pgfscope}%
\pgfsetbuttcap%
\pgfsetmiterjoin%
\definecolor{currentfill}{rgb}{1.000000,1.000000,1.000000}%
\pgfsetfillcolor{currentfill}%
\pgfsetlinewidth{0.000000pt}%
\definecolor{currentstroke}{rgb}{1.000000,1.000000,1.000000}%
\pgfsetstrokecolor{currentstroke}%
\pgfsetdash{}{0pt}%
\pgfpathmoveto{\pgfqpoint{0.000000in}{0.000000in}}%
\pgfpathlineto{\pgfqpoint{6.400000in}{0.000000in}}%
\pgfpathlineto{\pgfqpoint{6.400000in}{4.800000in}}%
\pgfpathlineto{\pgfqpoint{0.000000in}{4.800000in}}%
\pgfpathclose%
\pgfusepath{fill}%
\end{pgfscope}%
\begin{pgfscope}%
\pgfsetbuttcap%
\pgfsetmiterjoin%
\definecolor{currentfill}{rgb}{1.000000,1.000000,1.000000}%
\pgfsetfillcolor{currentfill}%
\pgfsetlinewidth{0.000000pt}%
\definecolor{currentstroke}{rgb}{0.000000,0.000000,0.000000}%
\pgfsetstrokecolor{currentstroke}%
\pgfsetstrokeopacity{0.000000}%
\pgfsetdash{}{0pt}%
\pgfpathmoveto{\pgfqpoint{0.800000in}{0.528000in}}%
\pgfpathlineto{\pgfqpoint{5.760000in}{0.528000in}}%
\pgfpathlineto{\pgfqpoint{5.760000in}{4.224000in}}%
\pgfpathlineto{\pgfqpoint{0.800000in}{4.224000in}}%
\pgfpathclose%
\pgfusepath{fill}%
\end{pgfscope}%
\begin{pgfscope}%
\pgfsetbuttcap%
\pgfsetroundjoin%
\definecolor{currentfill}{rgb}{0.000000,0.000000,0.000000}%
\pgfsetfillcolor{currentfill}%
\pgfsetlinewidth{0.803000pt}%
\definecolor{currentstroke}{rgb}{0.000000,0.000000,0.000000}%
\pgfsetstrokecolor{currentstroke}%
\pgfsetdash{}{0pt}%
\pgfsys@defobject{currentmarker}{\pgfqpoint{0.000000in}{-0.048611in}}{\pgfqpoint{0.000000in}{0.000000in}}{%
\pgfpathmoveto{\pgfqpoint{0.000000in}{0.000000in}}%
\pgfpathlineto{\pgfqpoint{0.000000in}{-0.048611in}}%
\pgfusepath{stroke,fill}%
}%
\begin{pgfscope}%
\pgfsys@transformshift{0.979862in}{0.528000in}%
\pgfsys@useobject{currentmarker}{}%
\end{pgfscope}%
\end{pgfscope}%
\begin{pgfscope}%
\pgftext[x=0.979862in,y=0.430778in,,top]{\sffamily\fontsize{10.000000}{12.000000}\selectfont 0}%
\end{pgfscope}%
\begin{pgfscope}%
\pgfsetbuttcap%
\pgfsetroundjoin%
\definecolor{currentfill}{rgb}{0.000000,0.000000,0.000000}%
\pgfsetfillcolor{currentfill}%
\pgfsetlinewidth{0.803000pt}%
\definecolor{currentstroke}{rgb}{0.000000,0.000000,0.000000}%
\pgfsetstrokecolor{currentstroke}%
\pgfsetdash{}{0pt}%
\pgfsys@defobject{currentmarker}{\pgfqpoint{0.000000in}{-0.048611in}}{\pgfqpoint{0.000000in}{0.000000in}}{%
\pgfpathmoveto{\pgfqpoint{0.000000in}{0.000000in}}%
\pgfpathlineto{\pgfqpoint{0.000000in}{-0.048611in}}%
\pgfusepath{stroke,fill}%
}%
\begin{pgfscope}%
\pgfsys@transformshift{1.891711in}{0.528000in}%
\pgfsys@useobject{currentmarker}{}%
\end{pgfscope}%
\end{pgfscope}%
\begin{pgfscope}%
\pgftext[x=1.891711in,y=0.430778in,,top]{\sffamily\fontsize{10.000000}{12.000000}\selectfont 200}%
\end{pgfscope}%
\begin{pgfscope}%
\pgfsetbuttcap%
\pgfsetroundjoin%
\definecolor{currentfill}{rgb}{0.000000,0.000000,0.000000}%
\pgfsetfillcolor{currentfill}%
\pgfsetlinewidth{0.803000pt}%
\definecolor{currentstroke}{rgb}{0.000000,0.000000,0.000000}%
\pgfsetstrokecolor{currentstroke}%
\pgfsetdash{}{0pt}%
\pgfsys@defobject{currentmarker}{\pgfqpoint{0.000000in}{-0.048611in}}{\pgfqpoint{0.000000in}{0.000000in}}{%
\pgfpathmoveto{\pgfqpoint{0.000000in}{0.000000in}}%
\pgfpathlineto{\pgfqpoint{0.000000in}{-0.048611in}}%
\pgfusepath{stroke,fill}%
}%
\begin{pgfscope}%
\pgfsys@transformshift{2.803559in}{0.528000in}%
\pgfsys@useobject{currentmarker}{}%
\end{pgfscope}%
\end{pgfscope}%
\begin{pgfscope}%
\pgftext[x=2.803559in,y=0.430778in,,top]{\sffamily\fontsize{10.000000}{12.000000}\selectfont 400}%
\end{pgfscope}%
\begin{pgfscope}%
\pgfsetbuttcap%
\pgfsetroundjoin%
\definecolor{currentfill}{rgb}{0.000000,0.000000,0.000000}%
\pgfsetfillcolor{currentfill}%
\pgfsetlinewidth{0.803000pt}%
\definecolor{currentstroke}{rgb}{0.000000,0.000000,0.000000}%
\pgfsetstrokecolor{currentstroke}%
\pgfsetdash{}{0pt}%
\pgfsys@defobject{currentmarker}{\pgfqpoint{0.000000in}{-0.048611in}}{\pgfqpoint{0.000000in}{0.000000in}}{%
\pgfpathmoveto{\pgfqpoint{0.000000in}{0.000000in}}%
\pgfpathlineto{\pgfqpoint{0.000000in}{-0.048611in}}%
\pgfusepath{stroke,fill}%
}%
\begin{pgfscope}%
\pgfsys@transformshift{3.715408in}{0.528000in}%
\pgfsys@useobject{currentmarker}{}%
\end{pgfscope}%
\end{pgfscope}%
\begin{pgfscope}%
\pgftext[x=3.715408in,y=0.430778in,,top]{\sffamily\fontsize{10.000000}{12.000000}\selectfont 600}%
\end{pgfscope}%
\begin{pgfscope}%
\pgfsetbuttcap%
\pgfsetroundjoin%
\definecolor{currentfill}{rgb}{0.000000,0.000000,0.000000}%
\pgfsetfillcolor{currentfill}%
\pgfsetlinewidth{0.803000pt}%
\definecolor{currentstroke}{rgb}{0.000000,0.000000,0.000000}%
\pgfsetstrokecolor{currentstroke}%
\pgfsetdash{}{0pt}%
\pgfsys@defobject{currentmarker}{\pgfqpoint{0.000000in}{-0.048611in}}{\pgfqpoint{0.000000in}{0.000000in}}{%
\pgfpathmoveto{\pgfqpoint{0.000000in}{0.000000in}}%
\pgfpathlineto{\pgfqpoint{0.000000in}{-0.048611in}}%
\pgfusepath{stroke,fill}%
}%
\begin{pgfscope}%
\pgfsys@transformshift{4.627256in}{0.528000in}%
\pgfsys@useobject{currentmarker}{}%
\end{pgfscope}%
\end{pgfscope}%
\begin{pgfscope}%
\pgftext[x=4.627256in,y=0.430778in,,top]{\sffamily\fontsize{10.000000}{12.000000}\selectfont 800}%
\end{pgfscope}%
\begin{pgfscope}%
\pgfsetbuttcap%
\pgfsetroundjoin%
\definecolor{currentfill}{rgb}{0.000000,0.000000,0.000000}%
\pgfsetfillcolor{currentfill}%
\pgfsetlinewidth{0.803000pt}%
\definecolor{currentstroke}{rgb}{0.000000,0.000000,0.000000}%
\pgfsetstrokecolor{currentstroke}%
\pgfsetdash{}{0pt}%
\pgfsys@defobject{currentmarker}{\pgfqpoint{0.000000in}{-0.048611in}}{\pgfqpoint{0.000000in}{0.000000in}}{%
\pgfpathmoveto{\pgfqpoint{0.000000in}{0.000000in}}%
\pgfpathlineto{\pgfqpoint{0.000000in}{-0.048611in}}%
\pgfusepath{stroke,fill}%
}%
\begin{pgfscope}%
\pgfsys@transformshift{5.539105in}{0.528000in}%
\pgfsys@useobject{currentmarker}{}%
\end{pgfscope}%
\end{pgfscope}%
\begin{pgfscope}%
\pgftext[x=5.539105in,y=0.430778in,,top]{\sffamily\fontsize{10.000000}{12.000000}\selectfont 1000}%
\end{pgfscope}%
\begin{pgfscope}%
\pgftext[x=3.280000in,y=0.240809in,,top]{\sffamily\fontsize{10.000000}{12.000000}\selectfont \(\displaystyle n\)}%
\end{pgfscope}%
\begin{pgfscope}%
\pgfsetbuttcap%
\pgfsetroundjoin%
\definecolor{currentfill}{rgb}{0.000000,0.000000,0.000000}%
\pgfsetfillcolor{currentfill}%
\pgfsetlinewidth{0.803000pt}%
\definecolor{currentstroke}{rgb}{0.000000,0.000000,0.000000}%
\pgfsetstrokecolor{currentstroke}%
\pgfsetdash{}{0pt}%
\pgfsys@defobject{currentmarker}{\pgfqpoint{-0.048611in}{0.000000in}}{\pgfqpoint{0.000000in}{0.000000in}}{%
\pgfpathmoveto{\pgfqpoint{0.000000in}{0.000000in}}%
\pgfpathlineto{\pgfqpoint{-0.048611in}{0.000000in}}%
\pgfusepath{stroke,fill}%
}%
\begin{pgfscope}%
\pgfsys@transformshift{0.800000in}{0.576742in}%
\pgfsys@useobject{currentmarker}{}%
\end{pgfscope}%
\end{pgfscope}%
\begin{pgfscope}%
\pgftext[x=0.614413in,y=0.523980in,left,base]{\sffamily\fontsize{10.000000}{12.000000}\selectfont 5}%
\end{pgfscope}%
\begin{pgfscope}%
\pgfsetbuttcap%
\pgfsetroundjoin%
\definecolor{currentfill}{rgb}{0.000000,0.000000,0.000000}%
\pgfsetfillcolor{currentfill}%
\pgfsetlinewidth{0.803000pt}%
\definecolor{currentstroke}{rgb}{0.000000,0.000000,0.000000}%
\pgfsetstrokecolor{currentstroke}%
\pgfsetdash{}{0pt}%
\pgfsys@defobject{currentmarker}{\pgfqpoint{-0.048611in}{0.000000in}}{\pgfqpoint{0.000000in}{0.000000in}}{%
\pgfpathmoveto{\pgfqpoint{0.000000in}{0.000000in}}%
\pgfpathlineto{\pgfqpoint{-0.048611in}{0.000000in}}%
\pgfusepath{stroke,fill}%
}%
\begin{pgfscope}%
\pgfsys@transformshift{0.800000in}{1.118825in}%
\pgfsys@useobject{currentmarker}{}%
\end{pgfscope}%
\end{pgfscope}%
\begin{pgfscope}%
\pgftext[x=0.526047in,y=1.066064in,left,base]{\sffamily\fontsize{10.000000}{12.000000}\selectfont 10}%
\end{pgfscope}%
\begin{pgfscope}%
\pgfsetbuttcap%
\pgfsetroundjoin%
\definecolor{currentfill}{rgb}{0.000000,0.000000,0.000000}%
\pgfsetfillcolor{currentfill}%
\pgfsetlinewidth{0.803000pt}%
\definecolor{currentstroke}{rgb}{0.000000,0.000000,0.000000}%
\pgfsetstrokecolor{currentstroke}%
\pgfsetdash{}{0pt}%
\pgfsys@defobject{currentmarker}{\pgfqpoint{-0.048611in}{0.000000in}}{\pgfqpoint{0.000000in}{0.000000in}}{%
\pgfpathmoveto{\pgfqpoint{0.000000in}{0.000000in}}%
\pgfpathlineto{\pgfqpoint{-0.048611in}{0.000000in}}%
\pgfusepath{stroke,fill}%
}%
\begin{pgfscope}%
\pgfsys@transformshift{0.800000in}{1.660909in}%
\pgfsys@useobject{currentmarker}{}%
\end{pgfscope}%
\end{pgfscope}%
\begin{pgfscope}%
\pgftext[x=0.526047in,y=1.608148in,left,base]{\sffamily\fontsize{10.000000}{12.000000}\selectfont 15}%
\end{pgfscope}%
\begin{pgfscope}%
\pgfsetbuttcap%
\pgfsetroundjoin%
\definecolor{currentfill}{rgb}{0.000000,0.000000,0.000000}%
\pgfsetfillcolor{currentfill}%
\pgfsetlinewidth{0.803000pt}%
\definecolor{currentstroke}{rgb}{0.000000,0.000000,0.000000}%
\pgfsetstrokecolor{currentstroke}%
\pgfsetdash{}{0pt}%
\pgfsys@defobject{currentmarker}{\pgfqpoint{-0.048611in}{0.000000in}}{\pgfqpoint{0.000000in}{0.000000in}}{%
\pgfpathmoveto{\pgfqpoint{0.000000in}{0.000000in}}%
\pgfpathlineto{\pgfqpoint{-0.048611in}{0.000000in}}%
\pgfusepath{stroke,fill}%
}%
\begin{pgfscope}%
\pgfsys@transformshift{0.800000in}{2.202993in}%
\pgfsys@useobject{currentmarker}{}%
\end{pgfscope}%
\end{pgfscope}%
\begin{pgfscope}%
\pgftext[x=0.526047in,y=2.150232in,left,base]{\sffamily\fontsize{10.000000}{12.000000}\selectfont 20}%
\end{pgfscope}%
\begin{pgfscope}%
\pgfsetbuttcap%
\pgfsetroundjoin%
\definecolor{currentfill}{rgb}{0.000000,0.000000,0.000000}%
\pgfsetfillcolor{currentfill}%
\pgfsetlinewidth{0.803000pt}%
\definecolor{currentstroke}{rgb}{0.000000,0.000000,0.000000}%
\pgfsetstrokecolor{currentstroke}%
\pgfsetdash{}{0pt}%
\pgfsys@defobject{currentmarker}{\pgfqpoint{-0.048611in}{0.000000in}}{\pgfqpoint{0.000000in}{0.000000in}}{%
\pgfpathmoveto{\pgfqpoint{0.000000in}{0.000000in}}%
\pgfpathlineto{\pgfqpoint{-0.048611in}{0.000000in}}%
\pgfusepath{stroke,fill}%
}%
\begin{pgfscope}%
\pgfsys@transformshift{0.800000in}{2.745077in}%
\pgfsys@useobject{currentmarker}{}%
\end{pgfscope}%
\end{pgfscope}%
\begin{pgfscope}%
\pgftext[x=0.526047in,y=2.692316in,left,base]{\sffamily\fontsize{10.000000}{12.000000}\selectfont 25}%
\end{pgfscope}%
\begin{pgfscope}%
\pgfsetbuttcap%
\pgfsetroundjoin%
\definecolor{currentfill}{rgb}{0.000000,0.000000,0.000000}%
\pgfsetfillcolor{currentfill}%
\pgfsetlinewidth{0.803000pt}%
\definecolor{currentstroke}{rgb}{0.000000,0.000000,0.000000}%
\pgfsetstrokecolor{currentstroke}%
\pgfsetdash{}{0pt}%
\pgfsys@defobject{currentmarker}{\pgfqpoint{-0.048611in}{0.000000in}}{\pgfqpoint{0.000000in}{0.000000in}}{%
\pgfpathmoveto{\pgfqpoint{0.000000in}{0.000000in}}%
\pgfpathlineto{\pgfqpoint{-0.048611in}{0.000000in}}%
\pgfusepath{stroke,fill}%
}%
\begin{pgfscope}%
\pgfsys@transformshift{0.800000in}{3.287161in}%
\pgfsys@useobject{currentmarker}{}%
\end{pgfscope}%
\end{pgfscope}%
\begin{pgfscope}%
\pgftext[x=0.526047in,y=3.234400in,left,base]{\sffamily\fontsize{10.000000}{12.000000}\selectfont 30}%
\end{pgfscope}%
\begin{pgfscope}%
\pgfsetbuttcap%
\pgfsetroundjoin%
\definecolor{currentfill}{rgb}{0.000000,0.000000,0.000000}%
\pgfsetfillcolor{currentfill}%
\pgfsetlinewidth{0.803000pt}%
\definecolor{currentstroke}{rgb}{0.000000,0.000000,0.000000}%
\pgfsetstrokecolor{currentstroke}%
\pgfsetdash{}{0pt}%
\pgfsys@defobject{currentmarker}{\pgfqpoint{-0.048611in}{0.000000in}}{\pgfqpoint{0.000000in}{0.000000in}}{%
\pgfpathmoveto{\pgfqpoint{0.000000in}{0.000000in}}%
\pgfpathlineto{\pgfqpoint{-0.048611in}{0.000000in}}%
\pgfusepath{stroke,fill}%
}%
\begin{pgfscope}%
\pgfsys@transformshift{0.800000in}{3.829245in}%
\pgfsys@useobject{currentmarker}{}%
\end{pgfscope}%
\end{pgfscope}%
\begin{pgfscope}%
\pgftext[x=0.526047in,y=3.776484in,left,base]{\sffamily\fontsize{10.000000}{12.000000}\selectfont 35}%
\end{pgfscope}%
\begin{pgfscope}%
\pgftext[x=0.470492in,y=2.376000in,,bottom,rotate=90.000000]{\sffamily\fontsize{10.000000}{12.000000}\selectfont \(\displaystyle \epsilon_n\)}%
\end{pgfscope}%
\begin{pgfscope}%
\pgfpathrectangle{\pgfqpoint{0.800000in}{0.528000in}}{\pgfqpoint{4.960000in}{3.696000in}} %
\pgfusepath{clip}%
\pgfsetrectcap%
\pgfsetroundjoin%
\pgfsetlinewidth{1.505625pt}%
\definecolor{currentstroke}{rgb}{0.121569,0.466667,0.705882}%
\pgfsetstrokecolor{currentstroke}%
\pgfsetdash{}{0pt}%
\pgfpathmoveto{\pgfqpoint{1.025455in}{0.696000in}}%
\pgfpathlineto{\pgfqpoint{1.030014in}{0.750208in}}%
\pgfpathlineto{\pgfqpoint{1.034573in}{0.750208in}}%
\pgfpathlineto{\pgfqpoint{1.039132in}{0.793575in}}%
\pgfpathlineto{\pgfqpoint{1.043692in}{0.815258in}}%
\pgfpathlineto{\pgfqpoint{1.052810in}{0.934517in}}%
\pgfpathlineto{\pgfqpoint{1.057369in}{0.826100in}}%
\pgfpathlineto{\pgfqpoint{1.061928in}{0.869467in}}%
\pgfpathlineto{\pgfqpoint{1.066488in}{0.880309in}}%
\pgfpathlineto{\pgfqpoint{1.071047in}{0.956200in}}%
\pgfpathlineto{\pgfqpoint{1.075606in}{1.053775in}}%
\pgfpathlineto{\pgfqpoint{1.080165in}{1.032092in}}%
\pgfpathlineto{\pgfqpoint{1.084725in}{1.021250in}}%
\pgfpathlineto{\pgfqpoint{1.089284in}{0.988725in}}%
\pgfpathlineto{\pgfqpoint{1.093843in}{1.042934in}}%
\pgfpathlineto{\pgfqpoint{1.098402in}{0.977884in}}%
\pgfpathlineto{\pgfqpoint{1.107521in}{1.064617in}}%
\pgfpathlineto{\pgfqpoint{1.112080in}{1.075459in}}%
\pgfpathlineto{\pgfqpoint{1.116639in}{1.151351in}}%
\pgfpathlineto{\pgfqpoint{1.121199in}{0.999567in}}%
\pgfpathlineto{\pgfqpoint{1.125758in}{1.107984in}}%
\pgfpathlineto{\pgfqpoint{1.130317in}{1.183876in}}%
\pgfpathlineto{\pgfqpoint{1.134876in}{1.053775in}}%
\pgfpathlineto{\pgfqpoint{1.139436in}{1.129667in}}%
\pgfpathlineto{\pgfqpoint{1.143995in}{1.173034in}}%
\pgfpathlineto{\pgfqpoint{1.148554in}{1.205559in}}%
\pgfpathlineto{\pgfqpoint{1.153113in}{1.281451in}}%
\pgfpathlineto{\pgfqpoint{1.157673in}{1.238084in}}%
\pgfpathlineto{\pgfqpoint{1.162232in}{1.227242in}}%
\pgfpathlineto{\pgfqpoint{1.166791in}{1.248926in}}%
\pgfpathlineto{\pgfqpoint{1.171350in}{1.248926in}}%
\pgfpathlineto{\pgfqpoint{1.175910in}{1.183876in}}%
\pgfpathlineto{\pgfqpoint{1.180469in}{1.227242in}}%
\pgfpathlineto{\pgfqpoint{1.185028in}{1.162192in}}%
\pgfpathlineto{\pgfqpoint{1.189587in}{1.259767in}}%
\pgfpathlineto{\pgfqpoint{1.194147in}{1.281451in}}%
\pgfpathlineto{\pgfqpoint{1.198706in}{1.292292in}}%
\pgfpathlineto{\pgfqpoint{1.203265in}{1.216401in}}%
\pgfpathlineto{\pgfqpoint{1.207824in}{1.313976in}}%
\pgfpathlineto{\pgfqpoint{1.212383in}{1.303134in}}%
\pgfpathlineto{\pgfqpoint{1.216943in}{1.324817in}}%
\pgfpathlineto{\pgfqpoint{1.221502in}{1.335659in}}%
\pgfpathlineto{\pgfqpoint{1.226061in}{1.411551in}}%
\pgfpathlineto{\pgfqpoint{1.230620in}{1.281451in}}%
\pgfpathlineto{\pgfqpoint{1.235180in}{1.313976in}}%
\pgfpathlineto{\pgfqpoint{1.239739in}{1.422392in}}%
\pgfpathlineto{\pgfqpoint{1.244298in}{1.379026in}}%
\pgfpathlineto{\pgfqpoint{1.253417in}{1.476601in}}%
\pgfpathlineto{\pgfqpoint{1.262535in}{1.454918in}}%
\pgfpathlineto{\pgfqpoint{1.267094in}{1.411551in}}%
\pgfpathlineto{\pgfqpoint{1.271654in}{1.379026in}}%
\pgfpathlineto{\pgfqpoint{1.276213in}{1.368184in}}%
\pgfpathlineto{\pgfqpoint{1.280772in}{1.379026in}}%
\pgfpathlineto{\pgfqpoint{1.285331in}{1.411551in}}%
\pgfpathlineto{\pgfqpoint{1.289891in}{1.574176in}}%
\pgfpathlineto{\pgfqpoint{1.294450in}{1.465759in}}%
\pgfpathlineto{\pgfqpoint{1.299009in}{1.444076in}}%
\pgfpathlineto{\pgfqpoint{1.303568in}{1.519968in}}%
\pgfpathlineto{\pgfqpoint{1.312687in}{1.433234in}}%
\pgfpathlineto{\pgfqpoint{1.317246in}{1.476601in}}%
\pgfpathlineto{\pgfqpoint{1.321805in}{1.606701in}}%
\pgfpathlineto{\pgfqpoint{1.326365in}{1.498284in}}%
\pgfpathlineto{\pgfqpoint{1.330924in}{1.595859in}}%
\pgfpathlineto{\pgfqpoint{1.335483in}{1.530809in}}%
\pgfpathlineto{\pgfqpoint{1.340042in}{1.541651in}}%
\pgfpathlineto{\pgfqpoint{1.344602in}{1.454918in}}%
\pgfpathlineto{\pgfqpoint{1.353720in}{1.574176in}}%
\pgfpathlineto{\pgfqpoint{1.358279in}{1.704276in}}%
\pgfpathlineto{\pgfqpoint{1.362838in}{1.519968in}}%
\pgfpathlineto{\pgfqpoint{1.367398in}{1.639226in}}%
\pgfpathlineto{\pgfqpoint{1.371957in}{1.585018in}}%
\pgfpathlineto{\pgfqpoint{1.376516in}{1.628384in}}%
\pgfpathlineto{\pgfqpoint{1.381075in}{1.639226in}}%
\pgfpathlineto{\pgfqpoint{1.385635in}{1.693434in}}%
\pgfpathlineto{\pgfqpoint{1.390194in}{1.606701in}}%
\pgfpathlineto{\pgfqpoint{1.394753in}{1.639226in}}%
\pgfpathlineto{\pgfqpoint{1.399312in}{1.595859in}}%
\pgfpathlineto{\pgfqpoint{1.403872in}{1.595859in}}%
\pgfpathlineto{\pgfqpoint{1.408431in}{1.715118in}}%
\pgfpathlineto{\pgfqpoint{1.412990in}{1.704276in}}%
\pgfpathlineto{\pgfqpoint{1.417549in}{1.780168in}}%
\pgfpathlineto{\pgfqpoint{1.422109in}{1.671751in}}%
\pgfpathlineto{\pgfqpoint{1.426668in}{1.660909in}}%
\pgfpathlineto{\pgfqpoint{1.431227in}{1.704276in}}%
\pgfpathlineto{\pgfqpoint{1.435786in}{1.574176in}}%
\pgfpathlineto{\pgfqpoint{1.440346in}{1.671751in}}%
\pgfpathlineto{\pgfqpoint{1.444905in}{1.606701in}}%
\pgfpathlineto{\pgfqpoint{1.449464in}{1.682593in}}%
\pgfpathlineto{\pgfqpoint{1.454023in}{1.650068in}}%
\pgfpathlineto{\pgfqpoint{1.458583in}{1.758485in}}%
\pgfpathlineto{\pgfqpoint{1.463142in}{1.660909in}}%
\pgfpathlineto{\pgfqpoint{1.467701in}{1.682593in}}%
\pgfpathlineto{\pgfqpoint{1.472260in}{1.747643in}}%
\pgfpathlineto{\pgfqpoint{1.476820in}{1.606701in}}%
\pgfpathlineto{\pgfqpoint{1.481379in}{1.845218in}}%
\pgfpathlineto{\pgfqpoint{1.485938in}{1.812693in}}%
\pgfpathlineto{\pgfqpoint{1.490497in}{1.704276in}}%
\pgfpathlineto{\pgfqpoint{1.495057in}{1.693434in}}%
\pgfpathlineto{\pgfqpoint{1.499616in}{1.736801in}}%
\pgfpathlineto{\pgfqpoint{1.504175in}{1.769326in}}%
\pgfpathlineto{\pgfqpoint{1.513294in}{1.791010in}}%
\pgfpathlineto{\pgfqpoint{1.517853in}{1.856060in}}%
\pgfpathlineto{\pgfqpoint{1.522412in}{1.856060in}}%
\pgfpathlineto{\pgfqpoint{1.526971in}{1.758485in}}%
\pgfpathlineto{\pgfqpoint{1.531530in}{1.725959in}}%
\pgfpathlineto{\pgfqpoint{1.536090in}{1.910268in}}%
\pgfpathlineto{\pgfqpoint{1.540649in}{1.834376in}}%
\pgfpathlineto{\pgfqpoint{1.545208in}{1.888585in}}%
\pgfpathlineto{\pgfqpoint{1.549767in}{1.801851in}}%
\pgfpathlineto{\pgfqpoint{1.554327in}{1.791010in}}%
\pgfpathlineto{\pgfqpoint{1.558886in}{1.812693in}}%
\pgfpathlineto{\pgfqpoint{1.563445in}{1.899426in}}%
\pgfpathlineto{\pgfqpoint{1.568004in}{1.845218in}}%
\pgfpathlineto{\pgfqpoint{1.572564in}{1.693434in}}%
\pgfpathlineto{\pgfqpoint{1.577123in}{1.964476in}}%
\pgfpathlineto{\pgfqpoint{1.581682in}{1.942793in}}%
\pgfpathlineto{\pgfqpoint{1.586241in}{1.823535in}}%
\pgfpathlineto{\pgfqpoint{1.590801in}{1.834376in}}%
\pgfpathlineto{\pgfqpoint{1.595360in}{1.834376in}}%
\pgfpathlineto{\pgfqpoint{1.599919in}{1.812693in}}%
\pgfpathlineto{\pgfqpoint{1.604478in}{1.997001in}}%
\pgfpathlineto{\pgfqpoint{1.609038in}{1.942793in}}%
\pgfpathlineto{\pgfqpoint{1.613597in}{1.747643in}}%
\pgfpathlineto{\pgfqpoint{1.618156in}{1.921110in}}%
\pgfpathlineto{\pgfqpoint{1.622715in}{1.845218in}}%
\pgfpathlineto{\pgfqpoint{1.627275in}{1.888585in}}%
\pgfpathlineto{\pgfqpoint{1.631834in}{1.888585in}}%
\pgfpathlineto{\pgfqpoint{1.636393in}{1.747643in}}%
\pgfpathlineto{\pgfqpoint{1.640952in}{1.910268in}}%
\pgfpathlineto{\pgfqpoint{1.645512in}{1.866901in}}%
\pgfpathlineto{\pgfqpoint{1.650071in}{1.910268in}}%
\pgfpathlineto{\pgfqpoint{1.654630in}{1.899426in}}%
\pgfpathlineto{\pgfqpoint{1.659189in}{1.834376in}}%
\pgfpathlineto{\pgfqpoint{1.663749in}{1.964476in}}%
\pgfpathlineto{\pgfqpoint{1.668308in}{1.899426in}}%
\pgfpathlineto{\pgfqpoint{1.672867in}{2.127102in}}%
\pgfpathlineto{\pgfqpoint{1.677426in}{1.812693in}}%
\pgfpathlineto{\pgfqpoint{1.681985in}{1.921110in}}%
\pgfpathlineto{\pgfqpoint{1.686545in}{1.877743in}}%
\pgfpathlineto{\pgfqpoint{1.691104in}{1.942793in}}%
\pgfpathlineto{\pgfqpoint{1.695663in}{1.931951in}}%
\pgfpathlineto{\pgfqpoint{1.700222in}{2.051210in}}%
\pgfpathlineto{\pgfqpoint{1.704782in}{1.931951in}}%
\pgfpathlineto{\pgfqpoint{1.709341in}{1.856060in}}%
\pgfpathlineto{\pgfqpoint{1.713900in}{1.975318in}}%
\pgfpathlineto{\pgfqpoint{1.718459in}{2.127102in}}%
\pgfpathlineto{\pgfqpoint{1.723019in}{2.029526in}}%
\pgfpathlineto{\pgfqpoint{1.727578in}{1.823535in}}%
\pgfpathlineto{\pgfqpoint{1.732137in}{1.856060in}}%
\pgfpathlineto{\pgfqpoint{1.736696in}{2.083735in}}%
\pgfpathlineto{\pgfqpoint{1.741256in}{1.866901in}}%
\pgfpathlineto{\pgfqpoint{1.745815in}{2.029526in}}%
\pgfpathlineto{\pgfqpoint{1.750374in}{1.986160in}}%
\pgfpathlineto{\pgfqpoint{1.754933in}{2.083735in}}%
\pgfpathlineto{\pgfqpoint{1.759493in}{1.964476in}}%
\pgfpathlineto{\pgfqpoint{1.764052in}{2.094577in}}%
\pgfpathlineto{\pgfqpoint{1.768611in}{2.040368in}}%
\pgfpathlineto{\pgfqpoint{1.773170in}{1.953635in}}%
\pgfpathlineto{\pgfqpoint{1.777730in}{2.083735in}}%
\pgfpathlineto{\pgfqpoint{1.786848in}{1.964476in}}%
\pgfpathlineto{\pgfqpoint{1.791407in}{2.192152in}}%
\pgfpathlineto{\pgfqpoint{1.795967in}{2.137943in}}%
\pgfpathlineto{\pgfqpoint{1.805085in}{2.072893in}}%
\pgfpathlineto{\pgfqpoint{1.809644in}{2.083735in}}%
\pgfpathlineto{\pgfqpoint{1.814204in}{1.986160in}}%
\pgfpathlineto{\pgfqpoint{1.818763in}{2.127102in}}%
\pgfpathlineto{\pgfqpoint{1.823322in}{2.029526in}}%
\pgfpathlineto{\pgfqpoint{1.827881in}{2.083735in}}%
\pgfpathlineto{\pgfqpoint{1.832440in}{2.094577in}}%
\pgfpathlineto{\pgfqpoint{1.837000in}{2.170468in}}%
\pgfpathlineto{\pgfqpoint{1.841559in}{2.072893in}}%
\pgfpathlineto{\pgfqpoint{1.846118in}{2.170468in}}%
\pgfpathlineto{\pgfqpoint{1.850677in}{2.018685in}}%
\pgfpathlineto{\pgfqpoint{1.855237in}{2.062052in}}%
\pgfpathlineto{\pgfqpoint{1.859796in}{2.062052in}}%
\pgfpathlineto{\pgfqpoint{1.864355in}{2.148785in}}%
\pgfpathlineto{\pgfqpoint{1.873474in}{2.192152in}}%
\pgfpathlineto{\pgfqpoint{1.878033in}{2.246360in}}%
\pgfpathlineto{\pgfqpoint{1.882592in}{2.137943in}}%
\pgfpathlineto{\pgfqpoint{1.887151in}{2.127102in}}%
\pgfpathlineto{\pgfqpoint{1.891711in}{2.127102in}}%
\pgfpathlineto{\pgfqpoint{1.896270in}{2.181310in}}%
\pgfpathlineto{\pgfqpoint{1.900829in}{2.051210in}}%
\pgfpathlineto{\pgfqpoint{1.905388in}{2.018685in}}%
\pgfpathlineto{\pgfqpoint{1.909948in}{2.213835in}}%
\pgfpathlineto{\pgfqpoint{1.914507in}{2.018685in}}%
\pgfpathlineto{\pgfqpoint{1.919066in}{2.181310in}}%
\pgfpathlineto{\pgfqpoint{1.923625in}{2.127102in}}%
\pgfpathlineto{\pgfqpoint{1.928185in}{2.213835in}}%
\pgfpathlineto{\pgfqpoint{1.932744in}{2.127102in}}%
\pgfpathlineto{\pgfqpoint{1.937303in}{2.083735in}}%
\pgfpathlineto{\pgfqpoint{1.941862in}{2.116260in}}%
\pgfpathlineto{\pgfqpoint{1.946422in}{2.278885in}}%
\pgfpathlineto{\pgfqpoint{1.950981in}{2.170468in}}%
\pgfpathlineto{\pgfqpoint{1.955540in}{2.127102in}}%
\pgfpathlineto{\pgfqpoint{1.960099in}{2.192152in}}%
\pgfpathlineto{\pgfqpoint{1.964659in}{2.137943in}}%
\pgfpathlineto{\pgfqpoint{1.969218in}{2.224677in}}%
\pgfpathlineto{\pgfqpoint{1.973777in}{2.235518in}}%
\pgfpathlineto{\pgfqpoint{1.978336in}{2.127102in}}%
\pgfpathlineto{\pgfqpoint{1.982895in}{2.246360in}}%
\pgfpathlineto{\pgfqpoint{1.987455in}{2.062052in}}%
\pgfpathlineto{\pgfqpoint{1.992014in}{2.322252in}}%
\pgfpathlineto{\pgfqpoint{1.996573in}{2.365619in}}%
\pgfpathlineto{\pgfqpoint{2.001132in}{2.148785in}}%
\pgfpathlineto{\pgfqpoint{2.005692in}{2.235518in}}%
\pgfpathlineto{\pgfqpoint{2.010251in}{2.235518in}}%
\pgfpathlineto{\pgfqpoint{2.014810in}{2.159627in}}%
\pgfpathlineto{\pgfqpoint{2.019369in}{2.127102in}}%
\pgfpathlineto{\pgfqpoint{2.023929in}{2.257202in}}%
\pgfpathlineto{\pgfqpoint{2.028488in}{2.224677in}}%
\pgfpathlineto{\pgfqpoint{2.033047in}{2.213835in}}%
\pgfpathlineto{\pgfqpoint{2.037606in}{2.311410in}}%
\pgfpathlineto{\pgfqpoint{2.042166in}{2.202993in}}%
\pgfpathlineto{\pgfqpoint{2.046725in}{2.148785in}}%
\pgfpathlineto{\pgfqpoint{2.051284in}{2.354777in}}%
\pgfpathlineto{\pgfqpoint{2.055843in}{2.246360in}}%
\pgfpathlineto{\pgfqpoint{2.060403in}{2.072893in}}%
\pgfpathlineto{\pgfqpoint{2.064962in}{2.246360in}}%
\pgfpathlineto{\pgfqpoint{2.069521in}{2.376460in}}%
\pgfpathlineto{\pgfqpoint{2.074080in}{2.137943in}}%
\pgfpathlineto{\pgfqpoint{2.078640in}{2.202993in}}%
\pgfpathlineto{\pgfqpoint{2.087758in}{2.202993in}}%
\pgfpathlineto{\pgfqpoint{2.092317in}{2.148785in}}%
\pgfpathlineto{\pgfqpoint{2.096877in}{2.354777in}}%
\pgfpathlineto{\pgfqpoint{2.101436in}{2.300568in}}%
\pgfpathlineto{\pgfqpoint{2.105995in}{2.311410in}}%
\pgfpathlineto{\pgfqpoint{2.110554in}{2.268043in}}%
\pgfpathlineto{\pgfqpoint{2.115114in}{2.311410in}}%
\pgfpathlineto{\pgfqpoint{2.119673in}{2.300568in}}%
\pgfpathlineto{\pgfqpoint{2.124232in}{2.333094in}}%
\pgfpathlineto{\pgfqpoint{2.128791in}{2.278885in}}%
\pgfpathlineto{\pgfqpoint{2.133350in}{2.311410in}}%
\pgfpathlineto{\pgfqpoint{2.137910in}{2.289727in}}%
\pgfpathlineto{\pgfqpoint{2.142469in}{2.311410in}}%
\pgfpathlineto{\pgfqpoint{2.147028in}{2.289727in}}%
\pgfpathlineto{\pgfqpoint{2.151587in}{2.311410in}}%
\pgfpathlineto{\pgfqpoint{2.156147in}{2.289727in}}%
\pgfpathlineto{\pgfqpoint{2.160706in}{2.430669in}}%
\pgfpathlineto{\pgfqpoint{2.165265in}{2.268043in}}%
\pgfpathlineto{\pgfqpoint{2.169824in}{2.311410in}}%
\pgfpathlineto{\pgfqpoint{2.174384in}{2.322252in}}%
\pgfpathlineto{\pgfqpoint{2.178943in}{2.322252in}}%
\pgfpathlineto{\pgfqpoint{2.183502in}{2.484877in}}%
\pgfpathlineto{\pgfqpoint{2.188061in}{2.224677in}}%
\pgfpathlineto{\pgfqpoint{2.192621in}{2.246360in}}%
\pgfpathlineto{\pgfqpoint{2.197180in}{2.365619in}}%
\pgfpathlineto{\pgfqpoint{2.201739in}{2.593294in}}%
\pgfpathlineto{\pgfqpoint{2.206298in}{2.387302in}}%
\pgfpathlineto{\pgfqpoint{2.210858in}{2.430669in}}%
\pgfpathlineto{\pgfqpoint{2.215417in}{2.408985in}}%
\pgfpathlineto{\pgfqpoint{2.219976in}{2.495719in}}%
\pgfpathlineto{\pgfqpoint{2.224535in}{2.376460in}}%
\pgfpathlineto{\pgfqpoint{2.229095in}{2.300568in}}%
\pgfpathlineto{\pgfqpoint{2.233654in}{2.246360in}}%
\pgfpathlineto{\pgfqpoint{2.238213in}{2.495719in}}%
\pgfpathlineto{\pgfqpoint{2.242772in}{2.387302in}}%
\pgfpathlineto{\pgfqpoint{2.247332in}{2.224677in}}%
\pgfpathlineto{\pgfqpoint{2.251891in}{2.549927in}}%
\pgfpathlineto{\pgfqpoint{2.256450in}{2.441510in}}%
\pgfpathlineto{\pgfqpoint{2.261009in}{2.365619in}}%
\pgfpathlineto{\pgfqpoint{2.265569in}{2.387302in}}%
\pgfpathlineto{\pgfqpoint{2.270128in}{2.333094in}}%
\pgfpathlineto{\pgfqpoint{2.274687in}{2.539085in}}%
\pgfpathlineto{\pgfqpoint{2.283805in}{2.419827in}}%
\pgfpathlineto{\pgfqpoint{2.288365in}{2.300568in}}%
\pgfpathlineto{\pgfqpoint{2.292924in}{2.343935in}}%
\pgfpathlineto{\pgfqpoint{2.297483in}{2.582452in}}%
\pgfpathlineto{\pgfqpoint{2.302042in}{2.452352in}}%
\pgfpathlineto{\pgfqpoint{2.306602in}{2.463194in}}%
\pgfpathlineto{\pgfqpoint{2.311161in}{2.408985in}}%
\pgfpathlineto{\pgfqpoint{2.315720in}{2.604135in}}%
\pgfpathlineto{\pgfqpoint{2.320279in}{2.463194in}}%
\pgfpathlineto{\pgfqpoint{2.329398in}{2.398144in}}%
\pgfpathlineto{\pgfqpoint{2.333957in}{2.441510in}}%
\pgfpathlineto{\pgfqpoint{2.338516in}{2.343935in}}%
\pgfpathlineto{\pgfqpoint{2.343076in}{2.484877in}}%
\pgfpathlineto{\pgfqpoint{2.347635in}{2.495719in}}%
\pgfpathlineto{\pgfqpoint{2.352194in}{2.376460in}}%
\pgfpathlineto{\pgfqpoint{2.356753in}{2.549927in}}%
\pgfpathlineto{\pgfqpoint{2.361313in}{2.506560in}}%
\pgfpathlineto{\pgfqpoint{2.365872in}{2.441510in}}%
\pgfpathlineto{\pgfqpoint{2.370431in}{2.571610in}}%
\pgfpathlineto{\pgfqpoint{2.374990in}{2.452352in}}%
\pgfpathlineto{\pgfqpoint{2.379550in}{2.506560in}}%
\pgfpathlineto{\pgfqpoint{2.384109in}{2.398144in}}%
\pgfpathlineto{\pgfqpoint{2.388668in}{2.484877in}}%
\pgfpathlineto{\pgfqpoint{2.393227in}{2.495719in}}%
\pgfpathlineto{\pgfqpoint{2.397787in}{2.474035in}}%
\pgfpathlineto{\pgfqpoint{2.402346in}{2.322252in}}%
\pgfpathlineto{\pgfqpoint{2.406905in}{2.463194in}}%
\pgfpathlineto{\pgfqpoint{2.411464in}{2.463194in}}%
\pgfpathlineto{\pgfqpoint{2.416024in}{2.495719in}}%
\pgfpathlineto{\pgfqpoint{2.420583in}{2.430669in}}%
\pgfpathlineto{\pgfqpoint{2.425142in}{2.506560in}}%
\pgfpathlineto{\pgfqpoint{2.429701in}{2.539085in}}%
\pgfpathlineto{\pgfqpoint{2.434261in}{2.517402in}}%
\pgfpathlineto{\pgfqpoint{2.438820in}{2.571610in}}%
\pgfpathlineto{\pgfqpoint{2.443379in}{2.484877in}}%
\pgfpathlineto{\pgfqpoint{2.447938in}{2.517402in}}%
\pgfpathlineto{\pgfqpoint{2.452497in}{2.484877in}}%
\pgfpathlineto{\pgfqpoint{2.457057in}{2.636661in}}%
\pgfpathlineto{\pgfqpoint{2.461616in}{2.506560in}}%
\pgfpathlineto{\pgfqpoint{2.466175in}{2.549927in}}%
\pgfpathlineto{\pgfqpoint{2.470734in}{2.441510in}}%
\pgfpathlineto{\pgfqpoint{2.475294in}{2.452352in}}%
\pgfpathlineto{\pgfqpoint{2.479853in}{2.495719in}}%
\pgfpathlineto{\pgfqpoint{2.484412in}{2.463194in}}%
\pgfpathlineto{\pgfqpoint{2.488971in}{2.625819in}}%
\pgfpathlineto{\pgfqpoint{2.493531in}{2.474035in}}%
\pgfpathlineto{\pgfqpoint{2.498090in}{2.593294in}}%
\pgfpathlineto{\pgfqpoint{2.502649in}{2.484877in}}%
\pgfpathlineto{\pgfqpoint{2.507208in}{2.560769in}}%
\pgfpathlineto{\pgfqpoint{2.511768in}{2.658344in}}%
\pgfpathlineto{\pgfqpoint{2.516327in}{2.539085in}}%
\pgfpathlineto{\pgfqpoint{2.520886in}{2.484877in}}%
\pgfpathlineto{\pgfqpoint{2.525445in}{2.636661in}}%
\pgfpathlineto{\pgfqpoint{2.530005in}{2.604135in}}%
\pgfpathlineto{\pgfqpoint{2.534564in}{2.506560in}}%
\pgfpathlineto{\pgfqpoint{2.539123in}{2.463194in}}%
\pgfpathlineto{\pgfqpoint{2.543682in}{2.614977in}}%
\pgfpathlineto{\pgfqpoint{2.548242in}{2.669186in}}%
\pgfpathlineto{\pgfqpoint{2.552801in}{2.441510in}}%
\pgfpathlineto{\pgfqpoint{2.557360in}{2.441510in}}%
\pgfpathlineto{\pgfqpoint{2.561919in}{2.571610in}}%
\pgfpathlineto{\pgfqpoint{2.566479in}{2.571610in}}%
\pgfpathlineto{\pgfqpoint{2.571038in}{2.560769in}}%
\pgfpathlineto{\pgfqpoint{2.575597in}{2.495719in}}%
\pgfpathlineto{\pgfqpoint{2.580156in}{2.658344in}}%
\pgfpathlineto{\pgfqpoint{2.584716in}{2.560769in}}%
\pgfpathlineto{\pgfqpoint{2.589275in}{2.614977in}}%
\pgfpathlineto{\pgfqpoint{2.593834in}{2.582452in}}%
\pgfpathlineto{\pgfqpoint{2.598393in}{2.582452in}}%
\pgfpathlineto{\pgfqpoint{2.602952in}{2.484877in}}%
\pgfpathlineto{\pgfqpoint{2.607512in}{2.604135in}}%
\pgfpathlineto{\pgfqpoint{2.616630in}{2.560769in}}%
\pgfpathlineto{\pgfqpoint{2.621189in}{2.484877in}}%
\pgfpathlineto{\pgfqpoint{2.625749in}{2.560769in}}%
\pgfpathlineto{\pgfqpoint{2.630308in}{2.528244in}}%
\pgfpathlineto{\pgfqpoint{2.634867in}{2.625819in}}%
\pgfpathlineto{\pgfqpoint{2.639426in}{2.528244in}}%
\pgfpathlineto{\pgfqpoint{2.653104in}{2.625819in}}%
\pgfpathlineto{\pgfqpoint{2.657663in}{2.614977in}}%
\pgfpathlineto{\pgfqpoint{2.662223in}{2.539085in}}%
\pgfpathlineto{\pgfqpoint{2.666782in}{2.680027in}}%
\pgfpathlineto{\pgfqpoint{2.671341in}{2.484877in}}%
\pgfpathlineto{\pgfqpoint{2.675900in}{2.799286in}}%
\pgfpathlineto{\pgfqpoint{2.680460in}{2.712552in}}%
\pgfpathlineto{\pgfqpoint{2.685019in}{2.604135in}}%
\pgfpathlineto{\pgfqpoint{2.689578in}{2.690869in}}%
\pgfpathlineto{\pgfqpoint{2.694137in}{2.582452in}}%
\pgfpathlineto{\pgfqpoint{2.698697in}{2.604135in}}%
\pgfpathlineto{\pgfqpoint{2.703256in}{2.614977in}}%
\pgfpathlineto{\pgfqpoint{2.707815in}{2.669186in}}%
\pgfpathlineto{\pgfqpoint{2.712374in}{2.614977in}}%
\pgfpathlineto{\pgfqpoint{2.716934in}{2.636661in}}%
\pgfpathlineto{\pgfqpoint{2.721493in}{2.441510in}}%
\pgfpathlineto{\pgfqpoint{2.726052in}{2.636661in}}%
\pgfpathlineto{\pgfqpoint{2.730611in}{2.528244in}}%
\pgfpathlineto{\pgfqpoint{2.735171in}{2.658344in}}%
\pgfpathlineto{\pgfqpoint{2.739730in}{2.701711in}}%
\pgfpathlineto{\pgfqpoint{2.744289in}{2.680027in}}%
\pgfpathlineto{\pgfqpoint{2.748848in}{2.755919in}}%
\pgfpathlineto{\pgfqpoint{2.753407in}{2.745077in}}%
\pgfpathlineto{\pgfqpoint{2.757967in}{2.528244in}}%
\pgfpathlineto{\pgfqpoint{2.762526in}{2.680027in}}%
\pgfpathlineto{\pgfqpoint{2.767085in}{2.604135in}}%
\pgfpathlineto{\pgfqpoint{2.771644in}{2.669186in}}%
\pgfpathlineto{\pgfqpoint{2.776204in}{2.582452in}}%
\pgfpathlineto{\pgfqpoint{2.780763in}{2.604135in}}%
\pgfpathlineto{\pgfqpoint{2.785322in}{2.745077in}}%
\pgfpathlineto{\pgfqpoint{2.789881in}{2.788444in}}%
\pgfpathlineto{\pgfqpoint{2.794441in}{2.864336in}}%
\pgfpathlineto{\pgfqpoint{2.799000in}{2.560769in}}%
\pgfpathlineto{\pgfqpoint{2.803559in}{2.463194in}}%
\pgfpathlineto{\pgfqpoint{2.808118in}{2.810127in}}%
\pgfpathlineto{\pgfqpoint{2.812678in}{2.755919in}}%
\pgfpathlineto{\pgfqpoint{2.817237in}{2.799286in}}%
\pgfpathlineto{\pgfqpoint{2.821796in}{2.614977in}}%
\pgfpathlineto{\pgfqpoint{2.826355in}{2.723394in}}%
\pgfpathlineto{\pgfqpoint{2.830915in}{2.680027in}}%
\pgfpathlineto{\pgfqpoint{2.835474in}{2.539085in}}%
\pgfpathlineto{\pgfqpoint{2.840033in}{2.593294in}}%
\pgfpathlineto{\pgfqpoint{2.844592in}{2.669186in}}%
\pgfpathlineto{\pgfqpoint{2.849152in}{2.723394in}}%
\pgfpathlineto{\pgfqpoint{2.853711in}{2.831811in}}%
\pgfpathlineto{\pgfqpoint{2.858270in}{2.723394in}}%
\pgfpathlineto{\pgfqpoint{2.862829in}{2.777602in}}%
\pgfpathlineto{\pgfqpoint{2.867389in}{2.582452in}}%
\pgfpathlineto{\pgfqpoint{2.871948in}{2.647502in}}%
\pgfpathlineto{\pgfqpoint{2.876507in}{2.658344in}}%
\pgfpathlineto{\pgfqpoint{2.881066in}{2.614977in}}%
\pgfpathlineto{\pgfqpoint{2.885626in}{2.669186in}}%
\pgfpathlineto{\pgfqpoint{2.890185in}{2.745077in}}%
\pgfpathlineto{\pgfqpoint{2.894744in}{2.658344in}}%
\pgfpathlineto{\pgfqpoint{2.899303in}{2.745077in}}%
\pgfpathlineto{\pgfqpoint{2.903862in}{2.777602in}}%
\pgfpathlineto{\pgfqpoint{2.908422in}{2.820969in}}%
\pgfpathlineto{\pgfqpoint{2.912981in}{2.690869in}}%
\pgfpathlineto{\pgfqpoint{2.922099in}{2.734236in}}%
\pgfpathlineto{\pgfqpoint{2.926659in}{2.712552in}}%
\pgfpathlineto{\pgfqpoint{2.931218in}{2.701711in}}%
\pgfpathlineto{\pgfqpoint{2.935777in}{2.745077in}}%
\pgfpathlineto{\pgfqpoint{2.940336in}{2.766761in}}%
\pgfpathlineto{\pgfqpoint{2.944896in}{2.755919in}}%
\pgfpathlineto{\pgfqpoint{2.949455in}{2.766761in}}%
\pgfpathlineto{\pgfqpoint{2.954014in}{2.907702in}}%
\pgfpathlineto{\pgfqpoint{2.963133in}{2.647502in}}%
\pgfpathlineto{\pgfqpoint{2.967692in}{2.734236in}}%
\pgfpathlineto{\pgfqpoint{2.972251in}{2.712552in}}%
\pgfpathlineto{\pgfqpoint{2.976810in}{2.766761in}}%
\pgfpathlineto{\pgfqpoint{2.981370in}{2.712552in}}%
\pgfpathlineto{\pgfqpoint{2.985929in}{2.896861in}}%
\pgfpathlineto{\pgfqpoint{2.990488in}{2.929386in}}%
\pgfpathlineto{\pgfqpoint{2.995047in}{2.886019in}}%
\pgfpathlineto{\pgfqpoint{2.999607in}{2.875177in}}%
\pgfpathlineto{\pgfqpoint{3.004166in}{2.755919in}}%
\pgfpathlineto{\pgfqpoint{3.008725in}{2.886019in}}%
\pgfpathlineto{\pgfqpoint{3.013284in}{2.658344in}}%
\pgfpathlineto{\pgfqpoint{3.017844in}{2.820969in}}%
\pgfpathlineto{\pgfqpoint{3.022403in}{2.831811in}}%
\pgfpathlineto{\pgfqpoint{3.026962in}{2.755919in}}%
\pgfpathlineto{\pgfqpoint{3.031521in}{2.712552in}}%
\pgfpathlineto{\pgfqpoint{3.036081in}{2.961911in}}%
\pgfpathlineto{\pgfqpoint{3.040640in}{2.788444in}}%
\pgfpathlineto{\pgfqpoint{3.045199in}{2.907702in}}%
\pgfpathlineto{\pgfqpoint{3.049758in}{2.810127in}}%
\pgfpathlineto{\pgfqpoint{3.054317in}{2.886019in}}%
\pgfpathlineto{\pgfqpoint{3.058877in}{2.690869in}}%
\pgfpathlineto{\pgfqpoint{3.063436in}{2.680027in}}%
\pgfpathlineto{\pgfqpoint{3.067995in}{2.755919in}}%
\pgfpathlineto{\pgfqpoint{3.072554in}{2.777602in}}%
\pgfpathlineto{\pgfqpoint{3.077114in}{2.929386in}}%
\pgfpathlineto{\pgfqpoint{3.081673in}{2.820969in}}%
\pgfpathlineto{\pgfqpoint{3.086232in}{2.745077in}}%
\pgfpathlineto{\pgfqpoint{3.090791in}{2.842652in}}%
\pgfpathlineto{\pgfqpoint{3.095351in}{2.745077in}}%
\pgfpathlineto{\pgfqpoint{3.099910in}{2.777602in}}%
\pgfpathlineto{\pgfqpoint{3.104469in}{2.864336in}}%
\pgfpathlineto{\pgfqpoint{3.109028in}{2.755919in}}%
\pgfpathlineto{\pgfqpoint{3.113588in}{2.886019in}}%
\pgfpathlineto{\pgfqpoint{3.118147in}{2.983594in}}%
\pgfpathlineto{\pgfqpoint{3.122706in}{2.907702in}}%
\pgfpathlineto{\pgfqpoint{3.127265in}{2.777602in}}%
\pgfpathlineto{\pgfqpoint{3.131825in}{3.026961in}}%
\pgfpathlineto{\pgfqpoint{3.136384in}{2.831811in}}%
\pgfpathlineto{\pgfqpoint{3.140943in}{2.972753in}}%
\pgfpathlineto{\pgfqpoint{3.145502in}{3.048644in}}%
\pgfpathlineto{\pgfqpoint{3.150062in}{2.875177in}}%
\pgfpathlineto{\pgfqpoint{3.154621in}{2.799286in}}%
\pgfpathlineto{\pgfqpoint{3.159180in}{2.940228in}}%
\pgfpathlineto{\pgfqpoint{3.163739in}{2.788444in}}%
\pgfpathlineto{\pgfqpoint{3.168299in}{2.951069in}}%
\pgfpathlineto{\pgfqpoint{3.172858in}{2.831811in}}%
\pgfpathlineto{\pgfqpoint{3.177417in}{3.016119in}}%
\pgfpathlineto{\pgfqpoint{3.181976in}{2.972753in}}%
\pgfpathlineto{\pgfqpoint{3.186536in}{3.059486in}}%
\pgfpathlineto{\pgfqpoint{3.191095in}{2.853494in}}%
\pgfpathlineto{\pgfqpoint{3.195654in}{2.745077in}}%
\pgfpathlineto{\pgfqpoint{3.200213in}{3.026961in}}%
\pgfpathlineto{\pgfqpoint{3.204772in}{2.842652in}}%
\pgfpathlineto{\pgfqpoint{3.209332in}{2.853494in}}%
\pgfpathlineto{\pgfqpoint{3.213891in}{3.005278in}}%
\pgfpathlineto{\pgfqpoint{3.218450in}{2.799286in}}%
\pgfpathlineto{\pgfqpoint{3.223009in}{2.864336in}}%
\pgfpathlineto{\pgfqpoint{3.227569in}{2.994436in}}%
\pgfpathlineto{\pgfqpoint{3.236687in}{2.831811in}}%
\pgfpathlineto{\pgfqpoint{3.241246in}{2.896861in}}%
\pgfpathlineto{\pgfqpoint{3.245806in}{2.907702in}}%
\pgfpathlineto{\pgfqpoint{3.250365in}{2.961911in}}%
\pgfpathlineto{\pgfqpoint{3.254924in}{2.940228in}}%
\pgfpathlineto{\pgfqpoint{3.259483in}{2.929386in}}%
\pgfpathlineto{\pgfqpoint{3.264043in}{2.766761in}}%
\pgfpathlineto{\pgfqpoint{3.268602in}{2.983594in}}%
\pgfpathlineto{\pgfqpoint{3.273161in}{2.918544in}}%
\pgfpathlineto{\pgfqpoint{3.277720in}{3.048644in}}%
\pgfpathlineto{\pgfqpoint{3.282280in}{2.777602in}}%
\pgfpathlineto{\pgfqpoint{3.286839in}{2.810127in}}%
\pgfpathlineto{\pgfqpoint{3.291398in}{2.745077in}}%
\pgfpathlineto{\pgfqpoint{3.295957in}{3.005278in}}%
\pgfpathlineto{\pgfqpoint{3.300517in}{3.157061in}}%
\pgfpathlineto{\pgfqpoint{3.309635in}{2.940228in}}%
\pgfpathlineto{\pgfqpoint{3.314194in}{2.886019in}}%
\pgfpathlineto{\pgfqpoint{3.318754in}{2.886019in}}%
\pgfpathlineto{\pgfqpoint{3.323313in}{2.972753in}}%
\pgfpathlineto{\pgfqpoint{3.327872in}{2.864336in}}%
\pgfpathlineto{\pgfqpoint{3.332431in}{3.092011in}}%
\pgfpathlineto{\pgfqpoint{3.341550in}{2.886019in}}%
\pgfpathlineto{\pgfqpoint{3.346109in}{2.940228in}}%
\pgfpathlineto{\pgfqpoint{3.350668in}{2.875177in}}%
\pgfpathlineto{\pgfqpoint{3.355228in}{3.005278in}}%
\pgfpathlineto{\pgfqpoint{3.359787in}{3.048644in}}%
\pgfpathlineto{\pgfqpoint{3.364346in}{2.994436in}}%
\pgfpathlineto{\pgfqpoint{3.368905in}{2.972753in}}%
\pgfpathlineto{\pgfqpoint{3.373464in}{3.026961in}}%
\pgfpathlineto{\pgfqpoint{3.378024in}{2.853494in}}%
\pgfpathlineto{\pgfqpoint{3.387142in}{3.113694in}}%
\pgfpathlineto{\pgfqpoint{3.391701in}{2.929386in}}%
\pgfpathlineto{\pgfqpoint{3.396261in}{2.918544in}}%
\pgfpathlineto{\pgfqpoint{3.400820in}{2.842652in}}%
\pgfpathlineto{\pgfqpoint{3.405379in}{2.886019in}}%
\pgfpathlineto{\pgfqpoint{3.409938in}{2.961911in}}%
\pgfpathlineto{\pgfqpoint{3.414498in}{3.135378in}}%
\pgfpathlineto{\pgfqpoint{3.419057in}{2.994436in}}%
\pgfpathlineto{\pgfqpoint{3.423616in}{2.918544in}}%
\pgfpathlineto{\pgfqpoint{3.428175in}{2.983594in}}%
\pgfpathlineto{\pgfqpoint{3.432735in}{2.951069in}}%
\pgfpathlineto{\pgfqpoint{3.437294in}{2.951069in}}%
\pgfpathlineto{\pgfqpoint{3.441853in}{2.831811in}}%
\pgfpathlineto{\pgfqpoint{3.446412in}{2.961911in}}%
\pgfpathlineto{\pgfqpoint{3.450972in}{2.929386in}}%
\pgfpathlineto{\pgfqpoint{3.455531in}{2.918544in}}%
\pgfpathlineto{\pgfqpoint{3.460090in}{3.005278in}}%
\pgfpathlineto{\pgfqpoint{3.464649in}{2.820969in}}%
\pgfpathlineto{\pgfqpoint{3.469209in}{3.048644in}}%
\pgfpathlineto{\pgfqpoint{3.473768in}{2.972753in}}%
\pgfpathlineto{\pgfqpoint{3.478327in}{2.940228in}}%
\pgfpathlineto{\pgfqpoint{3.482886in}{2.940228in}}%
\pgfpathlineto{\pgfqpoint{3.487446in}{3.330528in}}%
\pgfpathlineto{\pgfqpoint{3.492005in}{2.994436in}}%
\pgfpathlineto{\pgfqpoint{3.496564in}{2.951069in}}%
\pgfpathlineto{\pgfqpoint{3.501123in}{2.972753in}}%
\pgfpathlineto{\pgfqpoint{3.505683in}{3.124536in}}%
\pgfpathlineto{\pgfqpoint{3.510242in}{2.940228in}}%
\pgfpathlineto{\pgfqpoint{3.514801in}{3.070328in}}%
\pgfpathlineto{\pgfqpoint{3.519360in}{3.026961in}}%
\pgfpathlineto{\pgfqpoint{3.523919in}{3.048644in}}%
\pgfpathlineto{\pgfqpoint{3.528479in}{3.189586in}}%
\pgfpathlineto{\pgfqpoint{3.533038in}{2.777602in}}%
\pgfpathlineto{\pgfqpoint{3.537597in}{3.026961in}}%
\pgfpathlineto{\pgfqpoint{3.542156in}{3.016119in}}%
\pgfpathlineto{\pgfqpoint{3.546716in}{3.070328in}}%
\pgfpathlineto{\pgfqpoint{3.551275in}{3.189586in}}%
\pgfpathlineto{\pgfqpoint{3.555834in}{2.994436in}}%
\pgfpathlineto{\pgfqpoint{3.560393in}{2.940228in}}%
\pgfpathlineto{\pgfqpoint{3.564953in}{3.026961in}}%
\pgfpathlineto{\pgfqpoint{3.569512in}{3.059486in}}%
\pgfpathlineto{\pgfqpoint{3.574071in}{3.232953in}}%
\pgfpathlineto{\pgfqpoint{3.578630in}{3.124536in}}%
\pgfpathlineto{\pgfqpoint{3.583190in}{2.983594in}}%
\pgfpathlineto{\pgfqpoint{3.587749in}{3.157061in}}%
\pgfpathlineto{\pgfqpoint{3.592308in}{3.005278in}}%
\pgfpathlineto{\pgfqpoint{3.596867in}{2.994436in}}%
\pgfpathlineto{\pgfqpoint{3.601427in}{2.972753in}}%
\pgfpathlineto{\pgfqpoint{3.605986in}{2.994436in}}%
\pgfpathlineto{\pgfqpoint{3.610545in}{2.951069in}}%
\pgfpathlineto{\pgfqpoint{3.615104in}{2.918544in}}%
\pgfpathlineto{\pgfqpoint{3.619664in}{3.037803in}}%
\pgfpathlineto{\pgfqpoint{3.624223in}{3.048644in}}%
\pgfpathlineto{\pgfqpoint{3.628782in}{2.940228in}}%
\pgfpathlineto{\pgfqpoint{3.633341in}{3.059486in}}%
\pgfpathlineto{\pgfqpoint{3.637901in}{3.026961in}}%
\pgfpathlineto{\pgfqpoint{3.642460in}{3.167903in}}%
\pgfpathlineto{\pgfqpoint{3.647019in}{3.146219in}}%
\pgfpathlineto{\pgfqpoint{3.651578in}{3.102853in}}%
\pgfpathlineto{\pgfqpoint{3.656138in}{3.157061in}}%
\pgfpathlineto{\pgfqpoint{3.660697in}{3.102853in}}%
\pgfpathlineto{\pgfqpoint{3.665256in}{3.070328in}}%
\pgfpathlineto{\pgfqpoint{3.674374in}{3.113694in}}%
\pgfpathlineto{\pgfqpoint{3.678934in}{3.059486in}}%
\pgfpathlineto{\pgfqpoint{3.683493in}{2.896861in}}%
\pgfpathlineto{\pgfqpoint{3.688052in}{3.178744in}}%
\pgfpathlineto{\pgfqpoint{3.692611in}{3.037803in}}%
\pgfpathlineto{\pgfqpoint{3.697171in}{2.994436in}}%
\pgfpathlineto{\pgfqpoint{3.701730in}{3.167903in}}%
\pgfpathlineto{\pgfqpoint{3.706289in}{3.113694in}}%
\pgfpathlineto{\pgfqpoint{3.710848in}{3.037803in}}%
\pgfpathlineto{\pgfqpoint{3.715408in}{3.048644in}}%
\pgfpathlineto{\pgfqpoint{3.719967in}{3.070328in}}%
\pgfpathlineto{\pgfqpoint{3.724526in}{3.135378in}}%
\pgfpathlineto{\pgfqpoint{3.729085in}{3.157061in}}%
\pgfpathlineto{\pgfqpoint{3.733645in}{3.102853in}}%
\pgfpathlineto{\pgfqpoint{3.738204in}{3.200428in}}%
\pgfpathlineto{\pgfqpoint{3.742763in}{2.951069in}}%
\pgfpathlineto{\pgfqpoint{3.747322in}{3.005278in}}%
\pgfpathlineto{\pgfqpoint{3.751882in}{3.189586in}}%
\pgfpathlineto{\pgfqpoint{3.756441in}{3.135378in}}%
\pgfpathlineto{\pgfqpoint{3.761000in}{3.157061in}}%
\pgfpathlineto{\pgfqpoint{3.765559in}{3.102853in}}%
\pgfpathlineto{\pgfqpoint{3.770119in}{2.907702in}}%
\pgfpathlineto{\pgfqpoint{3.774678in}{3.059486in}}%
\pgfpathlineto{\pgfqpoint{3.779237in}{3.254636in}}%
\pgfpathlineto{\pgfqpoint{3.783796in}{2.994436in}}%
\pgfpathlineto{\pgfqpoint{3.788356in}{3.081169in}}%
\pgfpathlineto{\pgfqpoint{3.792915in}{3.026961in}}%
\pgfpathlineto{\pgfqpoint{3.797474in}{3.070328in}}%
\pgfpathlineto{\pgfqpoint{3.802033in}{3.167903in}}%
\pgfpathlineto{\pgfqpoint{3.806593in}{3.232953in}}%
\pgfpathlineto{\pgfqpoint{3.811152in}{3.026961in}}%
\pgfpathlineto{\pgfqpoint{3.815711in}{3.037803in}}%
\pgfpathlineto{\pgfqpoint{3.820270in}{3.135378in}}%
\pgfpathlineto{\pgfqpoint{3.824829in}{3.319686in}}%
\pgfpathlineto{\pgfqpoint{3.829389in}{3.200428in}}%
\pgfpathlineto{\pgfqpoint{3.833948in}{3.243795in}}%
\pgfpathlineto{\pgfqpoint{3.838507in}{3.243795in}}%
\pgfpathlineto{\pgfqpoint{3.843066in}{3.113694in}}%
\pgfpathlineto{\pgfqpoint{3.847626in}{3.081169in}}%
\pgfpathlineto{\pgfqpoint{3.852185in}{3.113694in}}%
\pgfpathlineto{\pgfqpoint{3.856744in}{3.232953in}}%
\pgfpathlineto{\pgfqpoint{3.861303in}{3.265478in}}%
\pgfpathlineto{\pgfqpoint{3.865863in}{3.124536in}}%
\pgfpathlineto{\pgfqpoint{3.870422in}{3.092011in}}%
\pgfpathlineto{\pgfqpoint{3.874981in}{3.124536in}}%
\pgfpathlineto{\pgfqpoint{3.879540in}{3.135378in}}%
\pgfpathlineto{\pgfqpoint{3.884100in}{3.189586in}}%
\pgfpathlineto{\pgfqpoint{3.888659in}{3.102853in}}%
\pgfpathlineto{\pgfqpoint{3.893218in}{3.200428in}}%
\pgfpathlineto{\pgfqpoint{3.897777in}{3.092011in}}%
\pgfpathlineto{\pgfqpoint{3.902337in}{3.081169in}}%
\pgfpathlineto{\pgfqpoint{3.906896in}{3.265478in}}%
\pgfpathlineto{\pgfqpoint{3.911455in}{3.254636in}}%
\pgfpathlineto{\pgfqpoint{3.916014in}{3.222111in}}%
\pgfpathlineto{\pgfqpoint{3.920574in}{3.157061in}}%
\pgfpathlineto{\pgfqpoint{3.929692in}{3.135378in}}%
\pgfpathlineto{\pgfqpoint{3.934251in}{3.243795in}}%
\pgfpathlineto{\pgfqpoint{3.938811in}{2.994436in}}%
\pgfpathlineto{\pgfqpoint{3.943370in}{3.243795in}}%
\pgfpathlineto{\pgfqpoint{3.947929in}{3.048644in}}%
\pgfpathlineto{\pgfqpoint{3.952488in}{3.287161in}}%
\pgfpathlineto{\pgfqpoint{3.957048in}{3.016119in}}%
\pgfpathlineto{\pgfqpoint{3.961607in}{3.146219in}}%
\pgfpathlineto{\pgfqpoint{3.966166in}{3.037803in}}%
\pgfpathlineto{\pgfqpoint{3.970725in}{3.319686in}}%
\pgfpathlineto{\pgfqpoint{3.975284in}{3.200428in}}%
\pgfpathlineto{\pgfqpoint{3.979844in}{3.157061in}}%
\pgfpathlineto{\pgfqpoint{3.984403in}{3.222111in}}%
\pgfpathlineto{\pgfqpoint{3.988962in}{3.319686in}}%
\pgfpathlineto{\pgfqpoint{3.993521in}{3.124536in}}%
\pgfpathlineto{\pgfqpoint{3.998081in}{3.048644in}}%
\pgfpathlineto{\pgfqpoint{4.002640in}{3.178744in}}%
\pgfpathlineto{\pgfqpoint{4.011758in}{3.222111in}}%
\pgfpathlineto{\pgfqpoint{4.016318in}{3.232953in}}%
\pgfpathlineto{\pgfqpoint{4.020877in}{3.124536in}}%
\pgfpathlineto{\pgfqpoint{4.025436in}{3.254636in}}%
\pgfpathlineto{\pgfqpoint{4.029995in}{3.319686in}}%
\pgfpathlineto{\pgfqpoint{4.034555in}{3.243795in}}%
\pgfpathlineto{\pgfqpoint{4.039114in}{3.211269in}}%
\pgfpathlineto{\pgfqpoint{4.043673in}{3.222111in}}%
\pgfpathlineto{\pgfqpoint{4.048232in}{3.135378in}}%
\pgfpathlineto{\pgfqpoint{4.052792in}{3.016119in}}%
\pgfpathlineto{\pgfqpoint{4.057351in}{3.200428in}}%
\pgfpathlineto{\pgfqpoint{4.061910in}{3.070328in}}%
\pgfpathlineto{\pgfqpoint{4.066469in}{3.222111in}}%
\pgfpathlineto{\pgfqpoint{4.071029in}{3.330528in}}%
\pgfpathlineto{\pgfqpoint{4.075588in}{3.254636in}}%
\pgfpathlineto{\pgfqpoint{4.080147in}{3.287161in}}%
\pgfpathlineto{\pgfqpoint{4.084706in}{3.200428in}}%
\pgfpathlineto{\pgfqpoint{4.089266in}{3.254636in}}%
\pgfpathlineto{\pgfqpoint{4.093825in}{3.178744in}}%
\pgfpathlineto{\pgfqpoint{4.098384in}{3.135378in}}%
\pgfpathlineto{\pgfqpoint{4.102943in}{3.167903in}}%
\pgfpathlineto{\pgfqpoint{4.107503in}{3.102853in}}%
\pgfpathlineto{\pgfqpoint{4.112062in}{3.287161in}}%
\pgfpathlineto{\pgfqpoint{4.116621in}{3.298003in}}%
\pgfpathlineto{\pgfqpoint{4.121180in}{3.222111in}}%
\pgfpathlineto{\pgfqpoint{4.125739in}{3.254636in}}%
\pgfpathlineto{\pgfqpoint{4.130299in}{3.243795in}}%
\pgfpathlineto{\pgfqpoint{4.134858in}{3.135378in}}%
\pgfpathlineto{\pgfqpoint{4.139417in}{3.254636in}}%
\pgfpathlineto{\pgfqpoint{4.143976in}{3.232953in}}%
\pgfpathlineto{\pgfqpoint{4.148536in}{3.243795in}}%
\pgfpathlineto{\pgfqpoint{4.153095in}{3.146219in}}%
\pgfpathlineto{\pgfqpoint{4.157654in}{3.211269in}}%
\pgfpathlineto{\pgfqpoint{4.162213in}{3.298003in}}%
\pgfpathlineto{\pgfqpoint{4.166773in}{3.287161in}}%
\pgfpathlineto{\pgfqpoint{4.171332in}{3.287161in}}%
\pgfpathlineto{\pgfqpoint{4.175891in}{3.276320in}}%
\pgfpathlineto{\pgfqpoint{4.180450in}{3.157061in}}%
\pgfpathlineto{\pgfqpoint{4.185010in}{3.243795in}}%
\pgfpathlineto{\pgfqpoint{4.189569in}{3.254636in}}%
\pgfpathlineto{\pgfqpoint{4.198687in}{3.471470in}}%
\pgfpathlineto{\pgfqpoint{4.203247in}{3.254636in}}%
\pgfpathlineto{\pgfqpoint{4.207806in}{3.276320in}}%
\pgfpathlineto{\pgfqpoint{4.216924in}{3.254636in}}%
\pgfpathlineto{\pgfqpoint{4.221484in}{3.200428in}}%
\pgfpathlineto{\pgfqpoint{4.226043in}{3.102853in}}%
\pgfpathlineto{\pgfqpoint{4.230602in}{3.124536in}}%
\pgfpathlineto{\pgfqpoint{4.235161in}{3.287161in}}%
\pgfpathlineto{\pgfqpoint{4.239721in}{3.243795in}}%
\pgfpathlineto{\pgfqpoint{4.244280in}{3.265478in}}%
\pgfpathlineto{\pgfqpoint{4.248839in}{3.232953in}}%
\pgfpathlineto{\pgfqpoint{4.253398in}{3.113694in}}%
\pgfpathlineto{\pgfqpoint{4.257958in}{3.363053in}}%
\pgfpathlineto{\pgfqpoint{4.262517in}{3.330528in}}%
\pgfpathlineto{\pgfqpoint{4.267076in}{3.330528in}}%
\pgfpathlineto{\pgfqpoint{4.271635in}{3.287161in}}%
\pgfpathlineto{\pgfqpoint{4.276195in}{3.308845in}}%
\pgfpathlineto{\pgfqpoint{4.280754in}{3.276320in}}%
\pgfpathlineto{\pgfqpoint{4.285313in}{3.319686in}}%
\pgfpathlineto{\pgfqpoint{4.294431in}{3.157061in}}%
\pgfpathlineto{\pgfqpoint{4.298991in}{3.308845in}}%
\pgfpathlineto{\pgfqpoint{4.303550in}{3.319686in}}%
\pgfpathlineto{\pgfqpoint{4.308109in}{3.417261in}}%
\pgfpathlineto{\pgfqpoint{4.312668in}{3.428103in}}%
\pgfpathlineto{\pgfqpoint{4.317228in}{3.135378in}}%
\pgfpathlineto{\pgfqpoint{4.321787in}{3.135378in}}%
\pgfpathlineto{\pgfqpoint{4.326346in}{3.330528in}}%
\pgfpathlineto{\pgfqpoint{4.330905in}{3.254636in}}%
\pgfpathlineto{\pgfqpoint{4.335465in}{3.287161in}}%
\pgfpathlineto{\pgfqpoint{4.340024in}{3.384736in}}%
\pgfpathlineto{\pgfqpoint{4.344583in}{3.406420in}}%
\pgfpathlineto{\pgfqpoint{4.349142in}{3.048644in}}%
\pgfpathlineto{\pgfqpoint{4.353702in}{3.287161in}}%
\pgfpathlineto{\pgfqpoint{4.358261in}{3.308845in}}%
\pgfpathlineto{\pgfqpoint{4.362820in}{3.395578in}}%
\pgfpathlineto{\pgfqpoint{4.367379in}{3.384736in}}%
\pgfpathlineto{\pgfqpoint{4.376498in}{3.135378in}}%
\pgfpathlineto{\pgfqpoint{4.381057in}{3.363053in}}%
\pgfpathlineto{\pgfqpoint{4.385616in}{3.417261in}}%
\pgfpathlineto{\pgfqpoint{4.390176in}{3.406420in}}%
\pgfpathlineto{\pgfqpoint{4.394735in}{3.330528in}}%
\pgfpathlineto{\pgfqpoint{4.399294in}{3.135378in}}%
\pgfpathlineto{\pgfqpoint{4.403853in}{3.330528in}}%
\pgfpathlineto{\pgfqpoint{4.408413in}{3.373895in}}%
\pgfpathlineto{\pgfqpoint{4.412972in}{3.406420in}}%
\pgfpathlineto{\pgfqpoint{4.417531in}{3.341370in}}%
\pgfpathlineto{\pgfqpoint{4.422090in}{3.384736in}}%
\pgfpathlineto{\pgfqpoint{4.426650in}{3.298003in}}%
\pgfpathlineto{\pgfqpoint{4.431209in}{3.471470in}}%
\pgfpathlineto{\pgfqpoint{4.440327in}{3.352211in}}%
\pgfpathlineto{\pgfqpoint{4.449446in}{3.395578in}}%
\pgfpathlineto{\pgfqpoint{4.454005in}{3.341370in}}%
\pgfpathlineto{\pgfqpoint{4.458564in}{3.319686in}}%
\pgfpathlineto{\pgfqpoint{4.463123in}{3.363053in}}%
\pgfpathlineto{\pgfqpoint{4.467683in}{3.330528in}}%
\pgfpathlineto{\pgfqpoint{4.472242in}{3.363053in}}%
\pgfpathlineto{\pgfqpoint{4.476801in}{3.243795in}}%
\pgfpathlineto{\pgfqpoint{4.481360in}{3.536520in}}%
\pgfpathlineto{\pgfqpoint{4.485920in}{3.330528in}}%
\pgfpathlineto{\pgfqpoint{4.490479in}{3.384736in}}%
\pgfpathlineto{\pgfqpoint{4.495038in}{3.471470in}}%
\pgfpathlineto{\pgfqpoint{4.499597in}{3.319686in}}%
\pgfpathlineto{\pgfqpoint{4.504157in}{3.417261in}}%
\pgfpathlineto{\pgfqpoint{4.508716in}{3.482311in}}%
\pgfpathlineto{\pgfqpoint{4.513275in}{3.276320in}}%
\pgfpathlineto{\pgfqpoint{4.517834in}{3.406420in}}%
\pgfpathlineto{\pgfqpoint{4.522394in}{3.243795in}}%
\pgfpathlineto{\pgfqpoint{4.526953in}{3.298003in}}%
\pgfpathlineto{\pgfqpoint{4.531512in}{3.265478in}}%
\pgfpathlineto{\pgfqpoint{4.536071in}{3.503995in}}%
\pgfpathlineto{\pgfqpoint{4.545190in}{3.265478in}}%
\pgfpathlineto{\pgfqpoint{4.549749in}{3.428103in}}%
\pgfpathlineto{\pgfqpoint{4.554308in}{3.428103in}}%
\pgfpathlineto{\pgfqpoint{4.558868in}{3.363053in}}%
\pgfpathlineto{\pgfqpoint{4.563427in}{3.482311in}}%
\pgfpathlineto{\pgfqpoint{4.572545in}{3.265478in}}%
\pgfpathlineto{\pgfqpoint{4.577105in}{3.352211in}}%
\pgfpathlineto{\pgfqpoint{4.581664in}{3.384736in}}%
\pgfpathlineto{\pgfqpoint{4.586223in}{3.493153in}}%
\pgfpathlineto{\pgfqpoint{4.590782in}{3.092011in}}%
\pgfpathlineto{\pgfqpoint{4.595341in}{3.449786in}}%
\pgfpathlineto{\pgfqpoint{4.599901in}{3.428103in}}%
\pgfpathlineto{\pgfqpoint{4.604460in}{3.503995in}}%
\pgfpathlineto{\pgfqpoint{4.609019in}{3.384736in}}%
\pgfpathlineto{\pgfqpoint{4.613578in}{3.558203in}}%
\pgfpathlineto{\pgfqpoint{4.618138in}{3.308845in}}%
\pgfpathlineto{\pgfqpoint{4.622697in}{3.298003in}}%
\pgfpathlineto{\pgfqpoint{4.627256in}{3.406420in}}%
\pgfpathlineto{\pgfqpoint{4.631815in}{3.438945in}}%
\pgfpathlineto{\pgfqpoint{4.636375in}{3.579887in}}%
\pgfpathlineto{\pgfqpoint{4.640934in}{3.471470in}}%
\pgfpathlineto{\pgfqpoint{4.645493in}{3.471470in}}%
\pgfpathlineto{\pgfqpoint{4.650052in}{3.276320in}}%
\pgfpathlineto{\pgfqpoint{4.654612in}{3.341370in}}%
\pgfpathlineto{\pgfqpoint{4.659171in}{3.341370in}}%
\pgfpathlineto{\pgfqpoint{4.663730in}{3.395578in}}%
\pgfpathlineto{\pgfqpoint{4.668289in}{3.471470in}}%
\pgfpathlineto{\pgfqpoint{4.672849in}{3.438945in}}%
\pgfpathlineto{\pgfqpoint{4.677408in}{3.232953in}}%
\pgfpathlineto{\pgfqpoint{4.681967in}{3.417261in}}%
\pgfpathlineto{\pgfqpoint{4.686526in}{3.384736in}}%
\pgfpathlineto{\pgfqpoint{4.691086in}{3.384736in}}%
\pgfpathlineto{\pgfqpoint{4.695645in}{3.373895in}}%
\pgfpathlineto{\pgfqpoint{4.700204in}{3.438945in}}%
\pgfpathlineto{\pgfqpoint{4.704763in}{3.438945in}}%
\pgfpathlineto{\pgfqpoint{4.709323in}{3.449786in}}%
\pgfpathlineto{\pgfqpoint{4.713882in}{3.417261in}}%
\pgfpathlineto{\pgfqpoint{4.718441in}{3.471470in}}%
\pgfpathlineto{\pgfqpoint{4.727560in}{3.287161in}}%
\pgfpathlineto{\pgfqpoint{4.732119in}{3.428103in}}%
\pgfpathlineto{\pgfqpoint{4.736678in}{3.330528in}}%
\pgfpathlineto{\pgfqpoint{4.741237in}{3.471470in}}%
\pgfpathlineto{\pgfqpoint{4.745796in}{3.460628in}}%
\pgfpathlineto{\pgfqpoint{4.750356in}{3.384736in}}%
\pgfpathlineto{\pgfqpoint{4.754915in}{3.406420in}}%
\pgfpathlineto{\pgfqpoint{4.759474in}{3.547362in}}%
\pgfpathlineto{\pgfqpoint{4.764033in}{3.601570in}}%
\pgfpathlineto{\pgfqpoint{4.768593in}{3.330528in}}%
\pgfpathlineto{\pgfqpoint{4.773152in}{3.417261in}}%
\pgfpathlineto{\pgfqpoint{4.777711in}{3.406420in}}%
\pgfpathlineto{\pgfqpoint{4.782270in}{3.579887in}}%
\pgfpathlineto{\pgfqpoint{4.786830in}{3.449786in}}%
\pgfpathlineto{\pgfqpoint{4.791389in}{3.547362in}}%
\pgfpathlineto{\pgfqpoint{4.795948in}{3.471470in}}%
\pgfpathlineto{\pgfqpoint{4.800507in}{3.558203in}}%
\pgfpathlineto{\pgfqpoint{4.805067in}{3.438945in}}%
\pgfpathlineto{\pgfqpoint{4.809626in}{3.655778in}}%
\pgfpathlineto{\pgfqpoint{4.814185in}{3.569045in}}%
\pgfpathlineto{\pgfqpoint{4.818744in}{3.363053in}}%
\pgfpathlineto{\pgfqpoint{4.823304in}{3.211269in}}%
\pgfpathlineto{\pgfqpoint{4.827863in}{3.601570in}}%
\pgfpathlineto{\pgfqpoint{4.832422in}{3.428103in}}%
\pgfpathlineto{\pgfqpoint{4.836981in}{3.503995in}}%
\pgfpathlineto{\pgfqpoint{4.841541in}{3.298003in}}%
\pgfpathlineto{\pgfqpoint{4.846100in}{3.449786in}}%
\pgfpathlineto{\pgfqpoint{4.850659in}{3.493153in}}%
\pgfpathlineto{\pgfqpoint{4.855218in}{3.655778in}}%
\pgfpathlineto{\pgfqpoint{4.859778in}{3.460628in}}%
\pgfpathlineto{\pgfqpoint{4.864337in}{3.601570in}}%
\pgfpathlineto{\pgfqpoint{4.873455in}{3.395578in}}%
\pgfpathlineto{\pgfqpoint{4.878015in}{3.525678in}}%
\pgfpathlineto{\pgfqpoint{4.882574in}{3.287161in}}%
\pgfpathlineto{\pgfqpoint{4.887133in}{3.428103in}}%
\pgfpathlineto{\pgfqpoint{4.891692in}{3.373895in}}%
\pgfpathlineto{\pgfqpoint{4.896251in}{3.482311in}}%
\pgfpathlineto{\pgfqpoint{4.900811in}{3.503995in}}%
\pgfpathlineto{\pgfqpoint{4.905370in}{3.428103in}}%
\pgfpathlineto{\pgfqpoint{4.909929in}{3.493153in}}%
\pgfpathlineto{\pgfqpoint{4.914488in}{3.525678in}}%
\pgfpathlineto{\pgfqpoint{4.919048in}{3.308845in}}%
\pgfpathlineto{\pgfqpoint{4.923607in}{3.579887in}}%
\pgfpathlineto{\pgfqpoint{4.932725in}{3.406420in}}%
\pgfpathlineto{\pgfqpoint{4.937285in}{3.363053in}}%
\pgfpathlineto{\pgfqpoint{4.941844in}{3.666620in}}%
\pgfpathlineto{\pgfqpoint{4.946403in}{3.471470in}}%
\pgfpathlineto{\pgfqpoint{4.950962in}{3.395578in}}%
\pgfpathlineto{\pgfqpoint{4.955522in}{3.525678in}}%
\pgfpathlineto{\pgfqpoint{4.960081in}{3.569045in}}%
\pgfpathlineto{\pgfqpoint{4.964640in}{3.569045in}}%
\pgfpathlineto{\pgfqpoint{4.969199in}{3.406420in}}%
\pgfpathlineto{\pgfqpoint{4.973759in}{3.428103in}}%
\pgfpathlineto{\pgfqpoint{4.978318in}{3.623253in}}%
\pgfpathlineto{\pgfqpoint{4.982877in}{3.384736in}}%
\pgfpathlineto{\pgfqpoint{4.987436in}{3.438945in}}%
\pgfpathlineto{\pgfqpoint{4.991996in}{3.363053in}}%
\pgfpathlineto{\pgfqpoint{4.996555in}{3.493153in}}%
\pgfpathlineto{\pgfqpoint{5.001114in}{3.471470in}}%
\pgfpathlineto{\pgfqpoint{5.005673in}{3.298003in}}%
\pgfpathlineto{\pgfqpoint{5.010233in}{3.547362in}}%
\pgfpathlineto{\pgfqpoint{5.014792in}{3.449786in}}%
\pgfpathlineto{\pgfqpoint{5.019351in}{3.731670in}}%
\pgfpathlineto{\pgfqpoint{5.023910in}{3.493153in}}%
\pgfpathlineto{\pgfqpoint{5.028470in}{3.482311in}}%
\pgfpathlineto{\pgfqpoint{5.033029in}{3.569045in}}%
\pgfpathlineto{\pgfqpoint{5.037588in}{3.319686in}}%
\pgfpathlineto{\pgfqpoint{5.042147in}{3.514836in}}%
\pgfpathlineto{\pgfqpoint{5.046706in}{3.590728in}}%
\pgfpathlineto{\pgfqpoint{5.051266in}{3.536520in}}%
\pgfpathlineto{\pgfqpoint{5.055825in}{3.634095in}}%
\pgfpathlineto{\pgfqpoint{5.060384in}{3.395578in}}%
\pgfpathlineto{\pgfqpoint{5.064943in}{3.525678in}}%
\pgfpathlineto{\pgfqpoint{5.069503in}{3.536520in}}%
\pgfpathlineto{\pgfqpoint{5.074062in}{3.449786in}}%
\pgfpathlineto{\pgfqpoint{5.078621in}{3.460628in}}%
\pgfpathlineto{\pgfqpoint{5.083180in}{3.547362in}}%
\pgfpathlineto{\pgfqpoint{5.087740in}{3.547362in}}%
\pgfpathlineto{\pgfqpoint{5.092299in}{3.558203in}}%
\pgfpathlineto{\pgfqpoint{5.096858in}{3.590728in}}%
\pgfpathlineto{\pgfqpoint{5.101417in}{3.547362in}}%
\pgfpathlineto{\pgfqpoint{5.105977in}{3.579887in}}%
\pgfpathlineto{\pgfqpoint{5.110536in}{3.558203in}}%
\pgfpathlineto{\pgfqpoint{5.115095in}{3.460628in}}%
\pgfpathlineto{\pgfqpoint{5.119654in}{3.547362in}}%
\pgfpathlineto{\pgfqpoint{5.124214in}{3.677462in}}%
\pgfpathlineto{\pgfqpoint{5.128773in}{3.449786in}}%
\pgfpathlineto{\pgfqpoint{5.133332in}{3.493153in}}%
\pgfpathlineto{\pgfqpoint{5.137891in}{3.525678in}}%
\pgfpathlineto{\pgfqpoint{5.142451in}{3.482311in}}%
\pgfpathlineto{\pgfqpoint{5.147010in}{3.612412in}}%
\pgfpathlineto{\pgfqpoint{5.151569in}{3.644937in}}%
\pgfpathlineto{\pgfqpoint{5.156128in}{3.558203in}}%
\pgfpathlineto{\pgfqpoint{5.160688in}{3.655778in}}%
\pgfpathlineto{\pgfqpoint{5.169806in}{3.395578in}}%
\pgfpathlineto{\pgfqpoint{5.174365in}{3.677462in}}%
\pgfpathlineto{\pgfqpoint{5.178925in}{3.536520in}}%
\pgfpathlineto{\pgfqpoint{5.183484in}{3.644937in}}%
\pgfpathlineto{\pgfqpoint{5.188043in}{3.601570in}}%
\pgfpathlineto{\pgfqpoint{5.192602in}{3.677462in}}%
\pgfpathlineto{\pgfqpoint{5.197162in}{3.731670in}}%
\pgfpathlineto{\pgfqpoint{5.201721in}{3.579887in}}%
\pgfpathlineto{\pgfqpoint{5.206280in}{3.601570in}}%
\pgfpathlineto{\pgfqpoint{5.210839in}{3.525678in}}%
\pgfpathlineto{\pgfqpoint{5.215398in}{3.525678in}}%
\pgfpathlineto{\pgfqpoint{5.219958in}{3.536520in}}%
\pgfpathlineto{\pgfqpoint{5.224517in}{3.579887in}}%
\pgfpathlineto{\pgfqpoint{5.229076in}{3.579887in}}%
\pgfpathlineto{\pgfqpoint{5.233635in}{3.536520in}}%
\pgfpathlineto{\pgfqpoint{5.238195in}{3.471470in}}%
\pgfpathlineto{\pgfqpoint{5.242754in}{3.558203in}}%
\pgfpathlineto{\pgfqpoint{5.247313in}{3.775037in}}%
\pgfpathlineto{\pgfqpoint{5.251872in}{3.590728in}}%
\pgfpathlineto{\pgfqpoint{5.256432in}{3.785878in}}%
\pgfpathlineto{\pgfqpoint{5.260991in}{3.699145in}}%
\pgfpathlineto{\pgfqpoint{5.265550in}{3.655778in}}%
\pgfpathlineto{\pgfqpoint{5.270109in}{3.785878in}}%
\pgfpathlineto{\pgfqpoint{5.274669in}{3.569045in}}%
\pgfpathlineto{\pgfqpoint{5.279228in}{3.558203in}}%
\pgfpathlineto{\pgfqpoint{5.283787in}{3.514836in}}%
\pgfpathlineto{\pgfqpoint{5.288346in}{3.579887in}}%
\pgfpathlineto{\pgfqpoint{5.292906in}{3.807562in}}%
\pgfpathlineto{\pgfqpoint{5.297465in}{3.764195in}}%
\pgfpathlineto{\pgfqpoint{5.302024in}{3.547362in}}%
\pgfpathlineto{\pgfqpoint{5.306583in}{3.612412in}}%
\pgfpathlineto{\pgfqpoint{5.311143in}{3.579887in}}%
\pgfpathlineto{\pgfqpoint{5.315702in}{3.601570in}}%
\pgfpathlineto{\pgfqpoint{5.320261in}{3.590728in}}%
\pgfpathlineto{\pgfqpoint{5.324820in}{3.590728in}}%
\pgfpathlineto{\pgfqpoint{5.329380in}{3.503995in}}%
\pgfpathlineto{\pgfqpoint{5.333939in}{3.634095in}}%
\pgfpathlineto{\pgfqpoint{5.338498in}{3.699145in}}%
\pgfpathlineto{\pgfqpoint{5.343057in}{3.503995in}}%
\pgfpathlineto{\pgfqpoint{5.347617in}{3.590728in}}%
\pgfpathlineto{\pgfqpoint{5.352176in}{3.623253in}}%
\pgfpathlineto{\pgfqpoint{5.356735in}{3.482311in}}%
\pgfpathlineto{\pgfqpoint{5.361294in}{3.406420in}}%
\pgfpathlineto{\pgfqpoint{5.365853in}{3.655778in}}%
\pgfpathlineto{\pgfqpoint{5.370413in}{3.579887in}}%
\pgfpathlineto{\pgfqpoint{5.374972in}{3.731670in}}%
\pgfpathlineto{\pgfqpoint{5.379531in}{3.709987in}}%
\pgfpathlineto{\pgfqpoint{5.388650in}{3.547362in}}%
\pgfpathlineto{\pgfqpoint{5.393209in}{3.634095in}}%
\pgfpathlineto{\pgfqpoint{5.397768in}{3.503995in}}%
\pgfpathlineto{\pgfqpoint{5.402327in}{3.623253in}}%
\pgfpathlineto{\pgfqpoint{5.406887in}{3.623253in}}%
\pgfpathlineto{\pgfqpoint{5.411446in}{3.655778in}}%
\pgfpathlineto{\pgfqpoint{5.416005in}{3.579887in}}%
\pgfpathlineto{\pgfqpoint{5.420564in}{3.709987in}}%
\pgfpathlineto{\pgfqpoint{5.425124in}{3.601570in}}%
\pgfpathlineto{\pgfqpoint{5.429683in}{3.612412in}}%
\pgfpathlineto{\pgfqpoint{5.434242in}{3.666620in}}%
\pgfpathlineto{\pgfqpoint{5.438801in}{3.536520in}}%
\pgfpathlineto{\pgfqpoint{5.443361in}{3.829245in}}%
\pgfpathlineto{\pgfqpoint{5.447920in}{3.601570in}}%
\pgfpathlineto{\pgfqpoint{5.452479in}{3.514836in}}%
\pgfpathlineto{\pgfqpoint{5.457038in}{3.764195in}}%
\pgfpathlineto{\pgfqpoint{5.461598in}{3.775037in}}%
\pgfpathlineto{\pgfqpoint{5.466157in}{3.644937in}}%
\pgfpathlineto{\pgfqpoint{5.470716in}{3.677462in}}%
\pgfpathlineto{\pgfqpoint{5.475275in}{3.720828in}}%
\pgfpathlineto{\pgfqpoint{5.479835in}{3.547362in}}%
\pgfpathlineto{\pgfqpoint{5.484394in}{3.742512in}}%
\pgfpathlineto{\pgfqpoint{5.488953in}{3.612412in}}%
\pgfpathlineto{\pgfqpoint{5.493512in}{3.785878in}}%
\pgfpathlineto{\pgfqpoint{5.498072in}{3.655778in}}%
\pgfpathlineto{\pgfqpoint{5.502631in}{3.699145in}}%
\pgfpathlineto{\pgfqpoint{5.507190in}{3.634095in}}%
\pgfpathlineto{\pgfqpoint{5.511749in}{3.655778in}}%
\pgfpathlineto{\pgfqpoint{5.516308in}{3.438945in}}%
\pgfpathlineto{\pgfqpoint{5.520868in}{3.482311in}}%
\pgfpathlineto{\pgfqpoint{5.525427in}{3.753353in}}%
\pgfpathlineto{\pgfqpoint{5.529986in}{3.666620in}}%
\pgfpathlineto{\pgfqpoint{5.534545in}{3.655778in}}%
\pgfpathlineto{\pgfqpoint{5.534545in}{3.655778in}}%
\pgfusepath{stroke}%
\end{pgfscope}%
\begin{pgfscope}%
\pgfpathrectangle{\pgfqpoint{0.800000in}{0.528000in}}{\pgfqpoint{4.960000in}{3.696000in}} %
\pgfusepath{clip}%
\pgfsetrectcap%
\pgfsetroundjoin%
\pgfsetlinewidth{1.505625pt}%
\definecolor{currentstroke}{rgb}{1.000000,0.498039,0.054902}%
\pgfsetstrokecolor{currentstroke}%
\pgfsetdash{}{0pt}%
\pgfpathmoveto{\pgfqpoint{1.025455in}{0.812959in}}%
\pgfpathlineto{\pgfqpoint{1.039132in}{0.884091in}}%
\pgfpathlineto{\pgfqpoint{1.052810in}{0.944965in}}%
\pgfpathlineto{\pgfqpoint{1.071047in}{1.015256in}}%
\pgfpathlineto{\pgfqpoint{1.089284in}{1.076699in}}%
\pgfpathlineto{\pgfqpoint{1.112080in}{1.144550in}}%
\pgfpathlineto{\pgfqpoint{1.134876in}{1.204986in}}%
\pgfpathlineto{\pgfqpoint{1.157673in}{1.259752in}}%
\pgfpathlineto{\pgfqpoint{1.185028in}{1.319606in}}%
\pgfpathlineto{\pgfqpoint{1.212383in}{1.374349in}}%
\pgfpathlineto{\pgfqpoint{1.244298in}{1.433034in}}%
\pgfpathlineto{\pgfqpoint{1.276213in}{1.487168in}}%
\pgfpathlineto{\pgfqpoint{1.312687in}{1.544468in}}%
\pgfpathlineto{\pgfqpoint{1.349161in}{1.597720in}}%
\pgfpathlineto{\pgfqpoint{1.390194in}{1.653591in}}%
\pgfpathlineto{\pgfqpoint{1.431227in}{1.705850in}}%
\pgfpathlineto{\pgfqpoint{1.476820in}{1.760324in}}%
\pgfpathlineto{\pgfqpoint{1.526971in}{1.816524in}}%
\pgfpathlineto{\pgfqpoint{1.577123in}{1.869386in}}%
\pgfpathlineto{\pgfqpoint{1.631834in}{1.923780in}}%
\pgfpathlineto{\pgfqpoint{1.691104in}{1.979374in}}%
\pgfpathlineto{\pgfqpoint{1.750374in}{2.031959in}}%
\pgfpathlineto{\pgfqpoint{1.814204in}{2.085655in}}%
\pgfpathlineto{\pgfqpoint{1.882592in}{2.140228in}}%
\pgfpathlineto{\pgfqpoint{1.955540in}{2.195481in}}%
\pgfpathlineto{\pgfqpoint{2.033047in}{2.251248in}}%
\pgfpathlineto{\pgfqpoint{2.115114in}{2.307388in}}%
\pgfpathlineto{\pgfqpoint{2.201739in}{2.363784in}}%
\pgfpathlineto{\pgfqpoint{2.292924in}{2.420338in}}%
\pgfpathlineto{\pgfqpoint{2.388668in}{2.476968in}}%
\pgfpathlineto{\pgfqpoint{2.488971in}{2.533607in}}%
\pgfpathlineto{\pgfqpoint{2.593834in}{2.590196in}}%
\pgfpathlineto{\pgfqpoint{2.703256in}{2.646691in}}%
\pgfpathlineto{\pgfqpoint{2.817237in}{2.703050in}}%
\pgfpathlineto{\pgfqpoint{2.940336in}{2.761359in}}%
\pgfpathlineto{\pgfqpoint{3.067995in}{2.819303in}}%
\pgfpathlineto{\pgfqpoint{3.200213in}{2.876878in}}%
\pgfpathlineto{\pgfqpoint{3.341550in}{2.935949in}}%
\pgfpathlineto{\pgfqpoint{3.487446in}{2.994502in}}%
\pgfpathlineto{\pgfqpoint{3.642460in}{3.054278in}}%
\pgfpathlineto{\pgfqpoint{3.802033in}{3.113435in}}%
\pgfpathlineto{\pgfqpoint{3.970725in}{3.173595in}}%
\pgfpathlineto{\pgfqpoint{4.148536in}{3.234606in}}%
\pgfpathlineto{\pgfqpoint{4.335465in}{3.296332in}}%
\pgfpathlineto{\pgfqpoint{4.526953in}{3.357231in}}%
\pgfpathlineto{\pgfqpoint{4.727560in}{3.418722in}}%
\pgfpathlineto{\pgfqpoint{4.937285in}{3.480705in}}%
\pgfpathlineto{\pgfqpoint{5.156128in}{3.543091in}}%
\pgfpathlineto{\pgfqpoint{5.388650in}{3.607031in}}%
\pgfpathlineto{\pgfqpoint{5.534545in}{3.646010in}}%
\pgfpathlineto{\pgfqpoint{5.534545in}{3.646010in}}%
\pgfusepath{stroke}%
\end{pgfscope}%
\begin{pgfscope}%
\pgfpathrectangle{\pgfqpoint{0.800000in}{0.528000in}}{\pgfqpoint{4.960000in}{3.696000in}} %
\pgfusepath{clip}%
\pgfsetbuttcap%
\pgfsetroundjoin%
\pgfsetlinewidth{1.505625pt}%
\definecolor{currentstroke}{rgb}{0.172549,0.627451,0.172549}%
\pgfsetstrokecolor{currentstroke}%
\pgfsetdash{{5.600000pt}{2.400000pt}}{0.000000pt}%
\pgfpathmoveto{\pgfqpoint{1.025455in}{0.696000in}}%
\pgfpathlineto{\pgfqpoint{1.039132in}{0.756442in}}%
\pgfpathlineto{\pgfqpoint{1.052810in}{0.808168in}}%
\pgfpathlineto{\pgfqpoint{1.071047in}{0.867897in}}%
\pgfpathlineto{\pgfqpoint{1.089284in}{0.920106in}}%
\pgfpathlineto{\pgfqpoint{1.112080in}{0.977760in}}%
\pgfpathlineto{\pgfqpoint{1.134876in}{1.029115in}}%
\pgfpathlineto{\pgfqpoint{1.162232in}{1.084473in}}%
\pgfpathlineto{\pgfqpoint{1.189587in}{1.134538in}}%
\pgfpathlineto{\pgfqpoint{1.221502in}{1.187717in}}%
\pgfpathlineto{\pgfqpoint{1.253417in}{1.236397in}}%
\pgfpathlineto{\pgfqpoint{1.289891in}{1.287595in}}%
\pgfpathlineto{\pgfqpoint{1.326365in}{1.334920in}}%
\pgfpathlineto{\pgfqpoint{1.367398in}{1.384343in}}%
\pgfpathlineto{\pgfqpoint{1.408431in}{1.430391in}}%
\pgfpathlineto{\pgfqpoint{1.454023in}{1.478226in}}%
\pgfpathlineto{\pgfqpoint{1.504175in}{1.527426in}}%
\pgfpathlineto{\pgfqpoint{1.554327in}{1.573579in}}%
\pgfpathlineto{\pgfqpoint{1.609038in}{1.620960in}}%
\pgfpathlineto{\pgfqpoint{1.668308in}{1.669285in}}%
\pgfpathlineto{\pgfqpoint{1.732137in}{1.718317in}}%
\pgfpathlineto{\pgfqpoint{1.800526in}{1.767865in}}%
\pgfpathlineto{\pgfqpoint{1.873474in}{1.817769in}}%
\pgfpathlineto{\pgfqpoint{1.946422in}{1.865025in}}%
\pgfpathlineto{\pgfqpoint{2.023929in}{1.912698in}}%
\pgfpathlineto{\pgfqpoint{2.105995in}{1.960668in}}%
\pgfpathlineto{\pgfqpoint{2.197180in}{2.011309in}}%
\pgfpathlineto{\pgfqpoint{2.292924in}{2.061829in}}%
\pgfpathlineto{\pgfqpoint{2.393227in}{2.112186in}}%
\pgfpathlineto{\pgfqpoint{2.498090in}{2.162345in}}%
\pgfpathlineto{\pgfqpoint{2.607512in}{2.212279in}}%
\pgfpathlineto{\pgfqpoint{2.726052in}{2.263910in}}%
\pgfpathlineto{\pgfqpoint{2.849152in}{2.315110in}}%
\pgfpathlineto{\pgfqpoint{2.981370in}{2.367657in}}%
\pgfpathlineto{\pgfqpoint{3.118147in}{2.419633in}}%
\pgfpathlineto{\pgfqpoint{3.264043in}{2.472687in}}%
\pgfpathlineto{\pgfqpoint{3.419057in}{2.526636in}}%
\pgfpathlineto{\pgfqpoint{3.578630in}{2.579835in}}%
\pgfpathlineto{\pgfqpoint{3.747322in}{2.633755in}}%
\pgfpathlineto{\pgfqpoint{3.925133in}{2.688268in}}%
\pgfpathlineto{\pgfqpoint{4.112062in}{2.743260in}}%
\pgfpathlineto{\pgfqpoint{4.308109in}{2.798632in}}%
\pgfpathlineto{\pgfqpoint{4.517834in}{2.855509in}}%
\pgfpathlineto{\pgfqpoint{4.736678in}{2.912512in}}%
\pgfpathlineto{\pgfqpoint{4.964640in}{2.969582in}}%
\pgfpathlineto{\pgfqpoint{5.206280in}{3.027746in}}%
\pgfpathlineto{\pgfqpoint{5.457038in}{3.085807in}}%
\pgfpathlineto{\pgfqpoint{5.534545in}{3.103314in}}%
\pgfpathlineto{\pgfqpoint{5.534545in}{3.103314in}}%
\pgfusepath{stroke}%
\end{pgfscope}%
\begin{pgfscope}%
\pgfpathrectangle{\pgfqpoint{0.800000in}{0.528000in}}{\pgfqpoint{4.960000in}{3.696000in}} %
\pgfusepath{clip}%
\pgfsetbuttcap%
\pgfsetroundjoin%
\pgfsetlinewidth{1.505625pt}%
\definecolor{currentstroke}{rgb}{0.839216,0.152941,0.156863}%
\pgfsetstrokecolor{currentstroke}%
\pgfsetdash{{5.600000pt}{2.400000pt}}{0.000000pt}%
\pgfpathmoveto{\pgfqpoint{1.025455in}{0.901319in}}%
\pgfpathlineto{\pgfqpoint{1.039132in}{0.980525in}}%
\pgfpathlineto{\pgfqpoint{1.052810in}{1.048310in}}%
\pgfpathlineto{\pgfqpoint{1.071047in}{1.126582in}}%
\pgfpathlineto{\pgfqpoint{1.089284in}{1.195000in}}%
\pgfpathlineto{\pgfqpoint{1.112080in}{1.270553in}}%
\pgfpathlineto{\pgfqpoint{1.134876in}{1.337851in}}%
\pgfpathlineto{\pgfqpoint{1.157673in}{1.398835in}}%
\pgfpathlineto{\pgfqpoint{1.185028in}{1.465483in}}%
\pgfpathlineto{\pgfqpoint{1.212383in}{1.526442in}}%
\pgfpathlineto{\pgfqpoint{1.244298in}{1.591789in}}%
\pgfpathlineto{\pgfqpoint{1.276213in}{1.652068in}}%
\pgfpathlineto{\pgfqpoint{1.312687in}{1.715874in}}%
\pgfpathlineto{\pgfqpoint{1.349161in}{1.775172in}}%
\pgfpathlineto{\pgfqpoint{1.390194in}{1.837385in}}%
\pgfpathlineto{\pgfqpoint{1.431227in}{1.895577in}}%
\pgfpathlineto{\pgfqpoint{1.476820in}{1.956236in}}%
\pgfpathlineto{\pgfqpoint{1.522412in}{2.013289in}}%
\pgfpathlineto{\pgfqpoint{1.572564in}{2.072468in}}%
\pgfpathlineto{\pgfqpoint{1.627275in}{2.133333in}}%
\pgfpathlineto{\pgfqpoint{1.681985in}{2.190860in}}%
\pgfpathlineto{\pgfqpoint{1.741256in}{2.249901in}}%
\pgfpathlineto{\pgfqpoint{1.805085in}{2.310150in}}%
\pgfpathlineto{\pgfqpoint{1.868914in}{2.367368in}}%
\pgfpathlineto{\pgfqpoint{1.937303in}{2.425709in}}%
\pgfpathlineto{\pgfqpoint{2.010251in}{2.484954in}}%
\pgfpathlineto{\pgfqpoint{2.087758in}{2.544913in}}%
\pgfpathlineto{\pgfqpoint{2.169824in}{2.605425in}}%
\pgfpathlineto{\pgfqpoint{2.251891in}{2.663214in}}%
\pgfpathlineto{\pgfqpoint{2.338516in}{2.721577in}}%
\pgfpathlineto{\pgfqpoint{2.429701in}{2.780390in}}%
\pgfpathlineto{\pgfqpoint{2.525445in}{2.839547in}}%
\pgfpathlineto{\pgfqpoint{2.625749in}{2.898956in}}%
\pgfpathlineto{\pgfqpoint{2.730611in}{2.958538in}}%
\pgfpathlineto{\pgfqpoint{2.840033in}{3.018226in}}%
\pgfpathlineto{\pgfqpoint{2.958573in}{3.080302in}}%
\pgfpathlineto{\pgfqpoint{3.081673in}{3.142194in}}%
\pgfpathlineto{\pgfqpoint{3.209332in}{3.203876in}}%
\pgfpathlineto{\pgfqpoint{3.341550in}{3.265327in}}%
\pgfpathlineto{\pgfqpoint{3.482886in}{3.328529in}}%
\pgfpathlineto{\pgfqpoint{3.628782in}{3.391322in}}%
\pgfpathlineto{\pgfqpoint{3.783796in}{3.455562in}}%
\pgfpathlineto{\pgfqpoint{3.943370in}{3.519264in}}%
\pgfpathlineto{\pgfqpoint{4.112062in}{3.584165in}}%
\pgfpathlineto{\pgfqpoint{4.285313in}{3.648439in}}%
\pgfpathlineto{\pgfqpoint{4.467683in}{3.713713in}}%
\pgfpathlineto{\pgfqpoint{4.659171in}{3.779847in}}%
\pgfpathlineto{\pgfqpoint{4.859778in}{3.846712in}}%
\pgfpathlineto{\pgfqpoint{5.064943in}{3.912754in}}%
\pgfpathlineto{\pgfqpoint{5.279228in}{3.979410in}}%
\pgfpathlineto{\pgfqpoint{5.502631in}{4.046585in}}%
\pgfpathlineto{\pgfqpoint{5.534545in}{4.056000in}}%
\pgfpathlineto{\pgfqpoint{5.534545in}{4.056000in}}%
\pgfusepath{stroke}%
\end{pgfscope}%
\begin{pgfscope}%
\pgfsetrectcap%
\pgfsetmiterjoin%
\pgfsetlinewidth{0.803000pt}%
\definecolor{currentstroke}{rgb}{0.000000,0.000000,0.000000}%
\pgfsetstrokecolor{currentstroke}%
\pgfsetdash{}{0pt}%
\pgfpathmoveto{\pgfqpoint{0.800000in}{0.528000in}}%
\pgfpathlineto{\pgfqpoint{0.800000in}{4.224000in}}%
\pgfusepath{stroke}%
\end{pgfscope}%
\begin{pgfscope}%
\pgfsetrectcap%
\pgfsetmiterjoin%
\pgfsetlinewidth{0.803000pt}%
\definecolor{currentstroke}{rgb}{0.000000,0.000000,0.000000}%
\pgfsetstrokecolor{currentstroke}%
\pgfsetdash{}{0pt}%
\pgfpathmoveto{\pgfqpoint{5.760000in}{0.528000in}}%
\pgfpathlineto{\pgfqpoint{5.760000in}{4.224000in}}%
\pgfusepath{stroke}%
\end{pgfscope}%
\begin{pgfscope}%
\pgfsetrectcap%
\pgfsetmiterjoin%
\pgfsetlinewidth{0.803000pt}%
\definecolor{currentstroke}{rgb}{0.000000,0.000000,0.000000}%
\pgfsetstrokecolor{currentstroke}%
\pgfsetdash{}{0pt}%
\pgfpathmoveto{\pgfqpoint{0.800000in}{0.528000in}}%
\pgfpathlineto{\pgfqpoint{5.760000in}{0.528000in}}%
\pgfusepath{stroke}%
\end{pgfscope}%
\begin{pgfscope}%
\pgfsetrectcap%
\pgfsetmiterjoin%
\pgfsetlinewidth{0.803000pt}%
\definecolor{currentstroke}{rgb}{0.000000,0.000000,0.000000}%
\pgfsetstrokecolor{currentstroke}%
\pgfsetdash{}{0pt}%
\pgfpathmoveto{\pgfqpoint{0.800000in}{4.224000in}}%
\pgfpathlineto{\pgfqpoint{5.760000in}{4.224000in}}%
\pgfusepath{stroke}%
\end{pgfscope}%
\begin{pgfscope}%
\pgfsetbuttcap%
\pgfsetmiterjoin%
\definecolor{currentfill}{rgb}{1.000000,1.000000,1.000000}%
\pgfsetfillcolor{currentfill}%
\pgfsetfillopacity{0.800000}%
\pgfsetlinewidth{1.003750pt}%
\definecolor{currentstroke}{rgb}{0.800000,0.800000,0.800000}%
\pgfsetstrokecolor{currentstroke}%
\pgfsetstrokeopacity{0.800000}%
\pgfsetdash{}{0pt}%
\pgfpathmoveto{\pgfqpoint{0.897222in}{3.297460in}}%
\pgfpathlineto{\pgfqpoint{3.351215in}{3.297460in}}%
\pgfpathquadraticcurveto{\pgfqpoint{3.378993in}{3.297460in}}{\pgfqpoint{3.378993in}{3.325238in}}%
\pgfpathlineto{\pgfqpoint{3.378993in}{4.126778in}}%
\pgfpathquadraticcurveto{\pgfqpoint{3.378993in}{4.154556in}}{\pgfqpoint{3.351215in}{4.154556in}}%
\pgfpathlineto{\pgfqpoint{0.897222in}{4.154556in}}%
\pgfpathquadraticcurveto{\pgfqpoint{0.869444in}{4.154556in}}{\pgfqpoint{0.869444in}{4.126778in}}%
\pgfpathlineto{\pgfqpoint{0.869444in}{3.325238in}}%
\pgfpathquadraticcurveto{\pgfqpoint{0.869444in}{3.297460in}}{\pgfqpoint{0.897222in}{3.297460in}}%
\pgfpathclose%
\pgfusepath{stroke,fill}%
\end{pgfscope}%
\begin{pgfscope}%
\pgfsetrectcap%
\pgfsetroundjoin%
\pgfsetlinewidth{1.505625pt}%
\definecolor{currentstroke}{rgb}{0.121569,0.466667,0.705882}%
\pgfsetstrokecolor{currentstroke}%
\pgfsetdash{}{0pt}%
\pgfpathmoveto{\pgfqpoint{0.925000in}{4.042088in}}%
\pgfpathlineto{\pgfqpoint{1.202778in}{4.042088in}}%
\pgfusepath{stroke}%
\end{pgfscope}%
\begin{pgfscope}%
\pgftext[x=1.313889in,y=3.993477in,left,base]{\sffamily\fontsize{10.000000}{12.000000}\selectfont Data}%
\end{pgfscope}%
\begin{pgfscope}%
\pgfsetrectcap%
\pgfsetroundjoin%
\pgfsetlinewidth{1.505625pt}%
\definecolor{currentstroke}{rgb}{1.000000,0.498039,0.054902}%
\pgfsetstrokecolor{currentstroke}%
\pgfsetdash{}{0pt}%
\pgfpathmoveto{\pgfqpoint{0.925000in}{3.838231in}}%
\pgfpathlineto{\pgfqpoint{1.202778in}{3.838231in}}%
\pgfusepath{stroke}%
\end{pgfscope}%
\begin{pgfscope}%
\pgftext[x=1.313889in,y=3.789620in,left,base]{\sffamily\fontsize{10.000000}{12.000000}\selectfont Least squares approximation}%
\end{pgfscope}%
\begin{pgfscope}%
\pgfsetbuttcap%
\pgfsetroundjoin%
\pgfsetlinewidth{1.505625pt}%
\definecolor{currentstroke}{rgb}{0.172549,0.627451,0.172549}%
\pgfsetstrokecolor{currentstroke}%
\pgfsetdash{{5.600000pt}{2.400000pt}}{0.000000pt}%
\pgfpathmoveto{\pgfqpoint{0.925000in}{3.634374in}}%
\pgfpathlineto{\pgfqpoint{1.202778in}{3.634374in}}%
\pgfusepath{stroke}%
\end{pgfscope}%
\begin{pgfscope}%
\pgftext[x=1.313889in,y=3.585762in,left,base]{\sffamily\fontsize{10.000000}{12.000000}\selectfont cmin}%
\end{pgfscope}%
\begin{pgfscope}%
\pgfsetbuttcap%
\pgfsetroundjoin%
\pgfsetlinewidth{1.505625pt}%
\definecolor{currentstroke}{rgb}{0.839216,0.152941,0.156863}%
\pgfsetstrokecolor{currentstroke}%
\pgfsetdash{{5.600000pt}{2.400000pt}}{0.000000pt}%
\pgfpathmoveto{\pgfqpoint{0.925000in}{3.430516in}}%
\pgfpathlineto{\pgfqpoint{1.202778in}{3.430516in}}%
\pgfusepath{stroke}%
\end{pgfscope}%
\begin{pgfscope}%
\pgftext[x=1.313889in,y=3.381905in,left,base]{\sffamily\fontsize{10.000000}{12.000000}\selectfont cmax}%
\end{pgfscope}%
\end{pgfpicture}%
\makeatother%
\endgroup%

			\caption{Données expérimentales}
			\label{fig:exp_cercle}
		\end{figure}

		On peut alors se demander, comme le sujet le suggère, comment varie cette loi quand on tire ces points non plus dans un disque, mais dans un polygone régulier.
		Pour effectuer cette expérience, on garde l'algorithme de graham, mais on modifie le générateur de nombre aléatoires. Pour pouvoir tirer des points uniformément dans un polygone, on procède de la manière suivante : On tire aléatoirement une "tranche" $i$, c'est à dire un nombre entre $0$ et $k-1$ ou $k$ représente le nombre de sommets du polygone régulier choisi, puis on génère un nombre qui se trouvera dans le triangle composé de l'origine, du point d'affixe $e^{2ij\pi/k}$ et du point d'affixe $e^{2i(j+1)\pi/k}$
		Pour générer un point dans ce triangle, on génère un point dans le parallélogramme associé à ce triangle, puis on prend calcule ses coordonnées dans le triangle dans lequel ce point se trouve. 
		On trace alors le graphe Figure~\ref{fig:exp_poly}, et calculer la constante qui va miniser la somme des résidus quadratiques. (L'exemple utilise un pentagone)

		\begin{figure}[htpb]
			\centering
			%% Creator: Matplotlib, PGF backend
%%
%% To include the figure in your LaTeX document, write
%%   \input{<filename>.pgf}
%%
%% Make sure the required packages are loaded in your preamble
%%   \usepackage{pgf}
%%
%% Figures using additional raster images can only be included by \input if
%% they are in the same directory as the main LaTeX file. For loading figures
%% from other directories you can use the `import` package
%%   \usepackage{import}
%% and then include the figures with
%%   \import{<path to file>}{<filename>.pgf}
%%
%% Matplotlib used the following preamble
%%   \usepackage{fontspec}
%%   \setmainfont{DejaVu Serif}
%%   \setsansfont{DejaVu Sans}
%%   \setmonofont{DejaVu Sans Mono}
%%
\begingroup%
\makeatletter%
\begin{pgfpicture}%
\pgfpathrectangle{\pgfpointorigin}{\pgfqpoint{6.400000in}{4.800000in}}%
\pgfusepath{use as bounding box, clip}%
\begin{pgfscope}%
\pgfsetbuttcap%
\pgfsetmiterjoin%
\definecolor{currentfill}{rgb}{1.000000,1.000000,1.000000}%
\pgfsetfillcolor{currentfill}%
\pgfsetlinewidth{0.000000pt}%
\definecolor{currentstroke}{rgb}{1.000000,1.000000,1.000000}%
\pgfsetstrokecolor{currentstroke}%
\pgfsetdash{}{0pt}%
\pgfpathmoveto{\pgfqpoint{0.000000in}{0.000000in}}%
\pgfpathlineto{\pgfqpoint{6.400000in}{0.000000in}}%
\pgfpathlineto{\pgfqpoint{6.400000in}{4.800000in}}%
\pgfpathlineto{\pgfqpoint{0.000000in}{4.800000in}}%
\pgfpathclose%
\pgfusepath{fill}%
\end{pgfscope}%
\begin{pgfscope}%
\pgfsetbuttcap%
\pgfsetmiterjoin%
\definecolor{currentfill}{rgb}{1.000000,1.000000,1.000000}%
\pgfsetfillcolor{currentfill}%
\pgfsetlinewidth{0.000000pt}%
\definecolor{currentstroke}{rgb}{0.000000,0.000000,0.000000}%
\pgfsetstrokecolor{currentstroke}%
\pgfsetstrokeopacity{0.000000}%
\pgfsetdash{}{0pt}%
\pgfpathmoveto{\pgfqpoint{0.800000in}{0.528000in}}%
\pgfpathlineto{\pgfqpoint{5.760000in}{0.528000in}}%
\pgfpathlineto{\pgfqpoint{5.760000in}{4.224000in}}%
\pgfpathlineto{\pgfqpoint{0.800000in}{4.224000in}}%
\pgfpathclose%
\pgfusepath{fill}%
\end{pgfscope}%
\begin{pgfscope}%
\pgfsetbuttcap%
\pgfsetroundjoin%
\definecolor{currentfill}{rgb}{0.000000,0.000000,0.000000}%
\pgfsetfillcolor{currentfill}%
\pgfsetlinewidth{0.803000pt}%
\definecolor{currentstroke}{rgb}{0.000000,0.000000,0.000000}%
\pgfsetstrokecolor{currentstroke}%
\pgfsetdash{}{0pt}%
\pgfsys@defobject{currentmarker}{\pgfqpoint{0.000000in}{-0.048611in}}{\pgfqpoint{0.000000in}{0.000000in}}{%
\pgfpathmoveto{\pgfqpoint{0.000000in}{0.000000in}}%
\pgfpathlineto{\pgfqpoint{0.000000in}{-0.048611in}}%
\pgfusepath{stroke,fill}%
}%
\begin{pgfscope}%
\pgfsys@transformshift{0.979862in}{0.528000in}%
\pgfsys@useobject{currentmarker}{}%
\end{pgfscope}%
\end{pgfscope}%
\begin{pgfscope}%
\pgftext[x=0.979862in,y=0.430778in,,top]{\sffamily\fontsize{10.000000}{12.000000}\selectfont 0}%
\end{pgfscope}%
\begin{pgfscope}%
\pgfsetbuttcap%
\pgfsetroundjoin%
\definecolor{currentfill}{rgb}{0.000000,0.000000,0.000000}%
\pgfsetfillcolor{currentfill}%
\pgfsetlinewidth{0.803000pt}%
\definecolor{currentstroke}{rgb}{0.000000,0.000000,0.000000}%
\pgfsetstrokecolor{currentstroke}%
\pgfsetdash{}{0pt}%
\pgfsys@defobject{currentmarker}{\pgfqpoint{0.000000in}{-0.048611in}}{\pgfqpoint{0.000000in}{0.000000in}}{%
\pgfpathmoveto{\pgfqpoint{0.000000in}{0.000000in}}%
\pgfpathlineto{\pgfqpoint{0.000000in}{-0.048611in}}%
\pgfusepath{stroke,fill}%
}%
\begin{pgfscope}%
\pgfsys@transformshift{1.891711in}{0.528000in}%
\pgfsys@useobject{currentmarker}{}%
\end{pgfscope}%
\end{pgfscope}%
\begin{pgfscope}%
\pgftext[x=1.891711in,y=0.430778in,,top]{\sffamily\fontsize{10.000000}{12.000000}\selectfont 200}%
\end{pgfscope}%
\begin{pgfscope}%
\pgfsetbuttcap%
\pgfsetroundjoin%
\definecolor{currentfill}{rgb}{0.000000,0.000000,0.000000}%
\pgfsetfillcolor{currentfill}%
\pgfsetlinewidth{0.803000pt}%
\definecolor{currentstroke}{rgb}{0.000000,0.000000,0.000000}%
\pgfsetstrokecolor{currentstroke}%
\pgfsetdash{}{0pt}%
\pgfsys@defobject{currentmarker}{\pgfqpoint{0.000000in}{-0.048611in}}{\pgfqpoint{0.000000in}{0.000000in}}{%
\pgfpathmoveto{\pgfqpoint{0.000000in}{0.000000in}}%
\pgfpathlineto{\pgfqpoint{0.000000in}{-0.048611in}}%
\pgfusepath{stroke,fill}%
}%
\begin{pgfscope}%
\pgfsys@transformshift{2.803559in}{0.528000in}%
\pgfsys@useobject{currentmarker}{}%
\end{pgfscope}%
\end{pgfscope}%
\begin{pgfscope}%
\pgftext[x=2.803559in,y=0.430778in,,top]{\sffamily\fontsize{10.000000}{12.000000}\selectfont 400}%
\end{pgfscope}%
\begin{pgfscope}%
\pgfsetbuttcap%
\pgfsetroundjoin%
\definecolor{currentfill}{rgb}{0.000000,0.000000,0.000000}%
\pgfsetfillcolor{currentfill}%
\pgfsetlinewidth{0.803000pt}%
\definecolor{currentstroke}{rgb}{0.000000,0.000000,0.000000}%
\pgfsetstrokecolor{currentstroke}%
\pgfsetdash{}{0pt}%
\pgfsys@defobject{currentmarker}{\pgfqpoint{0.000000in}{-0.048611in}}{\pgfqpoint{0.000000in}{0.000000in}}{%
\pgfpathmoveto{\pgfqpoint{0.000000in}{0.000000in}}%
\pgfpathlineto{\pgfqpoint{0.000000in}{-0.048611in}}%
\pgfusepath{stroke,fill}%
}%
\begin{pgfscope}%
\pgfsys@transformshift{3.715408in}{0.528000in}%
\pgfsys@useobject{currentmarker}{}%
\end{pgfscope}%
\end{pgfscope}%
\begin{pgfscope}%
\pgftext[x=3.715408in,y=0.430778in,,top]{\sffamily\fontsize{10.000000}{12.000000}\selectfont 600}%
\end{pgfscope}%
\begin{pgfscope}%
\pgfsetbuttcap%
\pgfsetroundjoin%
\definecolor{currentfill}{rgb}{0.000000,0.000000,0.000000}%
\pgfsetfillcolor{currentfill}%
\pgfsetlinewidth{0.803000pt}%
\definecolor{currentstroke}{rgb}{0.000000,0.000000,0.000000}%
\pgfsetstrokecolor{currentstroke}%
\pgfsetdash{}{0pt}%
\pgfsys@defobject{currentmarker}{\pgfqpoint{0.000000in}{-0.048611in}}{\pgfqpoint{0.000000in}{0.000000in}}{%
\pgfpathmoveto{\pgfqpoint{0.000000in}{0.000000in}}%
\pgfpathlineto{\pgfqpoint{0.000000in}{-0.048611in}}%
\pgfusepath{stroke,fill}%
}%
\begin{pgfscope}%
\pgfsys@transformshift{4.627256in}{0.528000in}%
\pgfsys@useobject{currentmarker}{}%
\end{pgfscope}%
\end{pgfscope}%
\begin{pgfscope}%
\pgftext[x=4.627256in,y=0.430778in,,top]{\sffamily\fontsize{10.000000}{12.000000}\selectfont 800}%
\end{pgfscope}%
\begin{pgfscope}%
\pgfsetbuttcap%
\pgfsetroundjoin%
\definecolor{currentfill}{rgb}{0.000000,0.000000,0.000000}%
\pgfsetfillcolor{currentfill}%
\pgfsetlinewidth{0.803000pt}%
\definecolor{currentstroke}{rgb}{0.000000,0.000000,0.000000}%
\pgfsetstrokecolor{currentstroke}%
\pgfsetdash{}{0pt}%
\pgfsys@defobject{currentmarker}{\pgfqpoint{0.000000in}{-0.048611in}}{\pgfqpoint{0.000000in}{0.000000in}}{%
\pgfpathmoveto{\pgfqpoint{0.000000in}{0.000000in}}%
\pgfpathlineto{\pgfqpoint{0.000000in}{-0.048611in}}%
\pgfusepath{stroke,fill}%
}%
\begin{pgfscope}%
\pgfsys@transformshift{5.539105in}{0.528000in}%
\pgfsys@useobject{currentmarker}{}%
\end{pgfscope}%
\end{pgfscope}%
\begin{pgfscope}%
\pgftext[x=5.539105in,y=0.430778in,,top]{\sffamily\fontsize{10.000000}{12.000000}\selectfont 1000}%
\end{pgfscope}%
\begin{pgfscope}%
\pgftext[x=3.280000in,y=0.240809in,,top]{\sffamily\fontsize{10.000000}{12.000000}\selectfont \(\displaystyle n\)}%
\end{pgfscope}%
\begin{pgfscope}%
\pgfsetbuttcap%
\pgfsetroundjoin%
\definecolor{currentfill}{rgb}{0.000000,0.000000,0.000000}%
\pgfsetfillcolor{currentfill}%
\pgfsetlinewidth{0.803000pt}%
\definecolor{currentstroke}{rgb}{0.000000,0.000000,0.000000}%
\pgfsetstrokecolor{currentstroke}%
\pgfsetdash{}{0pt}%
\pgfsys@defobject{currentmarker}{\pgfqpoint{-0.048611in}{0.000000in}}{\pgfqpoint{0.000000in}{0.000000in}}{%
\pgfpathmoveto{\pgfqpoint{0.000000in}{0.000000in}}%
\pgfpathlineto{\pgfqpoint{-0.048611in}{0.000000in}}%
\pgfusepath{stroke,fill}%
}%
\begin{pgfscope}%
\pgfsys@transformshift{0.800000in}{0.684955in}%
\pgfsys@useobject{currentmarker}{}%
\end{pgfscope}%
\end{pgfscope}%
\begin{pgfscope}%
\pgftext[x=0.614413in,y=0.632193in,left,base]{\sffamily\fontsize{10.000000}{12.000000}\selectfont 6}%
\end{pgfscope}%
\begin{pgfscope}%
\pgfsetbuttcap%
\pgfsetroundjoin%
\definecolor{currentfill}{rgb}{0.000000,0.000000,0.000000}%
\pgfsetfillcolor{currentfill}%
\pgfsetlinewidth{0.803000pt}%
\definecolor{currentstroke}{rgb}{0.000000,0.000000,0.000000}%
\pgfsetstrokecolor{currentstroke}%
\pgfsetdash{}{0pt}%
\pgfsys@defobject{currentmarker}{\pgfqpoint{-0.048611in}{0.000000in}}{\pgfqpoint{0.000000in}{0.000000in}}{%
\pgfpathmoveto{\pgfqpoint{0.000000in}{0.000000in}}%
\pgfpathlineto{\pgfqpoint{-0.048611in}{0.000000in}}%
\pgfusepath{stroke,fill}%
}%
\begin{pgfscope}%
\pgfsys@transformshift{0.800000in}{1.126769in}%
\pgfsys@useobject{currentmarker}{}%
\end{pgfscope}%
\end{pgfscope}%
\begin{pgfscope}%
\pgftext[x=0.614413in,y=1.074008in,left,base]{\sffamily\fontsize{10.000000}{12.000000}\selectfont 8}%
\end{pgfscope}%
\begin{pgfscope}%
\pgfsetbuttcap%
\pgfsetroundjoin%
\definecolor{currentfill}{rgb}{0.000000,0.000000,0.000000}%
\pgfsetfillcolor{currentfill}%
\pgfsetlinewidth{0.803000pt}%
\definecolor{currentstroke}{rgb}{0.000000,0.000000,0.000000}%
\pgfsetstrokecolor{currentstroke}%
\pgfsetdash{}{0pt}%
\pgfsys@defobject{currentmarker}{\pgfqpoint{-0.048611in}{0.000000in}}{\pgfqpoint{0.000000in}{0.000000in}}{%
\pgfpathmoveto{\pgfqpoint{0.000000in}{0.000000in}}%
\pgfpathlineto{\pgfqpoint{-0.048611in}{0.000000in}}%
\pgfusepath{stroke,fill}%
}%
\begin{pgfscope}%
\pgfsys@transformshift{0.800000in}{1.568584in}%
\pgfsys@useobject{currentmarker}{}%
\end{pgfscope}%
\end{pgfscope}%
\begin{pgfscope}%
\pgftext[x=0.526047in,y=1.515822in,left,base]{\sffamily\fontsize{10.000000}{12.000000}\selectfont 10}%
\end{pgfscope}%
\begin{pgfscope}%
\pgfsetbuttcap%
\pgfsetroundjoin%
\definecolor{currentfill}{rgb}{0.000000,0.000000,0.000000}%
\pgfsetfillcolor{currentfill}%
\pgfsetlinewidth{0.803000pt}%
\definecolor{currentstroke}{rgb}{0.000000,0.000000,0.000000}%
\pgfsetstrokecolor{currentstroke}%
\pgfsetdash{}{0pt}%
\pgfsys@defobject{currentmarker}{\pgfqpoint{-0.048611in}{0.000000in}}{\pgfqpoint{0.000000in}{0.000000in}}{%
\pgfpathmoveto{\pgfqpoint{0.000000in}{0.000000in}}%
\pgfpathlineto{\pgfqpoint{-0.048611in}{0.000000in}}%
\pgfusepath{stroke,fill}%
}%
\begin{pgfscope}%
\pgfsys@transformshift{0.800000in}{2.010398in}%
\pgfsys@useobject{currentmarker}{}%
\end{pgfscope}%
\end{pgfscope}%
\begin{pgfscope}%
\pgftext[x=0.526047in,y=1.957637in,left,base]{\sffamily\fontsize{10.000000}{12.000000}\selectfont 12}%
\end{pgfscope}%
\begin{pgfscope}%
\pgfsetbuttcap%
\pgfsetroundjoin%
\definecolor{currentfill}{rgb}{0.000000,0.000000,0.000000}%
\pgfsetfillcolor{currentfill}%
\pgfsetlinewidth{0.803000pt}%
\definecolor{currentstroke}{rgb}{0.000000,0.000000,0.000000}%
\pgfsetstrokecolor{currentstroke}%
\pgfsetdash{}{0pt}%
\pgfsys@defobject{currentmarker}{\pgfqpoint{-0.048611in}{0.000000in}}{\pgfqpoint{0.000000in}{0.000000in}}{%
\pgfpathmoveto{\pgfqpoint{0.000000in}{0.000000in}}%
\pgfpathlineto{\pgfqpoint{-0.048611in}{0.000000in}}%
\pgfusepath{stroke,fill}%
}%
\begin{pgfscope}%
\pgfsys@transformshift{0.800000in}{2.452213in}%
\pgfsys@useobject{currentmarker}{}%
\end{pgfscope}%
\end{pgfscope}%
\begin{pgfscope}%
\pgftext[x=0.526047in,y=2.399451in,left,base]{\sffamily\fontsize{10.000000}{12.000000}\selectfont 14}%
\end{pgfscope}%
\begin{pgfscope}%
\pgfsetbuttcap%
\pgfsetroundjoin%
\definecolor{currentfill}{rgb}{0.000000,0.000000,0.000000}%
\pgfsetfillcolor{currentfill}%
\pgfsetlinewidth{0.803000pt}%
\definecolor{currentstroke}{rgb}{0.000000,0.000000,0.000000}%
\pgfsetstrokecolor{currentstroke}%
\pgfsetdash{}{0pt}%
\pgfsys@defobject{currentmarker}{\pgfqpoint{-0.048611in}{0.000000in}}{\pgfqpoint{0.000000in}{0.000000in}}{%
\pgfpathmoveto{\pgfqpoint{0.000000in}{0.000000in}}%
\pgfpathlineto{\pgfqpoint{-0.048611in}{0.000000in}}%
\pgfusepath{stroke,fill}%
}%
\begin{pgfscope}%
\pgfsys@transformshift{0.800000in}{2.894028in}%
\pgfsys@useobject{currentmarker}{}%
\end{pgfscope}%
\end{pgfscope}%
\begin{pgfscope}%
\pgftext[x=0.526047in,y=2.841266in,left,base]{\sffamily\fontsize{10.000000}{12.000000}\selectfont 16}%
\end{pgfscope}%
\begin{pgfscope}%
\pgfsetbuttcap%
\pgfsetroundjoin%
\definecolor{currentfill}{rgb}{0.000000,0.000000,0.000000}%
\pgfsetfillcolor{currentfill}%
\pgfsetlinewidth{0.803000pt}%
\definecolor{currentstroke}{rgb}{0.000000,0.000000,0.000000}%
\pgfsetstrokecolor{currentstroke}%
\pgfsetdash{}{0pt}%
\pgfsys@defobject{currentmarker}{\pgfqpoint{-0.048611in}{0.000000in}}{\pgfqpoint{0.000000in}{0.000000in}}{%
\pgfpathmoveto{\pgfqpoint{0.000000in}{0.000000in}}%
\pgfpathlineto{\pgfqpoint{-0.048611in}{0.000000in}}%
\pgfusepath{stroke,fill}%
}%
\begin{pgfscope}%
\pgfsys@transformshift{0.800000in}{3.335842in}%
\pgfsys@useobject{currentmarker}{}%
\end{pgfscope}%
\end{pgfscope}%
\begin{pgfscope}%
\pgftext[x=0.526047in,y=3.283081in,left,base]{\sffamily\fontsize{10.000000}{12.000000}\selectfont 18}%
\end{pgfscope}%
\begin{pgfscope}%
\pgfsetbuttcap%
\pgfsetroundjoin%
\definecolor{currentfill}{rgb}{0.000000,0.000000,0.000000}%
\pgfsetfillcolor{currentfill}%
\pgfsetlinewidth{0.803000pt}%
\definecolor{currentstroke}{rgb}{0.000000,0.000000,0.000000}%
\pgfsetstrokecolor{currentstroke}%
\pgfsetdash{}{0pt}%
\pgfsys@defobject{currentmarker}{\pgfqpoint{-0.048611in}{0.000000in}}{\pgfqpoint{0.000000in}{0.000000in}}{%
\pgfpathmoveto{\pgfqpoint{0.000000in}{0.000000in}}%
\pgfpathlineto{\pgfqpoint{-0.048611in}{0.000000in}}%
\pgfusepath{stroke,fill}%
}%
\begin{pgfscope}%
\pgfsys@transformshift{0.800000in}{3.777657in}%
\pgfsys@useobject{currentmarker}{}%
\end{pgfscope}%
\end{pgfscope}%
\begin{pgfscope}%
\pgftext[x=0.526047in,y=3.724895in,left,base]{\sffamily\fontsize{10.000000}{12.000000}\selectfont 20}%
\end{pgfscope}%
\begin{pgfscope}%
\pgfsetbuttcap%
\pgfsetroundjoin%
\definecolor{currentfill}{rgb}{0.000000,0.000000,0.000000}%
\pgfsetfillcolor{currentfill}%
\pgfsetlinewidth{0.803000pt}%
\definecolor{currentstroke}{rgb}{0.000000,0.000000,0.000000}%
\pgfsetstrokecolor{currentstroke}%
\pgfsetdash{}{0pt}%
\pgfsys@defobject{currentmarker}{\pgfqpoint{-0.048611in}{0.000000in}}{\pgfqpoint{0.000000in}{0.000000in}}{%
\pgfpathmoveto{\pgfqpoint{0.000000in}{0.000000in}}%
\pgfpathlineto{\pgfqpoint{-0.048611in}{0.000000in}}%
\pgfusepath{stroke,fill}%
}%
\begin{pgfscope}%
\pgfsys@transformshift{0.800000in}{4.219471in}%
\pgfsys@useobject{currentmarker}{}%
\end{pgfscope}%
\end{pgfscope}%
\begin{pgfscope}%
\pgftext[x=0.526047in,y=4.166710in,left,base]{\sffamily\fontsize{10.000000}{12.000000}\selectfont 22}%
\end{pgfscope}%
\begin{pgfscope}%
\pgftext[x=0.470492in,y=2.376000in,,bottom,rotate=90.000000]{\sffamily\fontsize{10.000000}{12.000000}\selectfont \(\displaystyle \epsilon_n\)}%
\end{pgfscope}%
\begin{pgfscope}%
\pgfpathrectangle{\pgfqpoint{0.800000in}{0.528000in}}{\pgfqpoint{4.960000in}{3.696000in}} %
\pgfusepath{clip}%
\pgfsetrectcap%
\pgfsetroundjoin%
\pgfsetlinewidth{1.505625pt}%
\definecolor{currentstroke}{rgb}{0.121569,0.466667,0.705882}%
\pgfsetstrokecolor{currentstroke}%
\pgfsetdash{}{0pt}%
\pgfpathmoveto{\pgfqpoint{1.025455in}{0.696000in}}%
\pgfpathlineto{\pgfqpoint{1.030014in}{0.749018in}}%
\pgfpathlineto{\pgfqpoint{1.034573in}{0.828544in}}%
\pgfpathlineto{\pgfqpoint{1.039132in}{0.859471in}}%
\pgfpathlineto{\pgfqpoint{1.043692in}{0.919116in}}%
\pgfpathlineto{\pgfqpoint{1.048251in}{0.927953in}}%
\pgfpathlineto{\pgfqpoint{1.052810in}{0.996434in}}%
\pgfpathlineto{\pgfqpoint{1.057369in}{1.016316in}}%
\pgfpathlineto{\pgfqpoint{1.061928in}{1.058288in}}%
\pgfpathlineto{\pgfqpoint{1.071047in}{1.102469in}}%
\pgfpathlineto{\pgfqpoint{1.075606in}{1.190832in}}%
\pgfpathlineto{\pgfqpoint{1.084725in}{1.221759in}}%
\pgfpathlineto{\pgfqpoint{1.089284in}{1.279195in}}%
\pgfpathlineto{\pgfqpoint{1.093843in}{1.316750in}}%
\pgfpathlineto{\pgfqpoint{1.098402in}{1.325586in}}%
\pgfpathlineto{\pgfqpoint{1.102962in}{1.265941in}}%
\pgfpathlineto{\pgfqpoint{1.107521in}{1.354304in}}%
\pgfpathlineto{\pgfqpoint{1.112080in}{1.391858in}}%
\pgfpathlineto{\pgfqpoint{1.121199in}{1.442667in}}%
\pgfpathlineto{\pgfqpoint{1.125758in}{1.484639in}}%
\pgfpathlineto{\pgfqpoint{1.130317in}{1.491266in}}%
\pgfpathlineto{\pgfqpoint{1.134876in}{1.493475in}}%
\pgfpathlineto{\pgfqpoint{1.143995in}{1.557538in}}%
\pgfpathlineto{\pgfqpoint{1.148554in}{1.537657in}}%
\pgfpathlineto{\pgfqpoint{1.153113in}{1.595093in}}%
\pgfpathlineto{\pgfqpoint{1.157673in}{1.491266in}}%
\pgfpathlineto{\pgfqpoint{1.162232in}{1.637065in}}%
\pgfpathlineto{\pgfqpoint{1.166791in}{1.643692in}}%
\pgfpathlineto{\pgfqpoint{1.171350in}{1.623811in}}%
\pgfpathlineto{\pgfqpoint{1.175910in}{1.661365in}}%
\pgfpathlineto{\pgfqpoint{1.180469in}{1.603929in}}%
\pgfpathlineto{\pgfqpoint{1.185028in}{1.650320in}}%
\pgfpathlineto{\pgfqpoint{1.189587in}{1.597302in}}%
\pgfpathlineto{\pgfqpoint{1.194147in}{1.751937in}}%
\pgfpathlineto{\pgfqpoint{1.198706in}{1.734264in}}%
\pgfpathlineto{\pgfqpoint{1.203265in}{1.685665in}}%
\pgfpathlineto{\pgfqpoint{1.207824in}{1.785073in}}%
\pgfpathlineto{\pgfqpoint{1.212383in}{1.749728in}}%
\pgfpathlineto{\pgfqpoint{1.216943in}{1.705546in}}%
\pgfpathlineto{\pgfqpoint{1.221502in}{1.822627in}}%
\pgfpathlineto{\pgfqpoint{1.226061in}{1.793909in}}%
\pgfpathlineto{\pgfqpoint{1.230620in}{1.893318in}}%
\pgfpathlineto{\pgfqpoint{1.235180in}{1.776237in}}%
\pgfpathlineto{\pgfqpoint{1.239739in}{1.864600in}}%
\pgfpathlineto{\pgfqpoint{1.244298in}{1.846927in}}%
\pgfpathlineto{\pgfqpoint{1.248857in}{1.877854in}}%
\pgfpathlineto{\pgfqpoint{1.253417in}{1.893318in}}%
\pgfpathlineto{\pgfqpoint{1.257976in}{1.986099in}}%
\pgfpathlineto{\pgfqpoint{1.262535in}{1.910990in}}%
\pgfpathlineto{\pgfqpoint{1.267094in}{1.972844in}}%
\pgfpathlineto{\pgfqpoint{1.271654in}{1.968426in}}%
\pgfpathlineto{\pgfqpoint{1.276213in}{1.895527in}}%
\pgfpathlineto{\pgfqpoint{1.280772in}{1.895527in}}%
\pgfpathlineto{\pgfqpoint{1.285331in}{1.961799in}}%
\pgfpathlineto{\pgfqpoint{1.289891in}{1.957381in}}%
\pgfpathlineto{\pgfqpoint{1.294450in}{1.961799in}}%
\pgfpathlineto{\pgfqpoint{1.299009in}{1.928663in}}%
\pgfpathlineto{\pgfqpoint{1.303568in}{2.043535in}}%
\pgfpathlineto{\pgfqpoint{1.308128in}{1.952963in}}%
\pgfpathlineto{\pgfqpoint{1.312687in}{2.065625in}}%
\pgfpathlineto{\pgfqpoint{1.317246in}{2.125270in}}%
\pgfpathlineto{\pgfqpoint{1.321805in}{1.975053in}}%
\pgfpathlineto{\pgfqpoint{1.326365in}{2.008189in}}%
\pgfpathlineto{\pgfqpoint{1.330924in}{2.089925in}}%
\pgfpathlineto{\pgfqpoint{1.335483in}{2.023653in}}%
\pgfpathlineto{\pgfqpoint{1.340042in}{2.182706in}}%
\pgfpathlineto{\pgfqpoint{1.344602in}{2.134107in}}%
\pgfpathlineto{\pgfqpoint{1.349161in}{2.105389in}}%
\pgfpathlineto{\pgfqpoint{1.353720in}{2.054580in}}%
\pgfpathlineto{\pgfqpoint{1.358279in}{2.114225in}}%
\pgfpathlineto{\pgfqpoint{1.362838in}{2.158406in}}%
\pgfpathlineto{\pgfqpoint{1.367398in}{2.182706in}}%
\pgfpathlineto{\pgfqpoint{1.371957in}{2.173870in}}%
\pgfpathlineto{\pgfqpoint{1.376516in}{2.211424in}}%
\pgfpathlineto{\pgfqpoint{1.381075in}{2.140734in}}%
\pgfpathlineto{\pgfqpoint{1.385635in}{2.103179in}}%
\pgfpathlineto{\pgfqpoint{1.390194in}{2.134107in}}%
\pgfpathlineto{\pgfqpoint{1.394753in}{2.153988in}}%
\pgfpathlineto{\pgfqpoint{1.399312in}{2.162824in}}%
\pgfpathlineto{\pgfqpoint{1.403872in}{2.176079in}}%
\pgfpathlineto{\pgfqpoint{1.408431in}{2.215842in}}%
\pgfpathlineto{\pgfqpoint{1.412990in}{2.178288in}}%
\pgfpathlineto{\pgfqpoint{1.417549in}{2.191542in}}%
\pgfpathlineto{\pgfqpoint{1.422109in}{2.224679in}}%
\pgfpathlineto{\pgfqpoint{1.426668in}{2.180497in}}%
\pgfpathlineto{\pgfqpoint{1.431227in}{2.264442in}}%
\pgfpathlineto{\pgfqpoint{1.435786in}{2.255606in}}%
\pgfpathlineto{\pgfqpoint{1.440346in}{2.288742in}}%
\pgfpathlineto{\pgfqpoint{1.444905in}{2.310832in}}%
\pgfpathlineto{\pgfqpoint{1.449464in}{2.215842in}}%
\pgfpathlineto{\pgfqpoint{1.454023in}{2.268860in}}%
\pgfpathlineto{\pgfqpoint{1.458583in}{2.262233in}}%
\pgfpathlineto{\pgfqpoint{1.463142in}{2.266651in}}%
\pgfpathlineto{\pgfqpoint{1.467701in}{2.288742in}}%
\pgfpathlineto{\pgfqpoint{1.472260in}{2.304205in}}%
\pgfpathlineto{\pgfqpoint{1.476820in}{2.315250in}}%
\pgfpathlineto{\pgfqpoint{1.481379in}{2.244560in}}%
\pgfpathlineto{\pgfqpoint{1.485938in}{2.308623in}}%
\pgfpathlineto{\pgfqpoint{1.490497in}{2.330714in}}%
\pgfpathlineto{\pgfqpoint{1.495057in}{2.306414in}}%
\pgfpathlineto{\pgfqpoint{1.499616in}{2.355014in}}%
\pgfpathlineto{\pgfqpoint{1.504175in}{2.425704in}}%
\pgfpathlineto{\pgfqpoint{1.508734in}{2.363850in}}%
\pgfpathlineto{\pgfqpoint{1.513294in}{2.399195in}}%
\pgfpathlineto{\pgfqpoint{1.517853in}{2.315250in}}%
\pgfpathlineto{\pgfqpoint{1.522412in}{2.341759in}}%
\pgfpathlineto{\pgfqpoint{1.526971in}{2.399195in}}%
\pgfpathlineto{\pgfqpoint{1.531530in}{2.379314in}}%
\pgfpathlineto{\pgfqpoint{1.536090in}{2.379314in}}%
\pgfpathlineto{\pgfqpoint{1.540649in}{2.476513in}}%
\pgfpathlineto{\pgfqpoint{1.545208in}{2.350596in}}%
\pgfpathlineto{\pgfqpoint{1.549767in}{2.423495in}}%
\pgfpathlineto{\pgfqpoint{1.554327in}{2.399195in}}%
\pgfpathlineto{\pgfqpoint{1.558886in}{2.416868in}}%
\pgfpathlineto{\pgfqpoint{1.563445in}{2.317460in}}%
\pgfpathlineto{\pgfqpoint{1.568004in}{2.476513in}}%
\pgfpathlineto{\pgfqpoint{1.572564in}{2.441168in}}%
\pgfpathlineto{\pgfqpoint{1.577123in}{2.454422in}}%
\pgfpathlineto{\pgfqpoint{1.581682in}{2.390359in}}%
\pgfpathlineto{\pgfqpoint{1.590801in}{2.489767in}}%
\pgfpathlineto{\pgfqpoint{1.595360in}{2.430122in}}%
\pgfpathlineto{\pgfqpoint{1.599919in}{2.503022in}}%
\pgfpathlineto{\pgfqpoint{1.604478in}{2.531740in}}%
\pgfpathlineto{\pgfqpoint{1.609038in}{2.454422in}}%
\pgfpathlineto{\pgfqpoint{1.613597in}{2.531740in}}%
\pgfpathlineto{\pgfqpoint{1.618156in}{2.496394in}}%
\pgfpathlineto{\pgfqpoint{1.622715in}{2.503022in}}%
\pgfpathlineto{\pgfqpoint{1.627275in}{2.491976in}}%
\pgfpathlineto{\pgfqpoint{1.631834in}{2.474304in}}%
\pgfpathlineto{\pgfqpoint{1.636393in}{2.487558in}}%
\pgfpathlineto{\pgfqpoint{1.640952in}{2.507440in}}%
\pgfpathlineto{\pgfqpoint{1.645512in}{2.447795in}}%
\pgfpathlineto{\pgfqpoint{1.654630in}{2.507440in}}%
\pgfpathlineto{\pgfqpoint{1.659189in}{2.516276in}}%
\pgfpathlineto{\pgfqpoint{1.663749in}{2.498604in}}%
\pgfpathlineto{\pgfqpoint{1.668308in}{2.540576in}}%
\pgfpathlineto{\pgfqpoint{1.672867in}{2.536158in}}%
\pgfpathlineto{\pgfqpoint{1.677426in}{2.593594in}}%
\pgfpathlineto{\pgfqpoint{1.681985in}{2.562667in}}%
\pgfpathlineto{\pgfqpoint{1.686545in}{2.553830in}}%
\pgfpathlineto{\pgfqpoint{1.691104in}{2.609057in}}%
\pgfpathlineto{\pgfqpoint{1.695663in}{2.536158in}}%
\pgfpathlineto{\pgfqpoint{1.700222in}{2.609057in}}%
\pgfpathlineto{\pgfqpoint{1.704782in}{2.575921in}}%
\pgfpathlineto{\pgfqpoint{1.709341in}{2.474304in}}%
\pgfpathlineto{\pgfqpoint{1.713900in}{2.507440in}}%
\pgfpathlineto{\pgfqpoint{1.718459in}{2.668702in}}%
\pgfpathlineto{\pgfqpoint{1.723019in}{2.666493in}}%
\pgfpathlineto{\pgfqpoint{1.727578in}{2.549412in}}%
\pgfpathlineto{\pgfqpoint{1.732137in}{2.542785in}}%
\pgfpathlineto{\pgfqpoint{1.736696in}{2.564876in}}%
\pgfpathlineto{\pgfqpoint{1.741256in}{2.571503in}}%
\pgfpathlineto{\pgfqpoint{1.745815in}{2.657657in}}%
\pgfpathlineto{\pgfqpoint{1.750374in}{2.653239in}}%
\pgfpathlineto{\pgfqpoint{1.754933in}{2.589176in}}%
\pgfpathlineto{\pgfqpoint{1.764052in}{2.730556in}}%
\pgfpathlineto{\pgfqpoint{1.768611in}{2.626730in}}%
\pgfpathlineto{\pgfqpoint{1.773170in}{2.648821in}}%
\pgfpathlineto{\pgfqpoint{1.777730in}{2.732765in}}%
\pgfpathlineto{\pgfqpoint{1.786848in}{2.589176in}}%
\pgfpathlineto{\pgfqpoint{1.791407in}{2.620103in}}%
\pgfpathlineto{\pgfqpoint{1.795967in}{2.668702in}}%
\pgfpathlineto{\pgfqpoint{1.800526in}{2.653239in}}%
\pgfpathlineto{\pgfqpoint{1.805085in}{2.655448in}}%
\pgfpathlineto{\pgfqpoint{1.809644in}{2.805665in}}%
\pgfpathlineto{\pgfqpoint{1.814204in}{2.748229in}}%
\pgfpathlineto{\pgfqpoint{1.818763in}{2.761483in}}%
\pgfpathlineto{\pgfqpoint{1.823322in}{2.620103in}}%
\pgfpathlineto{\pgfqpoint{1.827881in}{2.757065in}}%
\pgfpathlineto{\pgfqpoint{1.832440in}{2.684166in}}%
\pgfpathlineto{\pgfqpoint{1.837000in}{2.759274in}}%
\pgfpathlineto{\pgfqpoint{1.841559in}{2.759274in}}%
\pgfpathlineto{\pgfqpoint{1.846118in}{2.686375in}}%
\pgfpathlineto{\pgfqpoint{1.850677in}{2.695211in}}%
\pgfpathlineto{\pgfqpoint{1.855237in}{2.741602in}}%
\pgfpathlineto{\pgfqpoint{1.859796in}{2.763692in}}%
\pgfpathlineto{\pgfqpoint{1.864355in}{2.799037in}}%
\pgfpathlineto{\pgfqpoint{1.868914in}{2.746020in}}%
\pgfpathlineto{\pgfqpoint{1.873474in}{2.715093in}}%
\pgfpathlineto{\pgfqpoint{1.878033in}{2.757065in}}%
\pgfpathlineto{\pgfqpoint{1.882592in}{2.737183in}}%
\pgfpathlineto{\pgfqpoint{1.887151in}{2.781365in}}%
\pgfpathlineto{\pgfqpoint{1.896270in}{2.655448in}}%
\pgfpathlineto{\pgfqpoint{1.900829in}{2.757065in}}%
\pgfpathlineto{\pgfqpoint{1.905388in}{2.763692in}}%
\pgfpathlineto{\pgfqpoint{1.909948in}{2.825546in}}%
\pgfpathlineto{\pgfqpoint{1.914507in}{2.721720in}}%
\pgfpathlineto{\pgfqpoint{1.919066in}{2.821128in}}%
\pgfpathlineto{\pgfqpoint{1.923625in}{2.721720in}}%
\pgfpathlineto{\pgfqpoint{1.928185in}{2.854264in}}%
\pgfpathlineto{\pgfqpoint{1.932744in}{2.792410in}}%
\pgfpathlineto{\pgfqpoint{1.937303in}{2.715093in}}%
\pgfpathlineto{\pgfqpoint{1.946422in}{2.856473in}}%
\pgfpathlineto{\pgfqpoint{1.955540in}{2.801247in}}%
\pgfpathlineto{\pgfqpoint{1.960099in}{2.737183in}}%
\pgfpathlineto{\pgfqpoint{1.964659in}{2.787992in}}%
\pgfpathlineto{\pgfqpoint{1.969218in}{2.774738in}}%
\pgfpathlineto{\pgfqpoint{1.973777in}{2.807874in}}%
\pgfpathlineto{\pgfqpoint{1.978336in}{2.765901in}}%
\pgfpathlineto{\pgfqpoint{1.982895in}{2.807874in}}%
\pgfpathlineto{\pgfqpoint{1.987455in}{2.860892in}}%
\pgfpathlineto{\pgfqpoint{1.992014in}{2.799037in}}%
\pgfpathlineto{\pgfqpoint{1.996573in}{2.803456in}}%
\pgfpathlineto{\pgfqpoint{2.001132in}{2.849846in}}%
\pgfpathlineto{\pgfqpoint{2.005692in}{2.790201in}}%
\pgfpathlineto{\pgfqpoint{2.010251in}{2.898446in}}%
\pgfpathlineto{\pgfqpoint{2.014810in}{2.955882in}}%
\pgfpathlineto{\pgfqpoint{2.019369in}{2.836592in}}%
\pgfpathlineto{\pgfqpoint{2.023929in}{2.902864in}}%
\pgfpathlineto{\pgfqpoint{2.028488in}{2.827755in}}%
\pgfpathlineto{\pgfqpoint{2.033047in}{2.869728in}}%
\pgfpathlineto{\pgfqpoint{2.037606in}{2.814501in}}%
\pgfpathlineto{\pgfqpoint{2.046725in}{2.757065in}}%
\pgfpathlineto{\pgfqpoint{2.051284in}{2.854264in}}%
\pgfpathlineto{\pgfqpoint{2.055843in}{2.876355in}}%
\pgfpathlineto{\pgfqpoint{2.060403in}{2.874146in}}%
\pgfpathlineto{\pgfqpoint{2.064962in}{2.891819in}}%
\pgfpathlineto{\pgfqpoint{2.069521in}{2.878564in}}%
\pgfpathlineto{\pgfqpoint{2.074080in}{2.887400in}}%
\pgfpathlineto{\pgfqpoint{2.078640in}{2.989018in}}%
\pgfpathlineto{\pgfqpoint{2.083199in}{2.931582in}}%
\pgfpathlineto{\pgfqpoint{2.087758in}{2.958091in}}%
\pgfpathlineto{\pgfqpoint{2.092317in}{2.947045in}}%
\pgfpathlineto{\pgfqpoint{2.096877in}{2.869728in}}%
\pgfpathlineto{\pgfqpoint{2.101436in}{2.825546in}}%
\pgfpathlineto{\pgfqpoint{2.110554in}{2.907282in}}%
\pgfpathlineto{\pgfqpoint{2.115114in}{2.951464in}}%
\pgfpathlineto{\pgfqpoint{2.119673in}{2.949254in}}%
\pgfpathlineto{\pgfqpoint{2.128791in}{2.911700in}}%
\pgfpathlineto{\pgfqpoint{2.133350in}{2.902864in}}%
\pgfpathlineto{\pgfqpoint{2.137910in}{2.898446in}}%
\pgfpathlineto{\pgfqpoint{2.142469in}{2.960300in}}%
\pgfpathlineto{\pgfqpoint{2.147028in}{2.980181in}}%
\pgfpathlineto{\pgfqpoint{2.151587in}{2.902864in}}%
\pgfpathlineto{\pgfqpoint{2.156147in}{2.942627in}}%
\pgfpathlineto{\pgfqpoint{2.160706in}{2.891819in}}%
\pgfpathlineto{\pgfqpoint{2.165265in}{3.110517in}}%
\pgfpathlineto{\pgfqpoint{2.174384in}{2.882982in}}%
\pgfpathlineto{\pgfqpoint{2.178943in}{2.960300in}}%
\pgfpathlineto{\pgfqpoint{2.183502in}{2.894028in}}%
\pgfpathlineto{\pgfqpoint{2.188061in}{2.944836in}}%
\pgfpathlineto{\pgfqpoint{2.192621in}{2.938209in}}%
\pgfpathlineto{\pgfqpoint{2.197180in}{2.929373in}}%
\pgfpathlineto{\pgfqpoint{2.201739in}{2.933791in}}%
\pgfpathlineto{\pgfqpoint{2.206298in}{3.048663in}}%
\pgfpathlineto{\pgfqpoint{2.210858in}{2.938209in}}%
\pgfpathlineto{\pgfqpoint{2.215417in}{3.024363in}}%
\pgfpathlineto{\pgfqpoint{2.219976in}{2.984600in}}%
\pgfpathlineto{\pgfqpoint{2.229095in}{2.874146in}}%
\pgfpathlineto{\pgfqpoint{2.233654in}{3.000063in}}%
\pgfpathlineto{\pgfqpoint{2.238213in}{2.958091in}}%
\pgfpathlineto{\pgfqpoint{2.242772in}{3.097262in}}%
\pgfpathlineto{\pgfqpoint{2.247332in}{2.989018in}}%
\pgfpathlineto{\pgfqpoint{2.251891in}{3.066335in}}%
\pgfpathlineto{\pgfqpoint{2.256450in}{2.891819in}}%
\pgfpathlineto{\pgfqpoint{2.261009in}{2.966927in}}%
\pgfpathlineto{\pgfqpoint{2.265569in}{2.986809in}}%
\pgfpathlineto{\pgfqpoint{2.270128in}{2.953673in}}%
\pgfpathlineto{\pgfqpoint{2.274687in}{2.900655in}}%
\pgfpathlineto{\pgfqpoint{2.279246in}{2.929373in}}%
\pgfpathlineto{\pgfqpoint{2.288365in}{3.084008in}}%
\pgfpathlineto{\pgfqpoint{2.292924in}{3.011108in}}%
\pgfpathlineto{\pgfqpoint{2.297483in}{3.044245in}}%
\pgfpathlineto{\pgfqpoint{2.302042in}{3.026572in}}%
\pgfpathlineto{\pgfqpoint{2.306602in}{3.044245in}}%
\pgfpathlineto{\pgfqpoint{2.311161in}{3.044245in}}%
\pgfpathlineto{\pgfqpoint{2.315720in}{2.962509in}}%
\pgfpathlineto{\pgfqpoint{2.320279in}{3.114935in}}%
\pgfpathlineto{\pgfqpoint{2.324839in}{2.960300in}}%
\pgfpathlineto{\pgfqpoint{2.329398in}{3.042036in}}%
\pgfpathlineto{\pgfqpoint{2.333957in}{3.079590in}}%
\pgfpathlineto{\pgfqpoint{2.338516in}{3.035408in}}%
\pgfpathlineto{\pgfqpoint{2.343076in}{3.050872in}}%
\pgfpathlineto{\pgfqpoint{2.347635in}{3.044245in}}%
\pgfpathlineto{\pgfqpoint{2.356753in}{3.044245in}}%
\pgfpathlineto{\pgfqpoint{2.361313in}{2.980181in}}%
\pgfpathlineto{\pgfqpoint{2.365872in}{3.070753in}}%
\pgfpathlineto{\pgfqpoint{2.370431in}{3.066335in}}%
\pgfpathlineto{\pgfqpoint{2.374990in}{3.019945in}}%
\pgfpathlineto{\pgfqpoint{2.379550in}{3.130398in}}%
\pgfpathlineto{\pgfqpoint{2.384109in}{3.050872in}}%
\pgfpathlineto{\pgfqpoint{2.388668in}{3.152489in}}%
\pgfpathlineto{\pgfqpoint{2.393227in}{3.084008in}}%
\pgfpathlineto{\pgfqpoint{2.397787in}{3.061917in}}%
\pgfpathlineto{\pgfqpoint{2.402346in}{3.114935in}}%
\pgfpathlineto{\pgfqpoint{2.406905in}{3.035408in}}%
\pgfpathlineto{\pgfqpoint{2.411464in}{3.070753in}}%
\pgfpathlineto{\pgfqpoint{2.416024in}{3.033199in}}%
\pgfpathlineto{\pgfqpoint{2.420583in}{3.061917in}}%
\pgfpathlineto{\pgfqpoint{2.425142in}{2.980181in}}%
\pgfpathlineto{\pgfqpoint{2.429701in}{3.125980in}}%
\pgfpathlineto{\pgfqpoint{2.434261in}{3.119353in}}%
\pgfpathlineto{\pgfqpoint{2.438820in}{3.042036in}}%
\pgfpathlineto{\pgfqpoint{2.443379in}{3.099471in}}%
\pgfpathlineto{\pgfqpoint{2.447938in}{3.068544in}}%
\pgfpathlineto{\pgfqpoint{2.457057in}{3.205507in}}%
\pgfpathlineto{\pgfqpoint{2.461616in}{3.053081in}}%
\pgfpathlineto{\pgfqpoint{2.466175in}{3.019945in}}%
\pgfpathlineto{\pgfqpoint{2.470734in}{3.119353in}}%
\pgfpathlineto{\pgfqpoint{2.475294in}{3.081799in}}%
\pgfpathlineto{\pgfqpoint{2.479853in}{3.216552in}}%
\pgfpathlineto{\pgfqpoint{2.484412in}{3.218761in}}%
\pgfpathlineto{\pgfqpoint{2.488971in}{3.154698in}}%
\pgfpathlineto{\pgfqpoint{2.493531in}{3.148071in}}%
\pgfpathlineto{\pgfqpoint{2.498090in}{3.046454in}}%
\pgfpathlineto{\pgfqpoint{2.502649in}{3.119353in}}%
\pgfpathlineto{\pgfqpoint{2.511768in}{3.024363in}}%
\pgfpathlineto{\pgfqpoint{2.516327in}{3.170162in}}%
\pgfpathlineto{\pgfqpoint{2.520886in}{3.088426in}}%
\pgfpathlineto{\pgfqpoint{2.525445in}{3.185625in}}%
\pgfpathlineto{\pgfqpoint{2.530005in}{3.174580in}}%
\pgfpathlineto{\pgfqpoint{2.534564in}{3.172371in}}%
\pgfpathlineto{\pgfqpoint{2.539123in}{3.185625in}}%
\pgfpathlineto{\pgfqpoint{2.543682in}{3.256316in}}%
\pgfpathlineto{\pgfqpoint{2.548242in}{3.075172in}}%
\pgfpathlineto{\pgfqpoint{2.552801in}{3.167953in}}%
\pgfpathlineto{\pgfqpoint{2.557360in}{3.035408in}}%
\pgfpathlineto{\pgfqpoint{2.561919in}{3.232016in}}%
\pgfpathlineto{\pgfqpoint{2.566479in}{3.256316in}}%
\pgfpathlineto{\pgfqpoint{2.571038in}{3.251897in}}%
\pgfpathlineto{\pgfqpoint{2.575597in}{3.170162in}}%
\pgfpathlineto{\pgfqpoint{2.580156in}{3.141444in}}%
\pgfpathlineto{\pgfqpoint{2.584716in}{3.196671in}}%
\pgfpathlineto{\pgfqpoint{2.589275in}{3.236434in}}%
\pgfpathlineto{\pgfqpoint{2.593834in}{3.218761in}}%
\pgfpathlineto{\pgfqpoint{2.598393in}{3.130398in}}%
\pgfpathlineto{\pgfqpoint{2.602952in}{3.145862in}}%
\pgfpathlineto{\pgfqpoint{2.607512in}{3.205507in}}%
\pgfpathlineto{\pgfqpoint{2.612071in}{3.170162in}}%
\pgfpathlineto{\pgfqpoint{2.616630in}{3.258525in}}%
\pgfpathlineto{\pgfqpoint{2.621189in}{3.201089in}}%
\pgfpathlineto{\pgfqpoint{2.625749in}{3.103890in}}%
\pgfpathlineto{\pgfqpoint{2.630308in}{3.203298in}}%
\pgfpathlineto{\pgfqpoint{2.634867in}{3.232016in}}%
\pgfpathlineto{\pgfqpoint{2.639426in}{3.145862in}}%
\pgfpathlineto{\pgfqpoint{2.643986in}{3.234225in}}%
\pgfpathlineto{\pgfqpoint{2.648545in}{3.170162in}}%
\pgfpathlineto{\pgfqpoint{2.653104in}{3.141444in}}%
\pgfpathlineto{\pgfqpoint{2.657663in}{3.271779in}}%
\pgfpathlineto{\pgfqpoint{2.666782in}{3.128189in}}%
\pgfpathlineto{\pgfqpoint{2.671341in}{3.298288in}}%
\pgfpathlineto{\pgfqpoint{2.675900in}{3.145862in}}%
\pgfpathlineto{\pgfqpoint{2.680460in}{3.178998in}}%
\pgfpathlineto{\pgfqpoint{2.685019in}{3.152489in}}%
\pgfpathlineto{\pgfqpoint{2.689578in}{3.176789in}}%
\pgfpathlineto{\pgfqpoint{2.694137in}{3.187834in}}%
\pgfpathlineto{\pgfqpoint{2.698697in}{3.150280in}}%
\pgfpathlineto{\pgfqpoint{2.703256in}{3.196671in}}%
\pgfpathlineto{\pgfqpoint{2.707815in}{3.278406in}}%
\pgfpathlineto{\pgfqpoint{2.712374in}{3.243061in}}%
\pgfpathlineto{\pgfqpoint{2.716934in}{3.190043in}}%
\pgfpathlineto{\pgfqpoint{2.721493in}{3.190043in}}%
\pgfpathlineto{\pgfqpoint{2.726052in}{3.327006in}}%
\pgfpathlineto{\pgfqpoint{2.735171in}{3.163535in}}%
\pgfpathlineto{\pgfqpoint{2.739730in}{3.214343in}}%
\pgfpathlineto{\pgfqpoint{2.744289in}{3.289452in}}%
\pgfpathlineto{\pgfqpoint{2.748848in}{3.276197in}}%
\pgfpathlineto{\pgfqpoint{2.753407in}{3.207716in}}%
\pgfpathlineto{\pgfqpoint{2.757967in}{3.212134in}}%
\pgfpathlineto{\pgfqpoint{2.762526in}{3.260734in}}%
\pgfpathlineto{\pgfqpoint{2.767085in}{3.240852in}}%
\pgfpathlineto{\pgfqpoint{2.771644in}{3.287243in}}%
\pgfpathlineto{\pgfqpoint{2.776204in}{3.245270in}}%
\pgfpathlineto{\pgfqpoint{2.780763in}{3.269570in}}%
\pgfpathlineto{\pgfqpoint{2.785322in}{3.203298in}}%
\pgfpathlineto{\pgfqpoint{2.789881in}{3.267361in}}%
\pgfpathlineto{\pgfqpoint{2.794441in}{3.282824in}}%
\pgfpathlineto{\pgfqpoint{2.799000in}{3.236434in}}%
\pgfpathlineto{\pgfqpoint{2.803559in}{3.353515in}}%
\pgfpathlineto{\pgfqpoint{2.808118in}{3.307124in}}%
\pgfpathlineto{\pgfqpoint{2.812678in}{3.289452in}}%
\pgfpathlineto{\pgfqpoint{2.817237in}{3.209925in}}%
\pgfpathlineto{\pgfqpoint{2.821796in}{3.258525in}}%
\pgfpathlineto{\pgfqpoint{2.826355in}{3.172371in}}%
\pgfpathlineto{\pgfqpoint{2.830915in}{3.229807in}}%
\pgfpathlineto{\pgfqpoint{2.835474in}{3.223179in}}%
\pgfpathlineto{\pgfqpoint{2.840033in}{3.232016in}}%
\pgfpathlineto{\pgfqpoint{2.844592in}{3.300497in}}%
\pgfpathlineto{\pgfqpoint{2.849152in}{3.269570in}}%
\pgfpathlineto{\pgfqpoint{2.853711in}{3.311542in}}%
\pgfpathlineto{\pgfqpoint{2.858270in}{3.245270in}}%
\pgfpathlineto{\pgfqpoint{2.862829in}{3.298288in}}%
\pgfpathlineto{\pgfqpoint{2.871948in}{3.185625in}}%
\pgfpathlineto{\pgfqpoint{2.876507in}{3.262943in}}%
\pgfpathlineto{\pgfqpoint{2.881066in}{3.313751in}}%
\pgfpathlineto{\pgfqpoint{2.885626in}{3.285034in}}%
\pgfpathlineto{\pgfqpoint{2.890185in}{3.313751in}}%
\pgfpathlineto{\pgfqpoint{2.899303in}{3.293870in}}%
\pgfpathlineto{\pgfqpoint{2.903862in}{3.421996in}}%
\pgfpathlineto{\pgfqpoint{2.908422in}{3.236434in}}%
\pgfpathlineto{\pgfqpoint{2.912981in}{3.417578in}}%
\pgfpathlineto{\pgfqpoint{2.917540in}{3.364560in}}%
\pgfpathlineto{\pgfqpoint{2.922099in}{3.229807in}}%
\pgfpathlineto{\pgfqpoint{2.926659in}{3.413160in}}%
\pgfpathlineto{\pgfqpoint{2.931218in}{3.404323in}}%
\pgfpathlineto{\pgfqpoint{2.935777in}{3.273988in}}%
\pgfpathlineto{\pgfqpoint{2.940336in}{3.380024in}}%
\pgfpathlineto{\pgfqpoint{2.944896in}{3.280615in}}%
\pgfpathlineto{\pgfqpoint{2.949455in}{3.236434in}}%
\pgfpathlineto{\pgfqpoint{2.954014in}{3.335842in}}%
\pgfpathlineto{\pgfqpoint{2.958573in}{3.234225in}}%
\pgfpathlineto{\pgfqpoint{2.963133in}{3.207716in}}%
\pgfpathlineto{\pgfqpoint{2.967692in}{3.346888in}}%
\pgfpathlineto{\pgfqpoint{2.972251in}{3.262943in}}%
\pgfpathlineto{\pgfqpoint{2.976810in}{3.256316in}}%
\pgfpathlineto{\pgfqpoint{2.981370in}{3.342469in}}%
\pgfpathlineto{\pgfqpoint{2.985929in}{3.402114in}}%
\pgfpathlineto{\pgfqpoint{2.990488in}{3.229807in}}%
\pgfpathlineto{\pgfqpoint{2.995047in}{3.315961in}}%
\pgfpathlineto{\pgfqpoint{2.999607in}{3.430832in}}%
\pgfpathlineto{\pgfqpoint{3.004166in}{3.285034in}}%
\pgfpathlineto{\pgfqpoint{3.008725in}{3.232016in}}%
\pgfpathlineto{\pgfqpoint{3.013284in}{3.243061in}}%
\pgfpathlineto{\pgfqpoint{3.022403in}{3.366769in}}%
\pgfpathlineto{\pgfqpoint{3.026962in}{3.267361in}}%
\pgfpathlineto{\pgfqpoint{3.031521in}{3.435250in}}%
\pgfpathlineto{\pgfqpoint{3.040640in}{3.315961in}}%
\pgfpathlineto{\pgfqpoint{3.045199in}{3.472805in}}%
\pgfpathlineto{\pgfqpoint{3.049758in}{3.424205in}}%
\pgfpathlineto{\pgfqpoint{3.054317in}{3.296079in}}%
\pgfpathlineto{\pgfqpoint{3.058877in}{3.437460in}}%
\pgfpathlineto{\pgfqpoint{3.063436in}{3.353515in}}%
\pgfpathlineto{\pgfqpoint{3.067995in}{3.307124in}}%
\pgfpathlineto{\pgfqpoint{3.072554in}{3.291661in}}%
\pgfpathlineto{\pgfqpoint{3.077114in}{3.322588in}}%
\pgfpathlineto{\pgfqpoint{3.081673in}{3.424205in}}%
\pgfpathlineto{\pgfqpoint{3.086232in}{3.353515in}}%
\pgfpathlineto{\pgfqpoint{3.090791in}{3.404323in}}%
\pgfpathlineto{\pgfqpoint{3.095351in}{3.382233in}}%
\pgfpathlineto{\pgfqpoint{3.099910in}{3.351306in}}%
\pgfpathlineto{\pgfqpoint{3.104469in}{3.240852in}}%
\pgfpathlineto{\pgfqpoint{3.109028in}{3.393278in}}%
\pgfpathlineto{\pgfqpoint{3.113588in}{3.382233in}}%
\pgfpathlineto{\pgfqpoint{3.118147in}{3.322588in}}%
\pgfpathlineto{\pgfqpoint{3.122706in}{3.298288in}}%
\pgfpathlineto{\pgfqpoint{3.127265in}{3.364560in}}%
\pgfpathlineto{\pgfqpoint{3.131825in}{3.366769in}}%
\pgfpathlineto{\pgfqpoint{3.136384in}{3.357933in}}%
\pgfpathlineto{\pgfqpoint{3.140943in}{3.344679in}}%
\pgfpathlineto{\pgfqpoint{3.145502in}{3.393278in}}%
\pgfpathlineto{\pgfqpoint{3.150062in}{3.313751in}}%
\pgfpathlineto{\pgfqpoint{3.154621in}{3.384442in}}%
\pgfpathlineto{\pgfqpoint{3.159180in}{3.475014in}}%
\pgfpathlineto{\pgfqpoint{3.163739in}{3.441878in}}%
\pgfpathlineto{\pgfqpoint{3.168299in}{3.479432in}}%
\pgfpathlineto{\pgfqpoint{3.172858in}{3.399905in}}%
\pgfpathlineto{\pgfqpoint{3.177417in}{3.415369in}}%
\pgfpathlineto{\pgfqpoint{3.181976in}{3.357933in}}%
\pgfpathlineto{\pgfqpoint{3.186536in}{3.346888in}}%
\pgfpathlineto{\pgfqpoint{3.191095in}{3.477223in}}%
\pgfpathlineto{\pgfqpoint{3.195654in}{3.349097in}}%
\pgfpathlineto{\pgfqpoint{3.200213in}{3.368978in}}%
\pgfpathlineto{\pgfqpoint{3.204772in}{3.382233in}}%
\pgfpathlineto{\pgfqpoint{3.209332in}{3.470596in}}%
\pgfpathlineto{\pgfqpoint{3.213891in}{3.417578in}}%
\pgfpathlineto{\pgfqpoint{3.218450in}{3.384442in}}%
\pgfpathlineto{\pgfqpoint{3.223009in}{3.413160in}}%
\pgfpathlineto{\pgfqpoint{3.227569in}{3.406533in}}%
\pgfpathlineto{\pgfqpoint{3.232128in}{3.490477in}}%
\pgfpathlineto{\pgfqpoint{3.236687in}{3.282824in}}%
\pgfpathlineto{\pgfqpoint{3.245806in}{3.472805in}}%
\pgfpathlineto{\pgfqpoint{3.250365in}{3.338051in}}%
\pgfpathlineto{\pgfqpoint{3.254924in}{3.486059in}}%
\pgfpathlineto{\pgfqpoint{3.259483in}{3.441878in}}%
\pgfpathlineto{\pgfqpoint{3.264043in}{3.486059in}}%
\pgfpathlineto{\pgfqpoint{3.268602in}{3.377815in}}%
\pgfpathlineto{\pgfqpoint{3.273161in}{3.448505in}}%
\pgfpathlineto{\pgfqpoint{3.277720in}{3.393278in}}%
\pgfpathlineto{\pgfqpoint{3.282280in}{3.300497in}}%
\pgfpathlineto{\pgfqpoint{3.286839in}{3.567795in}}%
\pgfpathlineto{\pgfqpoint{3.291398in}{3.448505in}}%
\pgfpathlineto{\pgfqpoint{3.295957in}{3.512568in}}%
\pgfpathlineto{\pgfqpoint{3.300517in}{3.393278in}}%
\pgfpathlineto{\pgfqpoint{3.305076in}{3.534659in}}%
\pgfpathlineto{\pgfqpoint{3.309635in}{3.490477in}}%
\pgfpathlineto{\pgfqpoint{3.314194in}{3.397696in}}%
\pgfpathlineto{\pgfqpoint{3.318754in}{3.430832in}}%
\pgfpathlineto{\pgfqpoint{3.323313in}{3.419787in}}%
\pgfpathlineto{\pgfqpoint{3.327872in}{3.324797in}}%
\pgfpathlineto{\pgfqpoint{3.332431in}{3.435250in}}%
\pgfpathlineto{\pgfqpoint{3.336991in}{3.492686in}}%
\pgfpathlineto{\pgfqpoint{3.341550in}{3.572213in}}%
\pgfpathlineto{\pgfqpoint{3.346109in}{3.435250in}}%
\pgfpathlineto{\pgfqpoint{3.350668in}{3.457341in}}%
\pgfpathlineto{\pgfqpoint{3.355228in}{3.446296in}}%
\pgfpathlineto{\pgfqpoint{3.359787in}{3.393278in}}%
\pgfpathlineto{\pgfqpoint{3.364346in}{3.505941in}}%
\pgfpathlineto{\pgfqpoint{3.368905in}{3.475014in}}%
\pgfpathlineto{\pgfqpoint{3.373464in}{3.393278in}}%
\pgfpathlineto{\pgfqpoint{3.378024in}{3.457341in}}%
\pgfpathlineto{\pgfqpoint{3.382583in}{3.410951in}}%
\pgfpathlineto{\pgfqpoint{3.387142in}{3.475014in}}%
\pgfpathlineto{\pgfqpoint{3.391701in}{3.415369in}}%
\pgfpathlineto{\pgfqpoint{3.396261in}{3.384442in}}%
\pgfpathlineto{\pgfqpoint{3.400820in}{3.508150in}}%
\pgfpathlineto{\pgfqpoint{3.405379in}{3.556750in}}%
\pgfpathlineto{\pgfqpoint{3.409938in}{3.503732in}}%
\pgfpathlineto{\pgfqpoint{3.414498in}{3.490477in}}%
\pgfpathlineto{\pgfqpoint{3.419057in}{3.439669in}}%
\pgfpathlineto{\pgfqpoint{3.423616in}{3.428623in}}%
\pgfpathlineto{\pgfqpoint{3.428175in}{3.470596in}}%
\pgfpathlineto{\pgfqpoint{3.432735in}{3.441878in}}%
\pgfpathlineto{\pgfqpoint{3.441853in}{3.585467in}}%
\pgfpathlineto{\pgfqpoint{3.446412in}{3.399905in}}%
\pgfpathlineto{\pgfqpoint{3.450972in}{3.437460in}}%
\pgfpathlineto{\pgfqpoint{3.455531in}{3.512568in}}%
\pgfpathlineto{\pgfqpoint{3.460090in}{3.497105in}}%
\pgfpathlineto{\pgfqpoint{3.464649in}{3.519195in}}%
\pgfpathlineto{\pgfqpoint{3.469209in}{3.572213in}}%
\pgfpathlineto{\pgfqpoint{3.473768in}{3.472805in}}%
\pgfpathlineto{\pgfqpoint{3.478327in}{3.558959in}}%
\pgfpathlineto{\pgfqpoint{3.482886in}{3.466178in}}%
\pgfpathlineto{\pgfqpoint{3.487446in}{3.552331in}}%
\pgfpathlineto{\pgfqpoint{3.492005in}{3.609767in}}%
\pgfpathlineto{\pgfqpoint{3.496564in}{3.448505in}}%
\pgfpathlineto{\pgfqpoint{3.501123in}{3.523613in}}%
\pgfpathlineto{\pgfqpoint{3.505683in}{3.550122in}}%
\pgfpathlineto{\pgfqpoint{3.510242in}{3.534659in}}%
\pgfpathlineto{\pgfqpoint{3.514801in}{3.499314in}}%
\pgfpathlineto{\pgfqpoint{3.519360in}{3.472805in}}%
\pgfpathlineto{\pgfqpoint{3.523919in}{3.536868in}}%
\pgfpathlineto{\pgfqpoint{3.528479in}{3.415369in}}%
\pgfpathlineto{\pgfqpoint{3.533038in}{3.494895in}}%
\pgfpathlineto{\pgfqpoint{3.537597in}{3.497105in}}%
\pgfpathlineto{\pgfqpoint{3.542156in}{3.448505in}}%
\pgfpathlineto{\pgfqpoint{3.551275in}{3.570004in}}%
\pgfpathlineto{\pgfqpoint{3.555834in}{3.596513in}}%
\pgfpathlineto{\pgfqpoint{3.560393in}{3.486059in}}%
\pgfpathlineto{\pgfqpoint{3.564953in}{3.523613in}}%
\pgfpathlineto{\pgfqpoint{3.569512in}{3.446296in}}%
\pgfpathlineto{\pgfqpoint{3.574071in}{3.558959in}}%
\pgfpathlineto{\pgfqpoint{3.578630in}{3.536868in}}%
\pgfpathlineto{\pgfqpoint{3.583190in}{3.572213in}}%
\pgfpathlineto{\pgfqpoint{3.587749in}{3.494895in}}%
\pgfpathlineto{\pgfqpoint{3.592308in}{3.530241in}}%
\pgfpathlineto{\pgfqpoint{3.596867in}{3.523613in}}%
\pgfpathlineto{\pgfqpoint{3.601427in}{3.567795in}}%
\pgfpathlineto{\pgfqpoint{3.605986in}{3.466178in}}%
\pgfpathlineto{\pgfqpoint{3.610545in}{3.461759in}}%
\pgfpathlineto{\pgfqpoint{3.615104in}{3.541286in}}%
\pgfpathlineto{\pgfqpoint{3.619664in}{3.528032in}}%
\pgfpathlineto{\pgfqpoint{3.624223in}{3.600931in}}%
\pgfpathlineto{\pgfqpoint{3.628782in}{3.468387in}}%
\pgfpathlineto{\pgfqpoint{3.633341in}{3.636276in}}%
\pgfpathlineto{\pgfqpoint{3.637901in}{3.631858in}}%
\pgfpathlineto{\pgfqpoint{3.642460in}{3.592095in}}%
\pgfpathlineto{\pgfqpoint{3.647019in}{3.519195in}}%
\pgfpathlineto{\pgfqpoint{3.651578in}{3.614185in}}%
\pgfpathlineto{\pgfqpoint{3.656138in}{3.523613in}}%
\pgfpathlineto{\pgfqpoint{3.660697in}{3.490477in}}%
\pgfpathlineto{\pgfqpoint{3.665256in}{3.528032in}}%
\pgfpathlineto{\pgfqpoint{3.669815in}{3.618604in}}%
\pgfpathlineto{\pgfqpoint{3.674374in}{3.481641in}}%
\pgfpathlineto{\pgfqpoint{3.678934in}{3.598722in}}%
\pgfpathlineto{\pgfqpoint{3.683493in}{3.543495in}}%
\pgfpathlineto{\pgfqpoint{3.688052in}{3.671621in}}%
\pgfpathlineto{\pgfqpoint{3.692611in}{3.552331in}}%
\pgfpathlineto{\pgfqpoint{3.697171in}{3.525822in}}%
\pgfpathlineto{\pgfqpoint{3.701730in}{3.558959in}}%
\pgfpathlineto{\pgfqpoint{3.706289in}{3.523613in}}%
\pgfpathlineto{\pgfqpoint{3.710848in}{3.629649in}}%
\pgfpathlineto{\pgfqpoint{3.715408in}{3.585467in}}%
\pgfpathlineto{\pgfqpoint{3.719967in}{3.466178in}}%
\pgfpathlineto{\pgfqpoint{3.724526in}{3.620813in}}%
\pgfpathlineto{\pgfqpoint{3.729085in}{3.552331in}}%
\pgfpathlineto{\pgfqpoint{3.733645in}{3.543495in}}%
\pgfpathlineto{\pgfqpoint{3.738204in}{3.539077in}}%
\pgfpathlineto{\pgfqpoint{3.742763in}{3.671621in}}%
\pgfpathlineto{\pgfqpoint{3.747322in}{3.514777in}}%
\pgfpathlineto{\pgfqpoint{3.751882in}{3.532450in}}%
\pgfpathlineto{\pgfqpoint{3.756441in}{3.508150in}}%
\pgfpathlineto{\pgfqpoint{3.761000in}{3.539077in}}%
\pgfpathlineto{\pgfqpoint{3.765559in}{3.508150in}}%
\pgfpathlineto{\pgfqpoint{3.770119in}{3.508150in}}%
\pgfpathlineto{\pgfqpoint{3.774678in}{3.521404in}}%
\pgfpathlineto{\pgfqpoint{3.779237in}{3.574422in}}%
\pgfpathlineto{\pgfqpoint{3.783796in}{3.556750in}}%
\pgfpathlineto{\pgfqpoint{3.788356in}{3.494895in}}%
\pgfpathlineto{\pgfqpoint{3.792915in}{3.508150in}}%
\pgfpathlineto{\pgfqpoint{3.797474in}{3.457341in}}%
\pgfpathlineto{\pgfqpoint{3.802033in}{3.669412in}}%
\pgfpathlineto{\pgfqpoint{3.806593in}{3.616394in}}%
\pgfpathlineto{\pgfqpoint{3.811152in}{3.638485in}}%
\pgfpathlineto{\pgfqpoint{3.815711in}{3.669412in}}%
\pgfpathlineto{\pgfqpoint{3.820270in}{3.523613in}}%
\pgfpathlineto{\pgfqpoint{3.824829in}{3.609767in}}%
\pgfpathlineto{\pgfqpoint{3.829389in}{3.634067in}}%
\pgfpathlineto{\pgfqpoint{3.833948in}{3.676039in}}%
\pgfpathlineto{\pgfqpoint{3.838507in}{3.594304in}}%
\pgfpathlineto{\pgfqpoint{3.843066in}{3.709176in}}%
\pgfpathlineto{\pgfqpoint{3.847626in}{3.541286in}}%
\pgfpathlineto{\pgfqpoint{3.852185in}{3.578840in}}%
\pgfpathlineto{\pgfqpoint{3.856744in}{3.570004in}}%
\pgfpathlineto{\pgfqpoint{3.861303in}{3.629649in}}%
\pgfpathlineto{\pgfqpoint{3.865863in}{3.558959in}}%
\pgfpathlineto{\pgfqpoint{3.870422in}{3.682667in}}%
\pgfpathlineto{\pgfqpoint{3.874981in}{3.512568in}}%
\pgfpathlineto{\pgfqpoint{3.879540in}{3.475014in}}%
\pgfpathlineto{\pgfqpoint{3.884100in}{3.552331in}}%
\pgfpathlineto{\pgfqpoint{3.888659in}{3.565586in}}%
\pgfpathlineto{\pgfqpoint{3.893218in}{3.510359in}}%
\pgfpathlineto{\pgfqpoint{3.897777in}{3.488268in}}%
\pgfpathlineto{\pgfqpoint{3.902337in}{3.616394in}}%
\pgfpathlineto{\pgfqpoint{3.906896in}{3.620813in}}%
\pgfpathlineto{\pgfqpoint{3.911455in}{3.642903in}}%
\pgfpathlineto{\pgfqpoint{3.920574in}{3.525822in}}%
\pgfpathlineto{\pgfqpoint{3.925133in}{3.581049in}}%
\pgfpathlineto{\pgfqpoint{3.929692in}{3.673830in}}%
\pgfpathlineto{\pgfqpoint{3.934251in}{3.603140in}}%
\pgfpathlineto{\pgfqpoint{3.938811in}{3.614185in}}%
\pgfpathlineto{\pgfqpoint{3.943370in}{3.603140in}}%
\pgfpathlineto{\pgfqpoint{3.947929in}{3.653949in}}%
\pgfpathlineto{\pgfqpoint{3.952488in}{3.636276in}}%
\pgfpathlineto{\pgfqpoint{3.957048in}{3.541286in}}%
\pgfpathlineto{\pgfqpoint{3.961607in}{3.640694in}}%
\pgfpathlineto{\pgfqpoint{3.966166in}{3.651740in}}%
\pgfpathlineto{\pgfqpoint{3.970725in}{3.616394in}}%
\pgfpathlineto{\pgfqpoint{3.975284in}{3.658367in}}%
\pgfpathlineto{\pgfqpoint{3.984403in}{3.523613in}}%
\pgfpathlineto{\pgfqpoint{3.988962in}{3.576631in}}%
\pgfpathlineto{\pgfqpoint{3.993521in}{3.682667in}}%
\pgfpathlineto{\pgfqpoint{3.998081in}{3.647321in}}%
\pgfpathlineto{\pgfqpoint{4.002640in}{3.664994in}}%
\pgfpathlineto{\pgfqpoint{4.007199in}{3.636276in}}%
\pgfpathlineto{\pgfqpoint{4.011758in}{3.583258in}}%
\pgfpathlineto{\pgfqpoint{4.016318in}{3.625231in}}%
\pgfpathlineto{\pgfqpoint{4.020877in}{3.585467in}}%
\pgfpathlineto{\pgfqpoint{4.025436in}{3.556750in}}%
\pgfpathlineto{\pgfqpoint{4.029995in}{3.587677in}}%
\pgfpathlineto{\pgfqpoint{4.034555in}{3.600931in}}%
\pgfpathlineto{\pgfqpoint{4.039114in}{3.563377in}}%
\pgfpathlineto{\pgfqpoint{4.043673in}{3.704757in}}%
\pgfpathlineto{\pgfqpoint{4.048232in}{3.691503in}}%
\pgfpathlineto{\pgfqpoint{4.052792in}{3.691503in}}%
\pgfpathlineto{\pgfqpoint{4.057351in}{3.512568in}}%
\pgfpathlineto{\pgfqpoint{4.061910in}{3.698130in}}%
\pgfpathlineto{\pgfqpoint{4.066469in}{3.762193in}}%
\pgfpathlineto{\pgfqpoint{4.075588in}{3.552331in}}%
\pgfpathlineto{\pgfqpoint{4.080147in}{3.607558in}}%
\pgfpathlineto{\pgfqpoint{4.089266in}{3.764402in}}%
\pgfpathlineto{\pgfqpoint{4.093825in}{3.680458in}}%
\pgfpathlineto{\pgfqpoint{4.098384in}{3.574422in}}%
\pgfpathlineto{\pgfqpoint{4.102943in}{3.764402in}}%
\pgfpathlineto{\pgfqpoint{4.107503in}{3.649531in}}%
\pgfpathlineto{\pgfqpoint{4.112062in}{3.718012in}}%
\pgfpathlineto{\pgfqpoint{4.116621in}{3.581049in}}%
\pgfpathlineto{\pgfqpoint{4.121180in}{3.700339in}}%
\pgfpathlineto{\pgfqpoint{4.125739in}{3.565586in}}%
\pgfpathlineto{\pgfqpoint{4.130299in}{3.702548in}}%
\pgfpathlineto{\pgfqpoint{4.134858in}{3.748939in}}%
\pgfpathlineto{\pgfqpoint{4.139417in}{3.561168in}}%
\pgfpathlineto{\pgfqpoint{4.143976in}{3.587677in}}%
\pgfpathlineto{\pgfqpoint{4.148536in}{3.660576in}}%
\pgfpathlineto{\pgfqpoint{4.153095in}{3.645112in}}%
\pgfpathlineto{\pgfqpoint{4.157654in}{3.693712in}}%
\pgfpathlineto{\pgfqpoint{4.162213in}{3.711385in}}%
\pgfpathlineto{\pgfqpoint{4.166773in}{3.700339in}}%
\pgfpathlineto{\pgfqpoint{4.171332in}{3.762193in}}%
\pgfpathlineto{\pgfqpoint{4.175891in}{3.625231in}}%
\pgfpathlineto{\pgfqpoint{4.185010in}{3.698130in}}%
\pgfpathlineto{\pgfqpoint{4.189569in}{3.629649in}}%
\pgfpathlineto{\pgfqpoint{4.194128in}{3.676039in}}%
\pgfpathlineto{\pgfqpoint{4.198687in}{3.671621in}}%
\pgfpathlineto{\pgfqpoint{4.203247in}{3.684876in}}%
\pgfpathlineto{\pgfqpoint{4.207806in}{3.583258in}}%
\pgfpathlineto{\pgfqpoint{4.212365in}{3.678249in}}%
\pgfpathlineto{\pgfqpoint{4.216924in}{3.631858in}}%
\pgfpathlineto{\pgfqpoint{4.221484in}{3.709176in}}%
\pgfpathlineto{\pgfqpoint{4.226043in}{3.709176in}}%
\pgfpathlineto{\pgfqpoint{4.230602in}{3.689294in}}%
\pgfpathlineto{\pgfqpoint{4.235161in}{3.698130in}}%
\pgfpathlineto{\pgfqpoint{4.239721in}{3.676039in}}%
\pgfpathlineto{\pgfqpoint{4.244280in}{3.704757in}}%
\pgfpathlineto{\pgfqpoint{4.248839in}{3.742312in}}%
\pgfpathlineto{\pgfqpoint{4.253398in}{3.687085in}}%
\pgfpathlineto{\pgfqpoint{4.257958in}{3.667203in}}%
\pgfpathlineto{\pgfqpoint{4.262517in}{3.788702in}}%
\pgfpathlineto{\pgfqpoint{4.267076in}{3.662785in}}%
\pgfpathlineto{\pgfqpoint{4.271635in}{3.673830in}}%
\pgfpathlineto{\pgfqpoint{4.276195in}{3.583258in}}%
\pgfpathlineto{\pgfqpoint{4.280754in}{3.718012in}}%
\pgfpathlineto{\pgfqpoint{4.285313in}{3.753357in}}%
\pgfpathlineto{\pgfqpoint{4.289872in}{3.634067in}}%
\pgfpathlineto{\pgfqpoint{4.294431in}{3.731266in}}%
\pgfpathlineto{\pgfqpoint{4.298991in}{3.687085in}}%
\pgfpathlineto{\pgfqpoint{4.303550in}{3.682667in}}%
\pgfpathlineto{\pgfqpoint{4.308109in}{3.682667in}}%
\pgfpathlineto{\pgfqpoint{4.312668in}{3.689294in}}%
\pgfpathlineto{\pgfqpoint{4.317228in}{3.731266in}}%
\pgfpathlineto{\pgfqpoint{4.321787in}{3.673830in}}%
\pgfpathlineto{\pgfqpoint{4.326346in}{3.700339in}}%
\pgfpathlineto{\pgfqpoint{4.330905in}{3.751148in}}%
\pgfpathlineto{\pgfqpoint{4.335465in}{3.715803in}}%
\pgfpathlineto{\pgfqpoint{4.340024in}{3.806375in}}%
\pgfpathlineto{\pgfqpoint{4.344583in}{3.678249in}}%
\pgfpathlineto{\pgfqpoint{4.349142in}{3.669412in}}%
\pgfpathlineto{\pgfqpoint{4.353702in}{3.744521in}}%
\pgfpathlineto{\pgfqpoint{4.358261in}{3.649531in}}%
\pgfpathlineto{\pgfqpoint{4.362820in}{3.682667in}}%
\pgfpathlineto{\pgfqpoint{4.371939in}{3.731266in}}%
\pgfpathlineto{\pgfqpoint{4.376498in}{3.759984in}}%
\pgfpathlineto{\pgfqpoint{4.381057in}{3.700339in}}%
\pgfpathlineto{\pgfqpoint{4.385616in}{3.777657in}}%
\pgfpathlineto{\pgfqpoint{4.390176in}{3.799748in}}%
\pgfpathlineto{\pgfqpoint{4.394735in}{3.598722in}}%
\pgfpathlineto{\pgfqpoint{4.399294in}{3.715803in}}%
\pgfpathlineto{\pgfqpoint{4.403853in}{3.777657in}}%
\pgfpathlineto{\pgfqpoint{4.408413in}{3.790911in}}%
\pgfpathlineto{\pgfqpoint{4.412972in}{3.773239in}}%
\pgfpathlineto{\pgfqpoint{4.417531in}{3.815211in}}%
\pgfpathlineto{\pgfqpoint{4.422090in}{3.658367in}}%
\pgfpathlineto{\pgfqpoint{4.426650in}{3.764402in}}%
\pgfpathlineto{\pgfqpoint{4.431209in}{3.625231in}}%
\pgfpathlineto{\pgfqpoint{4.435768in}{3.673830in}}%
\pgfpathlineto{\pgfqpoint{4.440327in}{3.799748in}}%
\pgfpathlineto{\pgfqpoint{4.444886in}{3.819629in}}%
\pgfpathlineto{\pgfqpoint{4.449446in}{3.786493in}}%
\pgfpathlineto{\pgfqpoint{4.454005in}{3.713594in}}%
\pgfpathlineto{\pgfqpoint{4.458564in}{3.676039in}}%
\pgfpathlineto{\pgfqpoint{4.463123in}{3.759984in}}%
\pgfpathlineto{\pgfqpoint{4.467683in}{3.656158in}}%
\pgfpathlineto{\pgfqpoint{4.472242in}{3.748939in}}%
\pgfpathlineto{\pgfqpoint{4.476801in}{3.795329in}}%
\pgfpathlineto{\pgfqpoint{4.481360in}{3.746730in}}%
\pgfpathlineto{\pgfqpoint{4.485920in}{3.592095in}}%
\pgfpathlineto{\pgfqpoint{4.490479in}{3.627440in}}%
\pgfpathlineto{\pgfqpoint{4.495038in}{3.759984in}}%
\pgfpathlineto{\pgfqpoint{4.499597in}{3.775448in}}%
\pgfpathlineto{\pgfqpoint{4.504157in}{3.735684in}}%
\pgfpathlineto{\pgfqpoint{4.508716in}{3.799748in}}%
\pgfpathlineto{\pgfqpoint{4.513275in}{3.534659in}}%
\pgfpathlineto{\pgfqpoint{4.517834in}{3.755566in}}%
\pgfpathlineto{\pgfqpoint{4.522394in}{3.733475in}}%
\pgfpathlineto{\pgfqpoint{4.526953in}{3.680458in}}%
\pgfpathlineto{\pgfqpoint{4.531512in}{3.872647in}}%
\pgfpathlineto{\pgfqpoint{4.536071in}{3.806375in}}%
\pgfpathlineto{\pgfqpoint{4.540631in}{3.667203in}}%
\pgfpathlineto{\pgfqpoint{4.545190in}{3.726848in}}%
\pgfpathlineto{\pgfqpoint{4.549749in}{3.702548in}}%
\pgfpathlineto{\pgfqpoint{4.554308in}{3.720221in}}%
\pgfpathlineto{\pgfqpoint{4.558868in}{3.753357in}}%
\pgfpathlineto{\pgfqpoint{4.563427in}{3.711385in}}%
\pgfpathlineto{\pgfqpoint{4.567986in}{3.698130in}}%
\pgfpathlineto{\pgfqpoint{4.572545in}{3.733475in}}%
\pgfpathlineto{\pgfqpoint{4.577105in}{3.817420in}}%
\pgfpathlineto{\pgfqpoint{4.581664in}{3.830675in}}%
\pgfpathlineto{\pgfqpoint{4.586223in}{3.773239in}}%
\pgfpathlineto{\pgfqpoint{4.590782in}{3.695921in}}%
\pgfpathlineto{\pgfqpoint{4.595341in}{3.861602in}}%
\pgfpathlineto{\pgfqpoint{4.599901in}{3.806375in}}%
\pgfpathlineto{\pgfqpoint{4.604460in}{3.837302in}}%
\pgfpathlineto{\pgfqpoint{4.609019in}{3.729057in}}%
\pgfpathlineto{\pgfqpoint{4.613578in}{3.658367in}}%
\pgfpathlineto{\pgfqpoint{4.618138in}{3.737893in}}%
\pgfpathlineto{\pgfqpoint{4.622697in}{3.658367in}}%
\pgfpathlineto{\pgfqpoint{4.627256in}{3.808584in}}%
\pgfpathlineto{\pgfqpoint{4.631815in}{3.790911in}}%
\pgfpathlineto{\pgfqpoint{4.636375in}{3.737893in}}%
\pgfpathlineto{\pgfqpoint{4.640934in}{3.720221in}}%
\pgfpathlineto{\pgfqpoint{4.645493in}{3.861602in}}%
\pgfpathlineto{\pgfqpoint{4.650052in}{3.870438in}}%
\pgfpathlineto{\pgfqpoint{4.654612in}{3.775448in}}%
\pgfpathlineto{\pgfqpoint{4.659171in}{3.790911in}}%
\pgfpathlineto{\pgfqpoint{4.663730in}{3.974264in}}%
\pgfpathlineto{\pgfqpoint{4.668289in}{3.592095in}}%
\pgfpathlineto{\pgfqpoint{4.672849in}{3.788702in}}%
\pgfpathlineto{\pgfqpoint{4.677408in}{3.695921in}}%
\pgfpathlineto{\pgfqpoint{4.681967in}{3.753357in}}%
\pgfpathlineto{\pgfqpoint{4.686526in}{3.682667in}}%
\pgfpathlineto{\pgfqpoint{4.691086in}{3.764402in}}%
\pgfpathlineto{\pgfqpoint{4.695645in}{3.883692in}}%
\pgfpathlineto{\pgfqpoint{4.700204in}{3.691503in}}%
\pgfpathlineto{\pgfqpoint{4.709323in}{3.762193in}}%
\pgfpathlineto{\pgfqpoint{4.713882in}{3.773239in}}%
\pgfpathlineto{\pgfqpoint{4.718441in}{3.766611in}}%
\pgfpathlineto{\pgfqpoint{4.723000in}{3.852765in}}%
\pgfpathlineto{\pgfqpoint{4.727560in}{3.859393in}}%
\pgfpathlineto{\pgfqpoint{4.732119in}{3.744521in}}%
\pgfpathlineto{\pgfqpoint{4.736678in}{3.733475in}}%
\pgfpathlineto{\pgfqpoint{4.741237in}{3.764402in}}%
\pgfpathlineto{\pgfqpoint{4.745796in}{3.923456in}}%
\pgfpathlineto{\pgfqpoint{4.750356in}{3.753357in}}%
\pgfpathlineto{\pgfqpoint{4.759474in}{3.819629in}}%
\pgfpathlineto{\pgfqpoint{4.764033in}{3.817420in}}%
\pgfpathlineto{\pgfqpoint{4.768593in}{3.793120in}}%
\pgfpathlineto{\pgfqpoint{4.773152in}{3.892529in}}%
\pgfpathlineto{\pgfqpoint{4.777711in}{3.731266in}}%
\pgfpathlineto{\pgfqpoint{4.782270in}{3.799748in}}%
\pgfpathlineto{\pgfqpoint{4.786830in}{3.819629in}}%
\pgfpathlineto{\pgfqpoint{4.791389in}{3.824047in}}%
\pgfpathlineto{\pgfqpoint{4.800507in}{3.742312in}}%
\pgfpathlineto{\pgfqpoint{4.805067in}{3.899156in}}%
\pgfpathlineto{\pgfqpoint{4.809626in}{3.704757in}}%
\pgfpathlineto{\pgfqpoint{4.814185in}{3.892529in}}%
\pgfpathlineto{\pgfqpoint{4.818744in}{3.782075in}}%
\pgfpathlineto{\pgfqpoint{4.823304in}{3.839511in}}%
\pgfpathlineto{\pgfqpoint{4.827863in}{3.691503in}}%
\pgfpathlineto{\pgfqpoint{4.832422in}{3.879274in}}%
\pgfpathlineto{\pgfqpoint{4.836981in}{3.753357in}}%
\pgfpathlineto{\pgfqpoint{4.841541in}{3.788702in}}%
\pgfpathlineto{\pgfqpoint{4.846100in}{3.740103in}}%
\pgfpathlineto{\pgfqpoint{4.850659in}{3.826256in}}%
\pgfpathlineto{\pgfqpoint{4.855218in}{3.771030in}}%
\pgfpathlineto{\pgfqpoint{4.859778in}{3.759984in}}%
\pgfpathlineto{\pgfqpoint{4.864337in}{3.815211in}}%
\pgfpathlineto{\pgfqpoint{4.868896in}{3.658367in}}%
\pgfpathlineto{\pgfqpoint{4.873455in}{3.779866in}}%
\pgfpathlineto{\pgfqpoint{4.878015in}{3.733475in}}%
\pgfpathlineto{\pgfqpoint{4.882574in}{3.824047in}}%
\pgfpathlineto{\pgfqpoint{4.887133in}{3.868229in}}%
\pgfpathlineto{\pgfqpoint{4.891692in}{3.753357in}}%
\pgfpathlineto{\pgfqpoint{4.896251in}{3.888110in}}%
\pgfpathlineto{\pgfqpoint{4.900811in}{3.863811in}}%
\pgfpathlineto{\pgfqpoint{4.905370in}{3.764402in}}%
\pgfpathlineto{\pgfqpoint{4.909929in}{3.801957in}}%
\pgfpathlineto{\pgfqpoint{4.914488in}{3.782075in}}%
\pgfpathlineto{\pgfqpoint{4.919048in}{3.795329in}}%
\pgfpathlineto{\pgfqpoint{4.923607in}{3.726848in}}%
\pgfpathlineto{\pgfqpoint{4.928166in}{3.826256in}}%
\pgfpathlineto{\pgfqpoint{4.932725in}{3.744521in}}%
\pgfpathlineto{\pgfqpoint{4.941844in}{3.879274in}}%
\pgfpathlineto{\pgfqpoint{4.946403in}{3.817420in}}%
\pgfpathlineto{\pgfqpoint{4.950962in}{3.797538in}}%
\pgfpathlineto{\pgfqpoint{4.955522in}{3.883692in}}%
\pgfpathlineto{\pgfqpoint{4.960081in}{3.821838in}}%
\pgfpathlineto{\pgfqpoint{4.964640in}{3.799748in}}%
\pgfpathlineto{\pgfqpoint{4.969199in}{3.866020in}}%
\pgfpathlineto{\pgfqpoint{4.973759in}{3.839511in}}%
\pgfpathlineto{\pgfqpoint{4.978318in}{3.890320in}}%
\pgfpathlineto{\pgfqpoint{4.982877in}{3.857183in}}%
\pgfpathlineto{\pgfqpoint{4.987436in}{3.835093in}}%
\pgfpathlineto{\pgfqpoint{4.991996in}{3.793120in}}%
\pgfpathlineto{\pgfqpoint{4.996555in}{3.797538in}}%
\pgfpathlineto{\pgfqpoint{5.001114in}{3.857183in}}%
\pgfpathlineto{\pgfqpoint{5.005673in}{3.797538in}}%
\pgfpathlineto{\pgfqpoint{5.010233in}{3.890320in}}%
\pgfpathlineto{\pgfqpoint{5.014792in}{3.925665in}}%
\pgfpathlineto{\pgfqpoint{5.019351in}{3.808584in}}%
\pgfpathlineto{\pgfqpoint{5.023910in}{3.777657in}}%
\pgfpathlineto{\pgfqpoint{5.028470in}{4.005191in}}%
\pgfpathlineto{\pgfqpoint{5.033029in}{3.764402in}}%
\pgfpathlineto{\pgfqpoint{5.042147in}{3.804166in}}%
\pgfpathlineto{\pgfqpoint{5.046706in}{3.790911in}}%
\pgfpathlineto{\pgfqpoint{5.051266in}{3.813002in}}%
\pgfpathlineto{\pgfqpoint{5.060384in}{3.943337in}}%
\pgfpathlineto{\pgfqpoint{5.069503in}{3.866020in}}%
\pgfpathlineto{\pgfqpoint{5.074062in}{3.799748in}}%
\pgfpathlineto{\pgfqpoint{5.078621in}{3.813002in}}%
\pgfpathlineto{\pgfqpoint{5.083180in}{3.804166in}}%
\pgfpathlineto{\pgfqpoint{5.087740in}{3.925665in}}%
\pgfpathlineto{\pgfqpoint{5.092299in}{3.757775in}}%
\pgfpathlineto{\pgfqpoint{5.096858in}{3.839511in}}%
\pgfpathlineto{\pgfqpoint{5.101417in}{3.846138in}}%
\pgfpathlineto{\pgfqpoint{5.105977in}{3.729057in}}%
\pgfpathlineto{\pgfqpoint{5.110536in}{3.852765in}}%
\pgfpathlineto{\pgfqpoint{5.115095in}{3.890320in}}%
\pgfpathlineto{\pgfqpoint{5.119654in}{3.892529in}}%
\pgfpathlineto{\pgfqpoint{5.124214in}{3.881483in}}%
\pgfpathlineto{\pgfqpoint{5.128773in}{3.899156in}}%
\pgfpathlineto{\pgfqpoint{5.133332in}{3.854974in}}%
\pgfpathlineto{\pgfqpoint{5.142451in}{3.912410in}}%
\pgfpathlineto{\pgfqpoint{5.147010in}{3.885901in}}%
\pgfpathlineto{\pgfqpoint{5.156128in}{4.000773in}}%
\pgfpathlineto{\pgfqpoint{5.160688in}{3.819629in}}%
\pgfpathlineto{\pgfqpoint{5.165247in}{3.854974in}}%
\pgfpathlineto{\pgfqpoint{5.174365in}{3.945546in}}%
\pgfpathlineto{\pgfqpoint{5.178925in}{3.841720in}}%
\pgfpathlineto{\pgfqpoint{5.183484in}{3.894738in}}%
\pgfpathlineto{\pgfqpoint{5.188043in}{3.828465in}}%
\pgfpathlineto{\pgfqpoint{5.192602in}{3.881483in}}%
\pgfpathlineto{\pgfqpoint{5.197162in}{3.753357in}}%
\pgfpathlineto{\pgfqpoint{5.201721in}{3.826256in}}%
\pgfpathlineto{\pgfqpoint{5.206280in}{3.874856in}}%
\pgfpathlineto{\pgfqpoint{5.210839in}{3.837302in}}%
\pgfpathlineto{\pgfqpoint{5.215398in}{3.861602in}}%
\pgfpathlineto{\pgfqpoint{5.219958in}{4.002982in}}%
\pgfpathlineto{\pgfqpoint{5.224517in}{3.901365in}}%
\pgfpathlineto{\pgfqpoint{5.229076in}{3.927874in}}%
\pgfpathlineto{\pgfqpoint{5.233635in}{3.771030in}}%
\pgfpathlineto{\pgfqpoint{5.238195in}{3.952174in}}%
\pgfpathlineto{\pgfqpoint{5.242754in}{3.932292in}}%
\pgfpathlineto{\pgfqpoint{5.247313in}{3.839511in}}%
\pgfpathlineto{\pgfqpoint{5.251872in}{3.905783in}}%
\pgfpathlineto{\pgfqpoint{5.256432in}{3.870438in}}%
\pgfpathlineto{\pgfqpoint{5.260991in}{3.896947in}}%
\pgfpathlineto{\pgfqpoint{5.265550in}{3.863811in}}%
\pgfpathlineto{\pgfqpoint{5.270109in}{3.965428in}}%
\pgfpathlineto{\pgfqpoint{5.274669in}{3.936710in}}%
\pgfpathlineto{\pgfqpoint{5.279228in}{3.711385in}}%
\pgfpathlineto{\pgfqpoint{5.283787in}{3.843929in}}%
\pgfpathlineto{\pgfqpoint{5.288346in}{3.938919in}}%
\pgfpathlineto{\pgfqpoint{5.292906in}{3.870438in}}%
\pgfpathlineto{\pgfqpoint{5.297465in}{3.905783in}}%
\pgfpathlineto{\pgfqpoint{5.306583in}{3.881483in}}%
\pgfpathlineto{\pgfqpoint{5.311143in}{3.905783in}}%
\pgfpathlineto{\pgfqpoint{5.315702in}{3.967637in}}%
\pgfpathlineto{\pgfqpoint{5.320261in}{3.941128in}}%
\pgfpathlineto{\pgfqpoint{5.324820in}{3.877065in}}%
\pgfpathlineto{\pgfqpoint{5.333939in}{3.846138in}}%
\pgfpathlineto{\pgfqpoint{5.338498in}{3.943337in}}%
\pgfpathlineto{\pgfqpoint{5.343057in}{3.872647in}}%
\pgfpathlineto{\pgfqpoint{5.347617in}{4.007400in}}%
\pgfpathlineto{\pgfqpoint{5.352176in}{3.994146in}}%
\pgfpathlineto{\pgfqpoint{5.356735in}{3.782075in}}%
\pgfpathlineto{\pgfqpoint{5.361294in}{3.956592in}}%
\pgfpathlineto{\pgfqpoint{5.365853in}{4.029491in}}%
\pgfpathlineto{\pgfqpoint{5.374972in}{3.888110in}}%
\pgfpathlineto{\pgfqpoint{5.379531in}{4.025073in}}%
\pgfpathlineto{\pgfqpoint{5.384090in}{3.892529in}}%
\pgfpathlineto{\pgfqpoint{5.388650in}{4.011819in}}%
\pgfpathlineto{\pgfqpoint{5.393209in}{3.910201in}}%
\pgfpathlineto{\pgfqpoint{5.397768in}{3.782075in}}%
\pgfpathlineto{\pgfqpoint{5.402327in}{3.991937in}}%
\pgfpathlineto{\pgfqpoint{5.406887in}{3.866020in}}%
\pgfpathlineto{\pgfqpoint{5.411446in}{3.919037in}}%
\pgfpathlineto{\pgfqpoint{5.416005in}{3.872647in}}%
\pgfpathlineto{\pgfqpoint{5.420564in}{3.879274in}}%
\pgfpathlineto{\pgfqpoint{5.425124in}{3.883692in}}%
\pgfpathlineto{\pgfqpoint{5.429683in}{3.881483in}}%
\pgfpathlineto{\pgfqpoint{5.434242in}{3.954383in}}%
\pgfpathlineto{\pgfqpoint{5.438801in}{3.879274in}}%
\pgfpathlineto{\pgfqpoint{5.443361in}{3.972055in}}%
\pgfpathlineto{\pgfqpoint{5.447920in}{3.947755in}}%
\pgfpathlineto{\pgfqpoint{5.452479in}{3.961010in}}%
\pgfpathlineto{\pgfqpoint{5.457038in}{4.056000in}}%
\pgfpathlineto{\pgfqpoint{5.461598in}{3.813002in}}%
\pgfpathlineto{\pgfqpoint{5.466157in}{3.866020in}}%
\pgfpathlineto{\pgfqpoint{5.470716in}{3.890320in}}%
\pgfpathlineto{\pgfqpoint{5.475275in}{4.033909in}}%
\pgfpathlineto{\pgfqpoint{5.479835in}{3.901365in}}%
\pgfpathlineto{\pgfqpoint{5.484394in}{3.837302in}}%
\pgfpathlineto{\pgfqpoint{5.488953in}{3.914619in}}%
\pgfpathlineto{\pgfqpoint{5.493512in}{3.881483in}}%
\pgfpathlineto{\pgfqpoint{5.498072in}{3.742312in}}%
\pgfpathlineto{\pgfqpoint{5.502631in}{3.859393in}}%
\pgfpathlineto{\pgfqpoint{5.507190in}{3.879274in}}%
\pgfpathlineto{\pgfqpoint{5.511749in}{4.042746in}}%
\pgfpathlineto{\pgfqpoint{5.516308in}{3.852765in}}%
\pgfpathlineto{\pgfqpoint{5.520868in}{3.905783in}}%
\pgfpathlineto{\pgfqpoint{5.525427in}{3.848347in}}%
\pgfpathlineto{\pgfqpoint{5.529986in}{3.879274in}}%
\pgfpathlineto{\pgfqpoint{5.534545in}{3.963219in}}%
\pgfpathlineto{\pgfqpoint{5.534545in}{3.963219in}}%
\pgfusepath{stroke}%
\end{pgfscope}%
\begin{pgfscope}%
\pgfpathrectangle{\pgfqpoint{0.800000in}{0.528000in}}{\pgfqpoint{4.960000in}{3.696000in}} %
\pgfusepath{clip}%
\pgfsetrectcap%
\pgfsetroundjoin%
\pgfsetlinewidth{1.505625pt}%
\definecolor{currentstroke}{rgb}{1.000000,0.498039,0.054902}%
\pgfsetstrokecolor{currentstroke}%
\pgfsetdash{}{0pt}%
\pgfpathmoveto{\pgfqpoint{1.025455in}{0.813252in}}%
\pgfpathlineto{\pgfqpoint{1.030014in}{0.876616in}}%
\pgfpathlineto{\pgfqpoint{1.034573in}{0.934464in}}%
\pgfpathlineto{\pgfqpoint{1.048251in}{1.082815in}}%
\pgfpathlineto{\pgfqpoint{1.061928in}{1.204027in}}%
\pgfpathlineto{\pgfqpoint{1.075606in}{1.306510in}}%
\pgfpathlineto{\pgfqpoint{1.089284in}{1.395285in}}%
\pgfpathlineto{\pgfqpoint{1.107521in}{1.497768in}}%
\pgfpathlineto{\pgfqpoint{1.125758in}{1.586543in}}%
\pgfpathlineto{\pgfqpoint{1.143995in}{1.664848in}}%
\pgfpathlineto{\pgfqpoint{1.162232in}{1.734894in}}%
\pgfpathlineto{\pgfqpoint{1.185028in}{1.813199in}}%
\pgfpathlineto{\pgfqpoint{1.207824in}{1.883245in}}%
\pgfpathlineto{\pgfqpoint{1.230620in}{1.946610in}}%
\pgfpathlineto{\pgfqpoint{1.257976in}{2.015446in}}%
\pgfpathlineto{\pgfqpoint{1.285331in}{2.077819in}}%
\pgfpathlineto{\pgfqpoint{1.312687in}{2.134839in}}%
\pgfpathlineto{\pgfqpoint{1.344602in}{2.195715in}}%
\pgfpathlineto{\pgfqpoint{1.376516in}{2.251482in}}%
\pgfpathlineto{\pgfqpoint{1.408431in}{2.302930in}}%
\pgfpathlineto{\pgfqpoint{1.444905in}{2.357232in}}%
\pgfpathlineto{\pgfqpoint{1.481379in}{2.407431in}}%
\pgfpathlineto{\pgfqpoint{1.517853in}{2.454105in}}%
\pgfpathlineto{\pgfqpoint{1.558886in}{2.502971in}}%
\pgfpathlineto{\pgfqpoint{1.599919in}{2.548490in}}%
\pgfpathlineto{\pgfqpoint{1.645512in}{2.595660in}}%
\pgfpathlineto{\pgfqpoint{1.691104in}{2.639705in}}%
\pgfpathlineto{\pgfqpoint{1.741256in}{2.685004in}}%
\pgfpathlineto{\pgfqpoint{1.791407in}{2.727413in}}%
\pgfpathlineto{\pgfqpoint{1.846118in}{2.770787in}}%
\pgfpathlineto{\pgfqpoint{1.905388in}{2.814786in}}%
\pgfpathlineto{\pgfqpoint{1.964659in}{2.856053in}}%
\pgfpathlineto{\pgfqpoint{2.028488in}{2.897805in}}%
\pgfpathlineto{\pgfqpoint{2.096877in}{2.939808in}}%
\pgfpathlineto{\pgfqpoint{2.169824in}{2.981866in}}%
\pgfpathlineto{\pgfqpoint{2.247332in}{3.023817in}}%
\pgfpathlineto{\pgfqpoint{2.324839in}{3.063277in}}%
\pgfpathlineto{\pgfqpoint{2.406905in}{3.102653in}}%
\pgfpathlineto{\pgfqpoint{2.493531in}{3.141833in}}%
\pgfpathlineto{\pgfqpoint{2.589275in}{3.182608in}}%
\pgfpathlineto{\pgfqpoint{2.689578in}{3.222802in}}%
\pgfpathlineto{\pgfqpoint{2.794441in}{3.262377in}}%
\pgfpathlineto{\pgfqpoint{2.903862in}{3.301304in}}%
\pgfpathlineto{\pgfqpoint{3.022403in}{3.341053in}}%
\pgfpathlineto{\pgfqpoint{3.145502in}{3.379959in}}%
\pgfpathlineto{\pgfqpoint{3.277720in}{3.419358in}}%
\pgfpathlineto{\pgfqpoint{3.414498in}{3.457798in}}%
\pgfpathlineto{\pgfqpoint{3.560393in}{3.496489in}}%
\pgfpathlineto{\pgfqpoint{3.715408in}{3.535272in}}%
\pgfpathlineto{\pgfqpoint{3.879540in}{3.574011in}}%
\pgfpathlineto{\pgfqpoint{4.052792in}{3.612592in}}%
\pgfpathlineto{\pgfqpoint{4.235161in}{3.650921in}}%
\pgfpathlineto{\pgfqpoint{4.431209in}{3.689800in}}%
\pgfpathlineto{\pgfqpoint{4.636375in}{3.728190in}}%
\pgfpathlineto{\pgfqpoint{4.855218in}{3.766835in}}%
\pgfpathlineto{\pgfqpoint{5.083180in}{3.804835in}}%
\pgfpathlineto{\pgfqpoint{5.324820in}{3.842877in}}%
\pgfpathlineto{\pgfqpoint{5.534545in}{3.874216in}}%
\pgfpathlineto{\pgfqpoint{5.534545in}{3.874216in}}%
\pgfusepath{stroke}%
\end{pgfscope}%
\begin{pgfscope}%
\pgfsetrectcap%
\pgfsetmiterjoin%
\pgfsetlinewidth{0.803000pt}%
\definecolor{currentstroke}{rgb}{0.000000,0.000000,0.000000}%
\pgfsetstrokecolor{currentstroke}%
\pgfsetdash{}{0pt}%
\pgfpathmoveto{\pgfqpoint{0.800000in}{0.528000in}}%
\pgfpathlineto{\pgfqpoint{0.800000in}{4.224000in}}%
\pgfusepath{stroke}%
\end{pgfscope}%
\begin{pgfscope}%
\pgfsetrectcap%
\pgfsetmiterjoin%
\pgfsetlinewidth{0.803000pt}%
\definecolor{currentstroke}{rgb}{0.000000,0.000000,0.000000}%
\pgfsetstrokecolor{currentstroke}%
\pgfsetdash{}{0pt}%
\pgfpathmoveto{\pgfqpoint{5.760000in}{0.528000in}}%
\pgfpathlineto{\pgfqpoint{5.760000in}{4.224000in}}%
\pgfusepath{stroke}%
\end{pgfscope}%
\begin{pgfscope}%
\pgfsetrectcap%
\pgfsetmiterjoin%
\pgfsetlinewidth{0.803000pt}%
\definecolor{currentstroke}{rgb}{0.000000,0.000000,0.000000}%
\pgfsetstrokecolor{currentstroke}%
\pgfsetdash{}{0pt}%
\pgfpathmoveto{\pgfqpoint{0.800000in}{0.528000in}}%
\pgfpathlineto{\pgfqpoint{5.760000in}{0.528000in}}%
\pgfusepath{stroke}%
\end{pgfscope}%
\begin{pgfscope}%
\pgfsetrectcap%
\pgfsetmiterjoin%
\pgfsetlinewidth{0.803000pt}%
\definecolor{currentstroke}{rgb}{0.000000,0.000000,0.000000}%
\pgfsetstrokecolor{currentstroke}%
\pgfsetdash{}{0pt}%
\pgfpathmoveto{\pgfqpoint{0.800000in}{4.224000in}}%
\pgfpathlineto{\pgfqpoint{5.760000in}{4.224000in}}%
\pgfusepath{stroke}%
\end{pgfscope}%
\begin{pgfscope}%
\pgfsetbuttcap%
\pgfsetmiterjoin%
\definecolor{currentfill}{rgb}{1.000000,1.000000,1.000000}%
\pgfsetfillcolor{currentfill}%
\pgfsetfillopacity{0.800000}%
\pgfsetlinewidth{1.003750pt}%
\definecolor{currentstroke}{rgb}{0.800000,0.800000,0.800000}%
\pgfsetstrokecolor{currentstroke}%
\pgfsetstrokeopacity{0.800000}%
\pgfsetdash{}{0pt}%
\pgfpathmoveto{\pgfqpoint{0.897222in}{3.705174in}}%
\pgfpathlineto{\pgfqpoint{3.351215in}{3.705174in}}%
\pgfpathquadraticcurveto{\pgfqpoint{3.378993in}{3.705174in}}{\pgfqpoint{3.378993in}{3.732952in}}%
\pgfpathlineto{\pgfqpoint{3.378993in}{4.126778in}}%
\pgfpathquadraticcurveto{\pgfqpoint{3.378993in}{4.154556in}}{\pgfqpoint{3.351215in}{4.154556in}}%
\pgfpathlineto{\pgfqpoint{0.897222in}{4.154556in}}%
\pgfpathquadraticcurveto{\pgfqpoint{0.869444in}{4.154556in}}{\pgfqpoint{0.869444in}{4.126778in}}%
\pgfpathlineto{\pgfqpoint{0.869444in}{3.732952in}}%
\pgfpathquadraticcurveto{\pgfqpoint{0.869444in}{3.705174in}}{\pgfqpoint{0.897222in}{3.705174in}}%
\pgfpathclose%
\pgfusepath{stroke,fill}%
\end{pgfscope}%
\begin{pgfscope}%
\pgfsetrectcap%
\pgfsetroundjoin%
\pgfsetlinewidth{1.505625pt}%
\definecolor{currentstroke}{rgb}{0.121569,0.466667,0.705882}%
\pgfsetstrokecolor{currentstroke}%
\pgfsetdash{}{0pt}%
\pgfpathmoveto{\pgfqpoint{0.925000in}{4.042088in}}%
\pgfpathlineto{\pgfqpoint{1.202778in}{4.042088in}}%
\pgfusepath{stroke}%
\end{pgfscope}%
\begin{pgfscope}%
\pgftext[x=1.313889in,y=3.993477in,left,base]{\sffamily\fontsize{10.000000}{12.000000}\selectfont Data}%
\end{pgfscope}%
\begin{pgfscope}%
\pgfsetrectcap%
\pgfsetroundjoin%
\pgfsetlinewidth{1.505625pt}%
\definecolor{currentstroke}{rgb}{1.000000,0.498039,0.054902}%
\pgfsetstrokecolor{currentstroke}%
\pgfsetdash{}{0pt}%
\pgfpathmoveto{\pgfqpoint{0.925000in}{3.838231in}}%
\pgfpathlineto{\pgfqpoint{1.202778in}{3.838231in}}%
\pgfusepath{stroke}%
\end{pgfscope}%
\begin{pgfscope}%
\pgftext[x=1.313889in,y=3.789620in,left,base]{\sffamily\fontsize{10.000000}{12.000000}\selectfont Least squares approximation}%
\end{pgfscope}%
\end{pgfpicture}%
\makeatother%
\endgroup%

			\caption{Données expérimentales}
			\label{fig:exp_poly}
		\end{figure}

		On peut ensuite se demander comment varie la constante $c$ en fonction du nombre de sommets du polygone considéré. Intuitivement, on peut alors penser qu'en faisant tendre le nombre de points vers l'infini, notre polygone va converger uniformément vers un cercle, et qu'ainsi, on devrait retrouver asymptotiquement le comportement que l'on à avec un cercle. Cette convergence implique aussi une forme de stabilisation de $c$, car les contributions en aire apportés par les polygones de plus grand nombre de sommets deviennent de plus en plus petites.

		En calculant des regressions permettant d'estimer $c$ pour différents polygones, en tirant toujours le même nombre de points ($1000$), on trace le graphe Figure~\ref{fig:exp_c}

		\begin{figure}[htpb]
			\centering
			%% Creator: Matplotlib, PGF backend
%%
%% To include the figure in your LaTeX document, write
%%   \input{<filename>.pgf}
%%
%% Make sure the required packages are loaded in your preamble
%%   \usepackage{pgf}
%%
%% Figures using additional raster images can only be included by \input if
%% they are in the same directory as the main LaTeX file. For loading figures
%% from other directories you can use the `import` package
%%   \usepackage{import}
%% and then include the figures with
%%   \import{<path to file>}{<filename>.pgf}
%%
%% Matplotlib used the following preamble
%%   \usepackage{fontspec}
%%   \setmainfont{DejaVu Serif}
%%   \setsansfont{DejaVu Sans}
%%   \setmonofont{DejaVu Sans Mono}
%%
\begingroup%
\makeatletter%
\begin{pgfpicture}%
\pgfpathrectangle{\pgfpointorigin}{\pgfqpoint{6.400000in}{4.800000in}}%
\pgfusepath{use as bounding box, clip}%
\begin{pgfscope}%
\pgfsetbuttcap%
\pgfsetmiterjoin%
\definecolor{currentfill}{rgb}{1.000000,1.000000,1.000000}%
\pgfsetfillcolor{currentfill}%
\pgfsetlinewidth{0.000000pt}%
\definecolor{currentstroke}{rgb}{1.000000,1.000000,1.000000}%
\pgfsetstrokecolor{currentstroke}%
\pgfsetdash{}{0pt}%
\pgfpathmoveto{\pgfqpoint{0.000000in}{0.000000in}}%
\pgfpathlineto{\pgfqpoint{6.400000in}{0.000000in}}%
\pgfpathlineto{\pgfqpoint{6.400000in}{4.800000in}}%
\pgfpathlineto{\pgfqpoint{0.000000in}{4.800000in}}%
\pgfpathclose%
\pgfusepath{fill}%
\end{pgfscope}%
\begin{pgfscope}%
\pgfsetbuttcap%
\pgfsetmiterjoin%
\definecolor{currentfill}{rgb}{1.000000,1.000000,1.000000}%
\pgfsetfillcolor{currentfill}%
\pgfsetlinewidth{0.000000pt}%
\definecolor{currentstroke}{rgb}{0.000000,0.000000,0.000000}%
\pgfsetstrokecolor{currentstroke}%
\pgfsetstrokeopacity{0.000000}%
\pgfsetdash{}{0pt}%
\pgfpathmoveto{\pgfqpoint{0.800000in}{0.528000in}}%
\pgfpathlineto{\pgfqpoint{5.760000in}{0.528000in}}%
\pgfpathlineto{\pgfqpoint{5.760000in}{4.224000in}}%
\pgfpathlineto{\pgfqpoint{0.800000in}{4.224000in}}%
\pgfpathclose%
\pgfusepath{fill}%
\end{pgfscope}%
\begin{pgfscope}%
\pgfsetbuttcap%
\pgfsetroundjoin%
\definecolor{currentfill}{rgb}{0.000000,0.000000,0.000000}%
\pgfsetfillcolor{currentfill}%
\pgfsetlinewidth{0.803000pt}%
\definecolor{currentstroke}{rgb}{0.000000,0.000000,0.000000}%
\pgfsetstrokecolor{currentstroke}%
\pgfsetdash{}{0pt}%
\pgfsys@defobject{currentmarker}{\pgfqpoint{0.000000in}{-0.048611in}}{\pgfqpoint{0.000000in}{0.000000in}}{%
\pgfpathmoveto{\pgfqpoint{0.000000in}{0.000000in}}%
\pgfpathlineto{\pgfqpoint{0.000000in}{-0.048611in}}%
\pgfusepath{stroke,fill}%
}%
\begin{pgfscope}%
\pgfsys@transformshift{0.997791in}{0.528000in}%
\pgfsys@useobject{currentmarker}{}%
\end{pgfscope}%
\end{pgfscope}%
\begin{pgfscope}%
\pgftext[x=0.997791in,y=0.430778in,,top]{\sffamily\fontsize{10.000000}{12.000000}\selectfont 0}%
\end{pgfscope}%
\begin{pgfscope}%
\pgfsetbuttcap%
\pgfsetroundjoin%
\definecolor{currentfill}{rgb}{0.000000,0.000000,0.000000}%
\pgfsetfillcolor{currentfill}%
\pgfsetlinewidth{0.803000pt}%
\definecolor{currentstroke}{rgb}{0.000000,0.000000,0.000000}%
\pgfsetstrokecolor{currentstroke}%
\pgfsetdash{}{0pt}%
\pgfsys@defobject{currentmarker}{\pgfqpoint{0.000000in}{-0.048611in}}{\pgfqpoint{0.000000in}{0.000000in}}{%
\pgfpathmoveto{\pgfqpoint{0.000000in}{0.000000in}}%
\pgfpathlineto{\pgfqpoint{0.000000in}{-0.048611in}}%
\pgfusepath{stroke,fill}%
}%
\begin{pgfscope}%
\pgfsys@transformshift{1.919896in}{0.528000in}%
\pgfsys@useobject{currentmarker}{}%
\end{pgfscope}%
\end{pgfscope}%
\begin{pgfscope}%
\pgftext[x=1.919896in,y=0.430778in,,top]{\sffamily\fontsize{10.000000}{12.000000}\selectfont 100}%
\end{pgfscope}%
\begin{pgfscope}%
\pgfsetbuttcap%
\pgfsetroundjoin%
\definecolor{currentfill}{rgb}{0.000000,0.000000,0.000000}%
\pgfsetfillcolor{currentfill}%
\pgfsetlinewidth{0.803000pt}%
\definecolor{currentstroke}{rgb}{0.000000,0.000000,0.000000}%
\pgfsetstrokecolor{currentstroke}%
\pgfsetdash{}{0pt}%
\pgfsys@defobject{currentmarker}{\pgfqpoint{0.000000in}{-0.048611in}}{\pgfqpoint{0.000000in}{0.000000in}}{%
\pgfpathmoveto{\pgfqpoint{0.000000in}{0.000000in}}%
\pgfpathlineto{\pgfqpoint{0.000000in}{-0.048611in}}%
\pgfusepath{stroke,fill}%
}%
\begin{pgfscope}%
\pgfsys@transformshift{2.842000in}{0.528000in}%
\pgfsys@useobject{currentmarker}{}%
\end{pgfscope}%
\end{pgfscope}%
\begin{pgfscope}%
\pgftext[x=2.842000in,y=0.430778in,,top]{\sffamily\fontsize{10.000000}{12.000000}\selectfont 200}%
\end{pgfscope}%
\begin{pgfscope}%
\pgfsetbuttcap%
\pgfsetroundjoin%
\definecolor{currentfill}{rgb}{0.000000,0.000000,0.000000}%
\pgfsetfillcolor{currentfill}%
\pgfsetlinewidth{0.803000pt}%
\definecolor{currentstroke}{rgb}{0.000000,0.000000,0.000000}%
\pgfsetstrokecolor{currentstroke}%
\pgfsetdash{}{0pt}%
\pgfsys@defobject{currentmarker}{\pgfqpoint{0.000000in}{-0.048611in}}{\pgfqpoint{0.000000in}{0.000000in}}{%
\pgfpathmoveto{\pgfqpoint{0.000000in}{0.000000in}}%
\pgfpathlineto{\pgfqpoint{0.000000in}{-0.048611in}}%
\pgfusepath{stroke,fill}%
}%
\begin{pgfscope}%
\pgfsys@transformshift{3.764105in}{0.528000in}%
\pgfsys@useobject{currentmarker}{}%
\end{pgfscope}%
\end{pgfscope}%
\begin{pgfscope}%
\pgftext[x=3.764105in,y=0.430778in,,top]{\sffamily\fontsize{10.000000}{12.000000}\selectfont 300}%
\end{pgfscope}%
\begin{pgfscope}%
\pgfsetbuttcap%
\pgfsetroundjoin%
\definecolor{currentfill}{rgb}{0.000000,0.000000,0.000000}%
\pgfsetfillcolor{currentfill}%
\pgfsetlinewidth{0.803000pt}%
\definecolor{currentstroke}{rgb}{0.000000,0.000000,0.000000}%
\pgfsetstrokecolor{currentstroke}%
\pgfsetdash{}{0pt}%
\pgfsys@defobject{currentmarker}{\pgfqpoint{0.000000in}{-0.048611in}}{\pgfqpoint{0.000000in}{0.000000in}}{%
\pgfpathmoveto{\pgfqpoint{0.000000in}{0.000000in}}%
\pgfpathlineto{\pgfqpoint{0.000000in}{-0.048611in}}%
\pgfusepath{stroke,fill}%
}%
\begin{pgfscope}%
\pgfsys@transformshift{4.686209in}{0.528000in}%
\pgfsys@useobject{currentmarker}{}%
\end{pgfscope}%
\end{pgfscope}%
\begin{pgfscope}%
\pgftext[x=4.686209in,y=0.430778in,,top]{\sffamily\fontsize{10.000000}{12.000000}\selectfont 400}%
\end{pgfscope}%
\begin{pgfscope}%
\pgfsetbuttcap%
\pgfsetroundjoin%
\definecolor{currentfill}{rgb}{0.000000,0.000000,0.000000}%
\pgfsetfillcolor{currentfill}%
\pgfsetlinewidth{0.803000pt}%
\definecolor{currentstroke}{rgb}{0.000000,0.000000,0.000000}%
\pgfsetstrokecolor{currentstroke}%
\pgfsetdash{}{0pt}%
\pgfsys@defobject{currentmarker}{\pgfqpoint{0.000000in}{-0.048611in}}{\pgfqpoint{0.000000in}{0.000000in}}{%
\pgfpathmoveto{\pgfqpoint{0.000000in}{0.000000in}}%
\pgfpathlineto{\pgfqpoint{0.000000in}{-0.048611in}}%
\pgfusepath{stroke,fill}%
}%
\begin{pgfscope}%
\pgfsys@transformshift{5.608314in}{0.528000in}%
\pgfsys@useobject{currentmarker}{}%
\end{pgfscope}%
\end{pgfscope}%
\begin{pgfscope}%
\pgftext[x=5.608314in,y=0.430778in,,top]{\sffamily\fontsize{10.000000}{12.000000}\selectfont 500}%
\end{pgfscope}%
\begin{pgfscope}%
\pgfsetbuttcap%
\pgfsetroundjoin%
\definecolor{currentfill}{rgb}{0.000000,0.000000,0.000000}%
\pgfsetfillcolor{currentfill}%
\pgfsetlinewidth{0.803000pt}%
\definecolor{currentstroke}{rgb}{0.000000,0.000000,0.000000}%
\pgfsetstrokecolor{currentstroke}%
\pgfsetdash{}{0pt}%
\pgfsys@defobject{currentmarker}{\pgfqpoint{-0.048611in}{0.000000in}}{\pgfqpoint{0.000000in}{0.000000in}}{%
\pgfpathmoveto{\pgfqpoint{0.000000in}{0.000000in}}%
\pgfpathlineto{\pgfqpoint{-0.048611in}{0.000000in}}%
\pgfusepath{stroke,fill}%
}%
\begin{pgfscope}%
\pgfsys@transformshift{0.800000in}{1.116474in}%
\pgfsys@useobject{currentmarker}{}%
\end{pgfscope}%
\end{pgfscope}%
\begin{pgfscope}%
\pgftext[x=0.481898in,y=1.063712in,left,base]{\sffamily\fontsize{10.000000}{12.000000}\selectfont 2.5}%
\end{pgfscope}%
\begin{pgfscope}%
\pgfsetbuttcap%
\pgfsetroundjoin%
\definecolor{currentfill}{rgb}{0.000000,0.000000,0.000000}%
\pgfsetfillcolor{currentfill}%
\pgfsetlinewidth{0.803000pt}%
\definecolor{currentstroke}{rgb}{0.000000,0.000000,0.000000}%
\pgfsetstrokecolor{currentstroke}%
\pgfsetdash{}{0pt}%
\pgfsys@defobject{currentmarker}{\pgfqpoint{-0.048611in}{0.000000in}}{\pgfqpoint{0.000000in}{0.000000in}}{%
\pgfpathmoveto{\pgfqpoint{0.000000in}{0.000000in}}%
\pgfpathlineto{\pgfqpoint{-0.048611in}{0.000000in}}%
\pgfusepath{stroke,fill}%
}%
\begin{pgfscope}%
\pgfsys@transformshift{0.800000in}{1.896154in}%
\pgfsys@useobject{currentmarker}{}%
\end{pgfscope}%
\end{pgfscope}%
\begin{pgfscope}%
\pgftext[x=0.481898in,y=1.843393in,left,base]{\sffamily\fontsize{10.000000}{12.000000}\selectfont 3.0}%
\end{pgfscope}%
\begin{pgfscope}%
\pgfsetbuttcap%
\pgfsetroundjoin%
\definecolor{currentfill}{rgb}{0.000000,0.000000,0.000000}%
\pgfsetfillcolor{currentfill}%
\pgfsetlinewidth{0.803000pt}%
\definecolor{currentstroke}{rgb}{0.000000,0.000000,0.000000}%
\pgfsetstrokecolor{currentstroke}%
\pgfsetdash{}{0pt}%
\pgfsys@defobject{currentmarker}{\pgfqpoint{-0.048611in}{0.000000in}}{\pgfqpoint{0.000000in}{0.000000in}}{%
\pgfpathmoveto{\pgfqpoint{0.000000in}{0.000000in}}%
\pgfpathlineto{\pgfqpoint{-0.048611in}{0.000000in}}%
\pgfusepath{stroke,fill}%
}%
\begin{pgfscope}%
\pgfsys@transformshift{0.800000in}{2.675835in}%
\pgfsys@useobject{currentmarker}{}%
\end{pgfscope}%
\end{pgfscope}%
\begin{pgfscope}%
\pgftext[x=0.481898in,y=2.623073in,left,base]{\sffamily\fontsize{10.000000}{12.000000}\selectfont 3.5}%
\end{pgfscope}%
\begin{pgfscope}%
\pgfsetbuttcap%
\pgfsetroundjoin%
\definecolor{currentfill}{rgb}{0.000000,0.000000,0.000000}%
\pgfsetfillcolor{currentfill}%
\pgfsetlinewidth{0.803000pt}%
\definecolor{currentstroke}{rgb}{0.000000,0.000000,0.000000}%
\pgfsetstrokecolor{currentstroke}%
\pgfsetdash{}{0pt}%
\pgfsys@defobject{currentmarker}{\pgfqpoint{-0.048611in}{0.000000in}}{\pgfqpoint{0.000000in}{0.000000in}}{%
\pgfpathmoveto{\pgfqpoint{0.000000in}{0.000000in}}%
\pgfpathlineto{\pgfqpoint{-0.048611in}{0.000000in}}%
\pgfusepath{stroke,fill}%
}%
\begin{pgfscope}%
\pgfsys@transformshift{0.800000in}{3.455515in}%
\pgfsys@useobject{currentmarker}{}%
\end{pgfscope}%
\end{pgfscope}%
\begin{pgfscope}%
\pgftext[x=0.481898in,y=3.402754in,left,base]{\sffamily\fontsize{10.000000}{12.000000}\selectfont 4.0}%
\end{pgfscope}%
\begin{pgfscope}%
\pgfpathrectangle{\pgfqpoint{0.800000in}{0.528000in}}{\pgfqpoint{4.960000in}{3.696000in}} %
\pgfusepath{clip}%
\pgfsetrectcap%
\pgfsetroundjoin%
\pgfsetlinewidth{1.505625pt}%
\definecolor{currentstroke}{rgb}{0.121569,0.466667,0.705882}%
\pgfsetstrokecolor{currentstroke}%
\pgfsetdash{}{0pt}%
\pgfpathmoveto{\pgfqpoint{1.025455in}{0.696000in}}%
\pgfpathlineto{\pgfqpoint{1.034676in}{1.382603in}}%
\pgfpathlineto{\pgfqpoint{1.043897in}{1.910013in}}%
\pgfpathlineto{\pgfqpoint{1.053118in}{2.329666in}}%
\pgfpathlineto{\pgfqpoint{1.062339in}{2.669156in}}%
\pgfpathlineto{\pgfqpoint{1.071560in}{2.943625in}}%
\pgfpathlineto{\pgfqpoint{1.080781in}{3.163212in}}%
\pgfpathlineto{\pgfqpoint{1.090002in}{3.341713in}}%
\pgfpathlineto{\pgfqpoint{1.099223in}{3.481094in}}%
\pgfpathlineto{\pgfqpoint{1.108444in}{3.600198in}}%
\pgfpathlineto{\pgfqpoint{1.117665in}{3.689015in}}%
\pgfpathlineto{\pgfqpoint{1.126886in}{3.761673in}}%
\pgfpathlineto{\pgfqpoint{1.136107in}{3.818565in}}%
\pgfpathlineto{\pgfqpoint{1.145328in}{3.859996in}}%
\pgfpathlineto{\pgfqpoint{1.154549in}{3.893943in}}%
\pgfpathlineto{\pgfqpoint{1.163770in}{3.922683in}}%
\pgfpathlineto{\pgfqpoint{1.172991in}{3.947700in}}%
\pgfpathlineto{\pgfqpoint{1.182212in}{3.969314in}}%
\pgfpathlineto{\pgfqpoint{1.191433in}{3.981472in}}%
\pgfpathlineto{\pgfqpoint{1.200654in}{3.989165in}}%
\pgfpathlineto{\pgfqpoint{1.219096in}{4.011819in}}%
\pgfpathlineto{\pgfqpoint{1.237539in}{4.020648in}}%
\pgfpathlineto{\pgfqpoint{1.246760in}{4.026199in}}%
\pgfpathlineto{\pgfqpoint{1.255981in}{4.033465in}}%
\pgfpathlineto{\pgfqpoint{1.265202in}{4.030938in}}%
\pgfpathlineto{\pgfqpoint{1.274423in}{4.030988in}}%
\pgfpathlineto{\pgfqpoint{1.283644in}{4.035905in}}%
\pgfpathlineto{\pgfqpoint{1.292865in}{4.035458in}}%
\pgfpathlineto{\pgfqpoint{1.302086in}{4.039092in}}%
\pgfpathlineto{\pgfqpoint{1.311307in}{4.037729in}}%
\pgfpathlineto{\pgfqpoint{1.320528in}{4.043796in}}%
\pgfpathlineto{\pgfqpoint{1.329749in}{4.040051in}}%
\pgfpathlineto{\pgfqpoint{1.338970in}{4.044321in}}%
\pgfpathlineto{\pgfqpoint{1.348191in}{4.042520in}}%
\pgfpathlineto{\pgfqpoint{1.357412in}{4.043844in}}%
\pgfpathlineto{\pgfqpoint{1.366633in}{4.046732in}}%
\pgfpathlineto{\pgfqpoint{1.375854in}{4.047629in}}%
\pgfpathlineto{\pgfqpoint{1.403517in}{4.043551in}}%
\pgfpathlineto{\pgfqpoint{1.412738in}{4.050846in}}%
\pgfpathlineto{\pgfqpoint{1.421959in}{4.048312in}}%
\pgfpathlineto{\pgfqpoint{1.431181in}{4.049683in}}%
\pgfpathlineto{\pgfqpoint{1.440402in}{4.047002in}}%
\pgfpathlineto{\pgfqpoint{1.449623in}{4.048097in}}%
\pgfpathlineto{\pgfqpoint{1.458844in}{4.046706in}}%
\pgfpathlineto{\pgfqpoint{1.477286in}{4.049450in}}%
\pgfpathlineto{\pgfqpoint{1.495728in}{4.048539in}}%
\pgfpathlineto{\pgfqpoint{1.514170in}{4.047245in}}%
\pgfpathlineto{\pgfqpoint{1.523391in}{4.053499in}}%
\pgfpathlineto{\pgfqpoint{1.541833in}{4.046656in}}%
\pgfpathlineto{\pgfqpoint{1.551054in}{4.050511in}}%
\pgfpathlineto{\pgfqpoint{1.569496in}{4.048782in}}%
\pgfpathlineto{\pgfqpoint{1.578717in}{4.046552in}}%
\pgfpathlineto{\pgfqpoint{1.587938in}{4.050195in}}%
\pgfpathlineto{\pgfqpoint{1.597159in}{4.048284in}}%
\pgfpathlineto{\pgfqpoint{1.606380in}{4.048350in}}%
\pgfpathlineto{\pgfqpoint{1.615601in}{4.051112in}}%
\pgfpathlineto{\pgfqpoint{1.624822in}{4.049522in}}%
\pgfpathlineto{\pgfqpoint{1.634044in}{4.051647in}}%
\pgfpathlineto{\pgfqpoint{1.643265in}{4.049943in}}%
\pgfpathlineto{\pgfqpoint{1.652486in}{4.046770in}}%
\pgfpathlineto{\pgfqpoint{1.661707in}{4.053529in}}%
\pgfpathlineto{\pgfqpoint{1.670928in}{4.053545in}}%
\pgfpathlineto{\pgfqpoint{1.680149in}{4.050222in}}%
\pgfpathlineto{\pgfqpoint{1.698591in}{4.049876in}}%
\pgfpathlineto{\pgfqpoint{1.707812in}{4.048808in}}%
\pgfpathlineto{\pgfqpoint{1.717033in}{4.051965in}}%
\pgfpathlineto{\pgfqpoint{1.726254in}{4.048436in}}%
\pgfpathlineto{\pgfqpoint{1.735475in}{4.051978in}}%
\pgfpathlineto{\pgfqpoint{1.744696in}{4.050739in}}%
\pgfpathlineto{\pgfqpoint{1.753917in}{4.052336in}}%
\pgfpathlineto{\pgfqpoint{1.763138in}{4.047336in}}%
\pgfpathlineto{\pgfqpoint{1.772359in}{4.051537in}}%
\pgfpathlineto{\pgfqpoint{1.781580in}{4.048062in}}%
\pgfpathlineto{\pgfqpoint{1.809243in}{4.051107in}}%
\pgfpathlineto{\pgfqpoint{1.818464in}{4.046594in}}%
\pgfpathlineto{\pgfqpoint{1.827685in}{4.050596in}}%
\pgfpathlineto{\pgfqpoint{1.836906in}{4.049103in}}%
\pgfpathlineto{\pgfqpoint{1.846128in}{4.050097in}}%
\pgfpathlineto{\pgfqpoint{1.855349in}{4.048942in}}%
\pgfpathlineto{\pgfqpoint{1.864570in}{4.050829in}}%
\pgfpathlineto{\pgfqpoint{1.873791in}{4.049889in}}%
\pgfpathlineto{\pgfqpoint{1.883012in}{4.053040in}}%
\pgfpathlineto{\pgfqpoint{1.892233in}{4.048592in}}%
\pgfpathlineto{\pgfqpoint{1.901454in}{4.052535in}}%
\pgfpathlineto{\pgfqpoint{1.910675in}{4.050952in}}%
\pgfpathlineto{\pgfqpoint{1.919896in}{4.054932in}}%
\pgfpathlineto{\pgfqpoint{1.929117in}{4.050562in}}%
\pgfpathlineto{\pgfqpoint{1.947559in}{4.048465in}}%
\pgfpathlineto{\pgfqpoint{1.956780in}{4.051075in}}%
\pgfpathlineto{\pgfqpoint{1.975222in}{4.053379in}}%
\pgfpathlineto{\pgfqpoint{1.984443in}{4.049144in}}%
\pgfpathlineto{\pgfqpoint{2.002885in}{4.050633in}}%
\pgfpathlineto{\pgfqpoint{2.012106in}{4.049129in}}%
\pgfpathlineto{\pgfqpoint{2.021327in}{4.050868in}}%
\pgfpathlineto{\pgfqpoint{2.030548in}{4.050438in}}%
\pgfpathlineto{\pgfqpoint{2.048991in}{4.051191in}}%
\pgfpathlineto{\pgfqpoint{2.058212in}{4.047910in}}%
\pgfpathlineto{\pgfqpoint{2.067433in}{4.052268in}}%
\pgfpathlineto{\pgfqpoint{2.076654in}{4.053072in}}%
\pgfpathlineto{\pgfqpoint{2.095096in}{4.048874in}}%
\pgfpathlineto{\pgfqpoint{2.104317in}{4.050890in}}%
\pgfpathlineto{\pgfqpoint{2.113538in}{4.054335in}}%
\pgfpathlineto{\pgfqpoint{2.122759in}{4.054153in}}%
\pgfpathlineto{\pgfqpoint{2.131980in}{4.049552in}}%
\pgfpathlineto{\pgfqpoint{2.141201in}{4.053497in}}%
\pgfpathlineto{\pgfqpoint{2.150422in}{4.047522in}}%
\pgfpathlineto{\pgfqpoint{2.159643in}{4.048324in}}%
\pgfpathlineto{\pgfqpoint{2.168864in}{4.052007in}}%
\pgfpathlineto{\pgfqpoint{2.178085in}{4.049518in}}%
\pgfpathlineto{\pgfqpoint{2.187306in}{4.051461in}}%
\pgfpathlineto{\pgfqpoint{2.205748in}{4.049178in}}%
\pgfpathlineto{\pgfqpoint{2.214969in}{4.054945in}}%
\pgfpathlineto{\pgfqpoint{2.224190in}{4.051722in}}%
\pgfpathlineto{\pgfqpoint{2.233411in}{4.052053in}}%
\pgfpathlineto{\pgfqpoint{2.242632in}{4.046552in}}%
\pgfpathlineto{\pgfqpoint{2.251854in}{4.056000in}}%
\pgfpathlineto{\pgfqpoint{2.261075in}{4.049553in}}%
\pgfpathlineto{\pgfqpoint{2.270296in}{4.053073in}}%
\pgfpathlineto{\pgfqpoint{2.279517in}{4.050177in}}%
\pgfpathlineto{\pgfqpoint{2.297959in}{4.048026in}}%
\pgfpathlineto{\pgfqpoint{2.307180in}{4.053052in}}%
\pgfpathlineto{\pgfqpoint{2.316401in}{4.052369in}}%
\pgfpathlineto{\pgfqpoint{2.325622in}{4.048500in}}%
\pgfpathlineto{\pgfqpoint{2.371727in}{4.047895in}}%
\pgfpathlineto{\pgfqpoint{2.380948in}{4.051244in}}%
\pgfpathlineto{\pgfqpoint{2.390169in}{4.050679in}}%
\pgfpathlineto{\pgfqpoint{2.399390in}{4.044592in}}%
\pgfpathlineto{\pgfqpoint{2.408611in}{4.051343in}}%
\pgfpathlineto{\pgfqpoint{2.417832in}{4.054757in}}%
\pgfpathlineto{\pgfqpoint{2.427053in}{4.051892in}}%
\pgfpathlineto{\pgfqpoint{2.436274in}{4.052778in}}%
\pgfpathlineto{\pgfqpoint{2.445495in}{4.049781in}}%
\pgfpathlineto{\pgfqpoint{2.463938in}{4.048179in}}%
\pgfpathlineto{\pgfqpoint{2.473159in}{4.053137in}}%
\pgfpathlineto{\pgfqpoint{2.482380in}{4.053124in}}%
\pgfpathlineto{\pgfqpoint{2.491601in}{4.049027in}}%
\pgfpathlineto{\pgfqpoint{2.500822in}{4.050555in}}%
\pgfpathlineto{\pgfqpoint{2.510043in}{4.049064in}}%
\pgfpathlineto{\pgfqpoint{2.528485in}{4.052413in}}%
\pgfpathlineto{\pgfqpoint{2.537706in}{4.048119in}}%
\pgfpathlineto{\pgfqpoint{2.546927in}{4.055117in}}%
\pgfpathlineto{\pgfqpoint{2.556148in}{4.050635in}}%
\pgfpathlineto{\pgfqpoint{2.565369in}{4.048988in}}%
\pgfpathlineto{\pgfqpoint{2.574590in}{4.051995in}}%
\pgfpathlineto{\pgfqpoint{2.583811in}{4.050096in}}%
\pgfpathlineto{\pgfqpoint{2.593032in}{4.049750in}}%
\pgfpathlineto{\pgfqpoint{2.602253in}{4.047685in}}%
\pgfpathlineto{\pgfqpoint{2.611474in}{4.050889in}}%
\pgfpathlineto{\pgfqpoint{2.620695in}{4.052721in}}%
\pgfpathlineto{\pgfqpoint{2.629916in}{4.050328in}}%
\pgfpathlineto{\pgfqpoint{2.639137in}{4.046583in}}%
\pgfpathlineto{\pgfqpoint{2.657579in}{4.052299in}}%
\pgfpathlineto{\pgfqpoint{2.666801in}{4.050387in}}%
\pgfpathlineto{\pgfqpoint{2.676022in}{4.052323in}}%
\pgfpathlineto{\pgfqpoint{2.703685in}{4.052942in}}%
\pgfpathlineto{\pgfqpoint{2.712906in}{4.047431in}}%
\pgfpathlineto{\pgfqpoint{2.722127in}{4.052313in}}%
\pgfpathlineto{\pgfqpoint{2.731348in}{4.047688in}}%
\pgfpathlineto{\pgfqpoint{2.740569in}{4.051415in}}%
\pgfpathlineto{\pgfqpoint{2.749790in}{4.049272in}}%
\pgfpathlineto{\pgfqpoint{2.777453in}{4.049007in}}%
\pgfpathlineto{\pgfqpoint{2.786674in}{4.048986in}}%
\pgfpathlineto{\pgfqpoint{2.795895in}{4.051613in}}%
\pgfpathlineto{\pgfqpoint{2.805116in}{4.049045in}}%
\pgfpathlineto{\pgfqpoint{2.814337in}{4.050949in}}%
\pgfpathlineto{\pgfqpoint{2.823558in}{4.051571in}}%
\pgfpathlineto{\pgfqpoint{2.832779in}{4.050168in}}%
\pgfpathlineto{\pgfqpoint{2.842000in}{4.050517in}}%
\pgfpathlineto{\pgfqpoint{2.851221in}{4.049328in}}%
\pgfpathlineto{\pgfqpoint{2.860442in}{4.045441in}}%
\pgfpathlineto{\pgfqpoint{2.869664in}{4.050266in}}%
\pgfpathlineto{\pgfqpoint{2.878885in}{4.052016in}}%
\pgfpathlineto{\pgfqpoint{2.888106in}{4.046846in}}%
\pgfpathlineto{\pgfqpoint{2.897327in}{4.051083in}}%
\pgfpathlineto{\pgfqpoint{2.906548in}{4.051219in}}%
\pgfpathlineto{\pgfqpoint{2.915769in}{4.046826in}}%
\pgfpathlineto{\pgfqpoint{2.924990in}{4.048029in}}%
\pgfpathlineto{\pgfqpoint{2.934211in}{4.051727in}}%
\pgfpathlineto{\pgfqpoint{2.943432in}{4.047606in}}%
\pgfpathlineto{\pgfqpoint{2.961874in}{4.052615in}}%
\pgfpathlineto{\pgfqpoint{2.980316in}{4.052278in}}%
\pgfpathlineto{\pgfqpoint{2.989537in}{4.048301in}}%
\pgfpathlineto{\pgfqpoint{2.998758in}{4.050601in}}%
\pgfpathlineto{\pgfqpoint{3.007979in}{4.051678in}}%
\pgfpathlineto{\pgfqpoint{3.017200in}{4.054230in}}%
\pgfpathlineto{\pgfqpoint{3.035642in}{4.052752in}}%
\pgfpathlineto{\pgfqpoint{3.044863in}{4.047848in}}%
\pgfpathlineto{\pgfqpoint{3.063305in}{4.054961in}}%
\pgfpathlineto{\pgfqpoint{3.072526in}{4.053595in}}%
\pgfpathlineto{\pgfqpoint{3.081748in}{4.045988in}}%
\pgfpathlineto{\pgfqpoint{3.090969in}{4.049449in}}%
\pgfpathlineto{\pgfqpoint{3.100190in}{4.048085in}}%
\pgfpathlineto{\pgfqpoint{3.109411in}{4.053312in}}%
\pgfpathlineto{\pgfqpoint{3.118632in}{4.051329in}}%
\pgfpathlineto{\pgfqpoint{3.127853in}{4.048189in}}%
\pgfpathlineto{\pgfqpoint{3.137074in}{4.049676in}}%
\pgfpathlineto{\pgfqpoint{3.146295in}{4.049114in}}%
\pgfpathlineto{\pgfqpoint{3.155516in}{4.052519in}}%
\pgfpathlineto{\pgfqpoint{3.173958in}{4.048971in}}%
\pgfpathlineto{\pgfqpoint{3.183179in}{4.050123in}}%
\pgfpathlineto{\pgfqpoint{3.192400in}{4.052417in}}%
\pgfpathlineto{\pgfqpoint{3.201621in}{4.047951in}}%
\pgfpathlineto{\pgfqpoint{3.210842in}{4.048354in}}%
\pgfpathlineto{\pgfqpoint{3.220063in}{4.050026in}}%
\pgfpathlineto{\pgfqpoint{3.238505in}{4.050360in}}%
\pgfpathlineto{\pgfqpoint{3.247726in}{4.054901in}}%
\pgfpathlineto{\pgfqpoint{3.256947in}{4.052218in}}%
\pgfpathlineto{\pgfqpoint{3.266168in}{4.053102in}}%
\pgfpathlineto{\pgfqpoint{3.275389in}{4.048430in}}%
\pgfpathlineto{\pgfqpoint{3.284611in}{4.051544in}}%
\pgfpathlineto{\pgfqpoint{3.293832in}{4.046816in}}%
\pgfpathlineto{\pgfqpoint{3.303053in}{4.049483in}}%
\pgfpathlineto{\pgfqpoint{3.321495in}{4.051364in}}%
\pgfpathlineto{\pgfqpoint{3.330716in}{4.047774in}}%
\pgfpathlineto{\pgfqpoint{3.339937in}{4.054464in}}%
\pgfpathlineto{\pgfqpoint{3.349158in}{4.049698in}}%
\pgfpathlineto{\pgfqpoint{3.358379in}{4.053568in}}%
\pgfpathlineto{\pgfqpoint{3.367600in}{4.050978in}}%
\pgfpathlineto{\pgfqpoint{3.376821in}{4.050949in}}%
\pgfpathlineto{\pgfqpoint{3.386042in}{4.045066in}}%
\pgfpathlineto{\pgfqpoint{3.395263in}{4.048370in}}%
\pgfpathlineto{\pgfqpoint{3.404484in}{4.054083in}}%
\pgfpathlineto{\pgfqpoint{3.441368in}{4.050132in}}%
\pgfpathlineto{\pgfqpoint{3.450589in}{4.047566in}}%
\pgfpathlineto{\pgfqpoint{3.459810in}{4.051156in}}%
\pgfpathlineto{\pgfqpoint{3.469031in}{4.047307in}}%
\pgfpathlineto{\pgfqpoint{3.487474in}{4.050596in}}%
\pgfpathlineto{\pgfqpoint{3.505916in}{4.046196in}}%
\pgfpathlineto{\pgfqpoint{3.515137in}{4.051633in}}%
\pgfpathlineto{\pgfqpoint{3.524358in}{4.049906in}}%
\pgfpathlineto{\pgfqpoint{3.533579in}{4.054306in}}%
\pgfpathlineto{\pgfqpoint{3.542800in}{4.051665in}}%
\pgfpathlineto{\pgfqpoint{3.561242in}{4.053734in}}%
\pgfpathlineto{\pgfqpoint{3.579684in}{4.049295in}}%
\pgfpathlineto{\pgfqpoint{3.588905in}{4.053158in}}%
\pgfpathlineto{\pgfqpoint{3.598126in}{4.048781in}}%
\pgfpathlineto{\pgfqpoint{3.607347in}{4.051903in}}%
\pgfpathlineto{\pgfqpoint{3.616568in}{4.050981in}}%
\pgfpathlineto{\pgfqpoint{3.625789in}{4.051567in}}%
\pgfpathlineto{\pgfqpoint{3.635010in}{4.049462in}}%
\pgfpathlineto{\pgfqpoint{3.671894in}{4.051242in}}%
\pgfpathlineto{\pgfqpoint{3.681115in}{4.054592in}}%
\pgfpathlineto{\pgfqpoint{3.690336in}{4.047524in}}%
\pgfpathlineto{\pgfqpoint{3.699558in}{4.045973in}}%
\pgfpathlineto{\pgfqpoint{3.708779in}{4.045585in}}%
\pgfpathlineto{\pgfqpoint{3.718000in}{4.051101in}}%
\pgfpathlineto{\pgfqpoint{3.736442in}{4.049843in}}%
\pgfpathlineto{\pgfqpoint{3.745663in}{4.051116in}}%
\pgfpathlineto{\pgfqpoint{3.754884in}{4.048090in}}%
\pgfpathlineto{\pgfqpoint{3.764105in}{4.050513in}}%
\pgfpathlineto{\pgfqpoint{3.773326in}{4.047061in}}%
\pgfpathlineto{\pgfqpoint{3.791768in}{4.044810in}}%
\pgfpathlineto{\pgfqpoint{3.800989in}{4.050352in}}%
\pgfpathlineto{\pgfqpoint{3.810210in}{4.049072in}}%
\pgfpathlineto{\pgfqpoint{3.819431in}{4.049709in}}%
\pgfpathlineto{\pgfqpoint{3.828652in}{4.051687in}}%
\pgfpathlineto{\pgfqpoint{3.837873in}{4.055874in}}%
\pgfpathlineto{\pgfqpoint{3.856315in}{4.050581in}}%
\pgfpathlineto{\pgfqpoint{3.865536in}{4.051158in}}%
\pgfpathlineto{\pgfqpoint{3.874757in}{4.053611in}}%
\pgfpathlineto{\pgfqpoint{3.883978in}{4.046904in}}%
\pgfpathlineto{\pgfqpoint{3.893199in}{4.051542in}}%
\pgfpathlineto{\pgfqpoint{3.902421in}{4.050373in}}%
\pgfpathlineto{\pgfqpoint{3.911642in}{4.053855in}}%
\pgfpathlineto{\pgfqpoint{3.930084in}{4.047107in}}%
\pgfpathlineto{\pgfqpoint{3.939305in}{4.053848in}}%
\pgfpathlineto{\pgfqpoint{3.948526in}{4.049388in}}%
\pgfpathlineto{\pgfqpoint{3.957747in}{4.052825in}}%
\pgfpathlineto{\pgfqpoint{3.966968in}{4.050021in}}%
\pgfpathlineto{\pgfqpoint{3.976189in}{4.049643in}}%
\pgfpathlineto{\pgfqpoint{3.985410in}{4.047154in}}%
\pgfpathlineto{\pgfqpoint{3.994631in}{4.053772in}}%
\pgfpathlineto{\pgfqpoint{4.003852in}{4.051160in}}%
\pgfpathlineto{\pgfqpoint{4.013073in}{4.049914in}}%
\pgfpathlineto{\pgfqpoint{4.022294in}{4.050670in}}%
\pgfpathlineto{\pgfqpoint{4.031515in}{4.048024in}}%
\pgfpathlineto{\pgfqpoint{4.049957in}{4.048395in}}%
\pgfpathlineto{\pgfqpoint{4.059178in}{4.050880in}}%
\pgfpathlineto{\pgfqpoint{4.068399in}{4.050862in}}%
\pgfpathlineto{\pgfqpoint{4.086841in}{4.052253in}}%
\pgfpathlineto{\pgfqpoint{4.105284in}{4.048425in}}%
\pgfpathlineto{\pgfqpoint{4.123726in}{4.052838in}}%
\pgfpathlineto{\pgfqpoint{4.132947in}{4.050584in}}%
\pgfpathlineto{\pgfqpoint{4.142168in}{4.049607in}}%
\pgfpathlineto{\pgfqpoint{4.151389in}{4.051980in}}%
\pgfpathlineto{\pgfqpoint{4.160610in}{4.050857in}}%
\pgfpathlineto{\pgfqpoint{4.179052in}{4.055283in}}%
\pgfpathlineto{\pgfqpoint{4.188273in}{4.050934in}}%
\pgfpathlineto{\pgfqpoint{4.197494in}{4.049154in}}%
\pgfpathlineto{\pgfqpoint{4.206715in}{4.050626in}}%
\pgfpathlineto{\pgfqpoint{4.215936in}{4.050632in}}%
\pgfpathlineto{\pgfqpoint{4.225157in}{4.053664in}}%
\pgfpathlineto{\pgfqpoint{4.243599in}{4.049966in}}%
\pgfpathlineto{\pgfqpoint{4.262041in}{4.050499in}}%
\pgfpathlineto{\pgfqpoint{4.271262in}{4.055885in}}%
\pgfpathlineto{\pgfqpoint{4.280483in}{4.049885in}}%
\pgfpathlineto{\pgfqpoint{4.289704in}{4.049938in}}%
\pgfpathlineto{\pgfqpoint{4.298925in}{4.051174in}}%
\pgfpathlineto{\pgfqpoint{4.317368in}{4.046731in}}%
\pgfpathlineto{\pgfqpoint{4.326589in}{4.052684in}}%
\pgfpathlineto{\pgfqpoint{4.335810in}{4.050204in}}%
\pgfpathlineto{\pgfqpoint{4.345031in}{4.051369in}}%
\pgfpathlineto{\pgfqpoint{4.363473in}{4.049282in}}%
\pgfpathlineto{\pgfqpoint{4.381915in}{4.052791in}}%
\pgfpathlineto{\pgfqpoint{4.391136in}{4.051005in}}%
\pgfpathlineto{\pgfqpoint{4.400357in}{4.052629in}}%
\pgfpathlineto{\pgfqpoint{4.418799in}{4.050723in}}%
\pgfpathlineto{\pgfqpoint{4.428020in}{4.048986in}}%
\pgfpathlineto{\pgfqpoint{4.437241in}{4.050314in}}%
\pgfpathlineto{\pgfqpoint{4.474125in}{4.051659in}}%
\pgfpathlineto{\pgfqpoint{4.483346in}{4.049098in}}%
\pgfpathlineto{\pgfqpoint{4.492567in}{4.049262in}}%
\pgfpathlineto{\pgfqpoint{4.501788in}{4.052990in}}%
\pgfpathlineto{\pgfqpoint{4.511009in}{4.050429in}}%
\pgfpathlineto{\pgfqpoint{4.520231in}{4.052004in}}%
\pgfpathlineto{\pgfqpoint{4.529452in}{4.051616in}}%
\pgfpathlineto{\pgfqpoint{4.538673in}{4.049647in}}%
\pgfpathlineto{\pgfqpoint{4.547894in}{4.054108in}}%
\pgfpathlineto{\pgfqpoint{4.557115in}{4.050736in}}%
\pgfpathlineto{\pgfqpoint{4.575557in}{4.053275in}}%
\pgfpathlineto{\pgfqpoint{4.584778in}{4.049269in}}%
\pgfpathlineto{\pgfqpoint{4.593999in}{4.051409in}}%
\pgfpathlineto{\pgfqpoint{4.603220in}{4.049600in}}%
\pgfpathlineto{\pgfqpoint{4.612441in}{4.051227in}}%
\pgfpathlineto{\pgfqpoint{4.621662in}{4.051024in}}%
\pgfpathlineto{\pgfqpoint{4.630883in}{4.052156in}}%
\pgfpathlineto{\pgfqpoint{4.649325in}{4.048750in}}%
\pgfpathlineto{\pgfqpoint{4.658546in}{4.052971in}}%
\pgfpathlineto{\pgfqpoint{4.676988in}{4.052713in}}%
\pgfpathlineto{\pgfqpoint{4.686209in}{4.049947in}}%
\pgfpathlineto{\pgfqpoint{4.695430in}{4.045499in}}%
\pgfpathlineto{\pgfqpoint{4.704651in}{4.050775in}}%
\pgfpathlineto{\pgfqpoint{4.713872in}{4.049653in}}%
\pgfpathlineto{\pgfqpoint{4.723094in}{4.054703in}}%
\pgfpathlineto{\pgfqpoint{4.732315in}{4.053599in}}%
\pgfpathlineto{\pgfqpoint{4.741536in}{4.045822in}}%
\pgfpathlineto{\pgfqpoint{4.750757in}{4.053086in}}%
\pgfpathlineto{\pgfqpoint{4.759978in}{4.050147in}}%
\pgfpathlineto{\pgfqpoint{4.769199in}{4.050802in}}%
\pgfpathlineto{\pgfqpoint{4.778420in}{4.052666in}}%
\pgfpathlineto{\pgfqpoint{4.787641in}{4.050159in}}%
\pgfpathlineto{\pgfqpoint{4.796862in}{4.049701in}}%
\pgfpathlineto{\pgfqpoint{4.806083in}{4.053167in}}%
\pgfpathlineto{\pgfqpoint{4.815304in}{4.052660in}}%
\pgfpathlineto{\pgfqpoint{4.842967in}{4.046870in}}%
\pgfpathlineto{\pgfqpoint{4.852188in}{4.051468in}}%
\pgfpathlineto{\pgfqpoint{4.861409in}{4.051862in}}%
\pgfpathlineto{\pgfqpoint{4.870630in}{4.055187in}}%
\pgfpathlineto{\pgfqpoint{4.879851in}{4.050570in}}%
\pgfpathlineto{\pgfqpoint{4.889072in}{4.049216in}}%
\pgfpathlineto{\pgfqpoint{4.907514in}{4.053235in}}%
\pgfpathlineto{\pgfqpoint{4.916735in}{4.047695in}}%
\pgfpathlineto{\pgfqpoint{4.935178in}{4.051465in}}%
\pgfpathlineto{\pgfqpoint{4.944399in}{4.048298in}}%
\pgfpathlineto{\pgfqpoint{4.953620in}{4.051689in}}%
\pgfpathlineto{\pgfqpoint{4.981283in}{4.050630in}}%
\pgfpathlineto{\pgfqpoint{4.990504in}{4.048028in}}%
\pgfpathlineto{\pgfqpoint{4.999725in}{4.047924in}}%
\pgfpathlineto{\pgfqpoint{5.008946in}{4.050135in}}%
\pgfpathlineto{\pgfqpoint{5.018167in}{4.053718in}}%
\pgfpathlineto{\pgfqpoint{5.027388in}{4.046143in}}%
\pgfpathlineto{\pgfqpoint{5.036609in}{4.052945in}}%
\pgfpathlineto{\pgfqpoint{5.045830in}{4.049684in}}%
\pgfpathlineto{\pgfqpoint{5.055051in}{4.052500in}}%
\pgfpathlineto{\pgfqpoint{5.064272in}{4.053201in}}%
\pgfpathlineto{\pgfqpoint{5.091935in}{4.047668in}}%
\pgfpathlineto{\pgfqpoint{5.101156in}{4.050939in}}%
\pgfpathlineto{\pgfqpoint{5.110377in}{4.049653in}}%
\pgfpathlineto{\pgfqpoint{5.128819in}{4.053652in}}%
\pgfpathlineto{\pgfqpoint{5.138041in}{4.052217in}}%
\pgfpathlineto{\pgfqpoint{5.156483in}{4.050914in}}%
\pgfpathlineto{\pgfqpoint{5.165704in}{4.049156in}}%
\pgfpathlineto{\pgfqpoint{5.174925in}{4.051747in}}%
\pgfpathlineto{\pgfqpoint{5.184146in}{4.050534in}}%
\pgfpathlineto{\pgfqpoint{5.193367in}{4.050505in}}%
\pgfpathlineto{\pgfqpoint{5.202588in}{4.052857in}}%
\pgfpathlineto{\pgfqpoint{5.211809in}{4.047265in}}%
\pgfpathlineto{\pgfqpoint{5.221030in}{4.051361in}}%
\pgfpathlineto{\pgfqpoint{5.239472in}{4.050792in}}%
\pgfpathlineto{\pgfqpoint{5.267135in}{4.052992in}}%
\pgfpathlineto{\pgfqpoint{5.276356in}{4.049862in}}%
\pgfpathlineto{\pgfqpoint{5.285577in}{4.053552in}}%
\pgfpathlineto{\pgfqpoint{5.294798in}{4.050621in}}%
\pgfpathlineto{\pgfqpoint{5.313240in}{4.051342in}}%
\pgfpathlineto{\pgfqpoint{5.322461in}{4.046982in}}%
\pgfpathlineto{\pgfqpoint{5.340904in}{4.051825in}}%
\pgfpathlineto{\pgfqpoint{5.359346in}{4.052514in}}%
\pgfpathlineto{\pgfqpoint{5.368567in}{4.049802in}}%
\pgfpathlineto{\pgfqpoint{5.377788in}{4.051514in}}%
\pgfpathlineto{\pgfqpoint{5.387009in}{4.054593in}}%
\pgfpathlineto{\pgfqpoint{5.396230in}{4.048781in}}%
\pgfpathlineto{\pgfqpoint{5.405451in}{4.048103in}}%
\pgfpathlineto{\pgfqpoint{5.414672in}{4.051908in}}%
\pgfpathlineto{\pgfqpoint{5.433114in}{4.050720in}}%
\pgfpathlineto{\pgfqpoint{5.442335in}{4.053306in}}%
\pgfpathlineto{\pgfqpoint{5.451556in}{4.051047in}}%
\pgfpathlineto{\pgfqpoint{5.460777in}{4.053248in}}%
\pgfpathlineto{\pgfqpoint{5.469998in}{4.048718in}}%
\pgfpathlineto{\pgfqpoint{5.479219in}{4.049269in}}%
\pgfpathlineto{\pgfqpoint{5.488440in}{4.046273in}}%
\pgfpathlineto{\pgfqpoint{5.497661in}{4.049382in}}%
\pgfpathlineto{\pgfqpoint{5.506882in}{4.047474in}}%
\pgfpathlineto{\pgfqpoint{5.516103in}{4.052434in}}%
\pgfpathlineto{\pgfqpoint{5.525324in}{4.050891in}}%
\pgfpathlineto{\pgfqpoint{5.534545in}{4.051467in}}%
\pgfpathlineto{\pgfqpoint{5.534545in}{4.051467in}}%
\pgfusepath{stroke}%
\end{pgfscope}%
\begin{pgfscope}%
\pgfsetrectcap%
\pgfsetmiterjoin%
\pgfsetlinewidth{0.803000pt}%
\definecolor{currentstroke}{rgb}{0.000000,0.000000,0.000000}%
\pgfsetstrokecolor{currentstroke}%
\pgfsetdash{}{0pt}%
\pgfpathmoveto{\pgfqpoint{0.800000in}{0.528000in}}%
\pgfpathlineto{\pgfqpoint{0.800000in}{4.224000in}}%
\pgfusepath{stroke}%
\end{pgfscope}%
\begin{pgfscope}%
\pgfsetrectcap%
\pgfsetmiterjoin%
\pgfsetlinewidth{0.803000pt}%
\definecolor{currentstroke}{rgb}{0.000000,0.000000,0.000000}%
\pgfsetstrokecolor{currentstroke}%
\pgfsetdash{}{0pt}%
\pgfpathmoveto{\pgfqpoint{5.760000in}{0.528000in}}%
\pgfpathlineto{\pgfqpoint{5.760000in}{4.224000in}}%
\pgfusepath{stroke}%
\end{pgfscope}%
\begin{pgfscope}%
\pgfsetrectcap%
\pgfsetmiterjoin%
\pgfsetlinewidth{0.803000pt}%
\definecolor{currentstroke}{rgb}{0.000000,0.000000,0.000000}%
\pgfsetstrokecolor{currentstroke}%
\pgfsetdash{}{0pt}%
\pgfpathmoveto{\pgfqpoint{0.800000in}{0.528000in}}%
\pgfpathlineto{\pgfqpoint{5.760000in}{0.528000in}}%
\pgfusepath{stroke}%
\end{pgfscope}%
\begin{pgfscope}%
\pgfsetrectcap%
\pgfsetmiterjoin%
\pgfsetlinewidth{0.803000pt}%
\definecolor{currentstroke}{rgb}{0.000000,0.000000,0.000000}%
\pgfsetstrokecolor{currentstroke}%
\pgfsetdash{}{0pt}%
\pgfpathmoveto{\pgfqpoint{0.800000in}{4.224000in}}%
\pgfpathlineto{\pgfqpoint{5.760000in}{4.224000in}}%
\pgfusepath{stroke}%
\end{pgfscope}%
\end{pgfpicture}%
\makeatother%
\endgroup%

			\caption{Estimation de $c$}
			\label{fig:exp_c}
		\end{figure}

		On trouve bien que $c$ tend vers une valeur constante.


\end{document}

